\chapter{Classifiers and Evaluation}\label{chap:classifier-evaluation}

We are going to introduce the classifier interface, discuss what
classifiers do, and then show how to evaluate them.  In subsequent
chapters, we consider a selection of the classifier implementations
offered in LingPipe.

\section{What is a Classifier?}

A classifier takes inputs, which could be just about anything, and
return a classification of the input over a finite number of discrete
categories.  For example, a classifier might take a biomedical
research paper's abstract and determine if it is about genomics or
not.  The outputs are ``about genomics'' and ``not about genomics.''

Classifiers can have more than two outputs.  For instance, a
classifier might take a newswire article and classify whether it is
politics, sports, entertainment, science and technology, or world or
local news; this is what Google and Bing's news services do.%
%
\footnote{At \url{http://news.google.com} and
  \url{http://news.bing.com}.}
%
Other applications of text classifiers, either alone or in concert
with other components, range from classifying documents by topic,
identifying the language of a snippet of text, analyzing whether a
movie review is postive or negative, linking a mention of a gene name
in a text to a database entry for that gene, or resolving the sense of
an ambiguous word like \stringmention{bank} as meaning a savings
institution or a flowing body of water.  In each of these cases, we
have a known finite set of outcomes and some evidence in the form of
text.

\subsection{Exclusive and Exhaustive Categories}

In the standard formulation of classification, which LingPipe follows,
the categories are taken to be both exhaustive and mutually exclusive.
Thus every item being classified has exactly one category.

One way to get around the exhaustiveness problem is to include an
``other'' category that matches any input that doesn't match one of
the other categories.  The other category is sometimes called a
``sink'' or ``trash'' category, and may be handled differently than
the other categories during training.

Exclusivity is more difficult to engineer.  While it is possible to
allow categories that represent more than one outcome, if we have $n$
base categories, we'll have ${n \choose 2}$ unique pairs of
categories.  The combinatorics quickly gets out of hand.

If an item needs to be classified for two cross-cutting
categorizations, we can use two classifiers, one for each
categorization.  For instance, we might classify MEDLINE citations as
being about genomics or not, and about being about clinical trials or
not.  The two binary classifiers produce four possible outcomes.
Another example of this would be to use two binary classifiers for
sentiment, one indicating if a text had positive sentiment or not, and
another indicating if it had negative sentiment or not.  The result is
a four-way classification, with a neutral article having neither
positive nor negative sentiment, a mixed review having both positive
and negative sentiment, and a positive or negative review having one
or the other.

The latent Dirichlet allocation (LDA) model we consider in
\refchap{lda} assigns more than one category to each document, under
the assumption that every document is a mixture of topics blended
according to some document-specific ratio that the model infers.

\subsection{First-Best, Ranked and Probabilistic Results}

A simple first-best classifier need only return its best guess for the
category of each input item.  Some applications only allow a single
best guess, and thus some evaluations are geared toward evaluating
only a classifiers' first-best guess for an input.

A ranking classifier returns the possible categories in rank order of
their match to the input.  The top ranked answer is the first-best
result.  We still assume exhaustive and exclusive categories, but the
classfier supplies its second-best guess, third-best guess and so on.
In cases where there are large numbers of categories, an application
might return more than one possible answer to a user.

A scoring classifier goes one step further and assigns a (floating
point) score to each categories.  These may then be sorted to provide
a ranking and a first-best result, with higher scores taken to be
better matches.  For example, LingPipe's implementations of averaged
perceptrons returns scored results.

Scores for categories given an input are often normalized so that they
represent an estimate of the conditional probability of a category
given the input.  This is the case for LingPipe's logistic regression
classifiers and k-nearest neighbors classifiers.  For instance, a
classifier might see a MEDLINE citation about testing for a disease
with a known genetic component and estimate a 40\% probability it is
about genomics and 60\% probability that it is not about genomics.
Because the outcomes are exhaustive and exclusive, the probabilities
assigned to the categories must sum to 1.

In the case of generative statistical models, such as naive Bayes
classifiers or hidden Markov models, the score represents the joint
probabilty of the output category and the input being classified.
Given the joint probabilities, we can use the rule of total
probability to compute conditional probabilities of categories given
inputs.

\subsection{Reductions to Classification Problems}

Many problems that don't at first blush appear to be classification
problems may be reduced to classification problems.  For instance, the
standard document search problem (see \refchap{lucene}) may be recast
as a classification problem.  Given a user query, such as
\searchquery{be-bim-bop recipe}, documents may be classified as
relevant or not-relevant to the search.  

Just about any problem that may be cast in a hypothesize-and-test
algorithm may use a classifier to do the testing, with a simple binary
outcome of accept or reject.  For instance, we can generate possible
named-entity mention chunkings of a text and then use a binary
classifier to evaluate if they are correct or not.  

Such reductions are often not interpetable probabilistically in the
sense of assigning probabilities to possible outcomes.

\subsection{Ordinals, Counts, and Scalars}

A classifier is an instance of what statisticians call categorical
models.  The outcome of a classification is a category, not a number.

Classifiers deal with categorical outcomes.  Ordinal outcomes are like
categorical outcomes, only they come with an order.  Examples of
ordinal outcomes include rating movies on a $\{ 1, 2, 3, 4, 5 \}$
scale, answering a survey question with strongly-disagree, disagree,
neutral, agree, or strongly agree, and rating a political attitude as
left, center, or right.

In many cases, we will be dealing with counts, which are non-negative
natural numbers.  Examples of count variables include the number of
times a word appears in a document, the length of a document's title
in characters, and the number of home runs a baseball player gets in a
season.  Classifiers like naive Bayes convert word count data into
categorical outcomes based on a probability model for documents (see
\refchap{naive-bayes}).

Scalar outcomes are typically continuous.  Examples of continuous
scalar variables include a person's height, the length of the vowel
sequence in milliseconds in the pronunciation of the word
\charmention{lion}, or the number of miles between two cities.  Wes

Rating movies on an ordinal 1-5 scale doesn't allow ratings like
1.793823.  Half-star ratings could be accomodated by including
additional discrete outcomes like 1.5, 2.5, 3.5 and 4.5, leaving a
total of 9 possible ratings.  When there are 100 possible ratings, it
makes less sense to treat the outcome as ordinal.  Often, it is
approximated by a continuous scalar variable.

Models that predict or estimate the values of ordinal, count and
scalar values all have different evaluation metrics than categorical
outcomes.  In this section, we focus exclusively on categorical
outcomes.

