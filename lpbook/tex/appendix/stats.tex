\chapter{Statistics}\label{chapter:stats}

This chapter provides an overview of the basic low-level statistical
utility classes and methods used by LingPipe.  It may make sense to
use this chapter more as a reference than to read it on the first
pass.

\section{Binomial Distribution}

\subsection{Distribution}

The binomial distribution is parameterized by a chance-of-success
parameter $\theta \in [0,1]$ and a number-of-trials parameter $N \in
\nats$.  If a random variable $X$ has a binomial distribution with
parameters $\theta$ and $N$, we write $X \sim \dbin{\theta}{N}$.  The
probability distribution for $X \sim \dbin{\theta}{N}$ is given by 
%
\begin{equation}
p_X(n) = {N \choose n} \ \theta^{n} \ (1-\theta)^{N-n}.
\end{equation}
%
The term ${N \choose n}$ is the binomial coefficient, defined 
for $N \geq n \geq 0$ by
%
\begin{equation}
{N \choose n} = \frac{N! \ n!}{(N - n)!}.
\end{equation}
%
Here, the term $n!$ is the factorial operation, defined for $n > 0$ by
%
\begin{equation}
n! = 1 \times 2 \times \cdots \times n = \prod_{m=1}^n m.
\end{equation}
%
The convention is to set $0! = 1$ so that the boundary conditions of
definitions like the one we provided for the binomial coefficient work
out properly.

\subsection{Variance}\label{sect:stats-binomial-variance}

The variance and standard deviation of a binomially-distributed random
variable $X \sim \dbin{\theta}{N}$ are given by
%
\begin{equation}
\hfill
\var{X} = \frac{\theta (1 - \theta)}{N} 
\hspace*{0.25in}
\mbox{and}
\hspace*{0.25in}
\sd{X} = \sqrt{\frac{\theta (1 - \theta)}{N}}.
\hfill
\end{equation}
%

We are often more interested in the percentage of the $N$ trials that
resulted in success, which is given by the random variable $X/N$.  The
variable $X/N$ is scaled to be between 0 and 1.  With large $N$,
$X/N$, will be approximately equal to $\theta$ for almost every trial.
The variance and standard deviation of $X/N$ are derived in the usual
way, by dividing by the constant $N$,
%
\begin{equation}
\var{X/N} = \theta (1 - \theta) 
\hspace*{0.25in}
\mbox{and}
\hspace*{0.25in}
\sd{X/N} = \sqrt{\theta (1-\theta)}.
\end{equation}



\section{Distribution Divergence Measures}\label{section:stats-divergence}

Kullback-Leibler (KL) divergence is the standard method to compare the
distributions of two random variables.  Pairs of topic distributions
that have similar probability assignments to words will have low KL
divergences from each other.

We will start with the discrete case, as it is what Griffiths and
Steyvers used in their experiments.  We'll consider the continuous
case, and specifically consider the Dirichlet distributions underlying
the LDA models.

\subsection{Defining KL Divergence}

As is standard, our definition will use random variables, but we point
out ahead of time that only the distributions of the variables matter,
not the outcomes.  Suppose that $X$ and $Y$ are two random variables
with possible outcomes ranging from 1 to $N$ and probabilty
distributions $p_X(n)$ and $p_Y(n)$.  The KL divergence of $X$ from
$Y$ is given by
%
\begin{equation}
\kld{X}{Y}
= \sum_{n=1}^N p_X(n) \log_2 \frac{p_X(n)}{p_Y(n)}.
\end{equation}
%
Because we used base-2 logs, the result is in units of bits.  

If there is an outcome $n$ where $p_Y(n) = 0$ and $p_X(n) > 0$, the
divergence is infinite.  We may allow $p_X(n) = 0$ by interpreting $0
\log_2 \frac{0}{p_Y(n)} = 0$ by convention (even if $p_Y(n) > 0$).

Although we do not provide a proof, we note that $\kld{X}{Y} \geq 0$
for all random variables $X$ and $Y$.  We further note without proof
that $\kld{X}{Y} = 0$ if and only if $p_X = p_Y$.

For example, suppose we have three discrete random variables, $X$, $Y$
and $Z$, each with the same two outcomes, and distributions defined by
$p_X(1) = 0.2$, $p_X(2) = 0.8$ and $p_Y(1) = 0.4$, $p_Y(2) = 0.6$.  We
calculate 
%
\[
\kld{X}{Y} = 0.2 \log_2 (0.2/0.4) + 0.8 \log_2 (0.8/0.6) \approx 0.13.
\]
%
Consider a third random variable, $Z$, with $p_Z(1) = 0.5$ and
$p_Z(2) = 0.5$.  The KL-divergence calculation is 
%
\[
\kld{X}{Z} = 0.2 \log_2 (0.2/0.5) + 0.8 \log_2 (0.8/0.5) \approx 0.27.
\]
As we'd expect, $Z$ diverges further from $X$ than $Y$.  
Comparing $X$ to itself, we get
%
\[
\kld{X}{X} = 0.2 \log_2 (0.2/0.2) + 0.8 \log_2(0.8/0.8) = 0.
\]  
%
From this example, it's easy to see why $\kld{X}{X}$ is always 0.

KL divergence may expressed using expectation notation as
%
\begin{align}
\kld{X}{Y} 
&= \expect{\log_2 \frac{p_X(X)}{p_Y(X)}}
\\
&= \expect{\log_2 p_X(X)} - \expect{\log_2 p_Y(X)}.
\end{align}
%
As usual, unmarked expectation notation presupposes the natural
distributions for random variables, here using the distribution $p_X$
for the random variable $X$.%
%
\footnote{The first term is just the entropy of $X$,
$\entropy{X} = \expect{\log_2 p_X(X)}$.  Another way of interpreting
KL divergence is as the expected penalty in bits for using $p_Y$ to
model values of $X$ rather than $p_X$.}

\subsection{Symmetrized KL Divergence}\label{section:stats-symmetrized-kl-divergence}

KL divergence is not symmetric in the sense that there exist pairs of
random variables $X$ and $Y$ such that $\kld{X}{Y} \neq \kld{Y}{X}$.
We have an example to hand.  Consider our example $X$ and $Y$ above,
for which we noted $\kld{X}{Y} \approx 0.13$.  The other way around,
$\kld{Y}{X} = 0.4 \log_2 0.4/0.2 + 0.6 \log_2 0.6/0.8 \approx 0.15$.

There are several divergence measures derived from KL divergence that
are symmetric.  The simplest approach is just to introduce symmetry
by brute force.  The symmetrized KL-divergence between random variables
$X$ and $Y$ is defined by averaging KL divergence of $X$ from $Y$
and of $Y$ from $X$,
%
\begin{equation}
\skld{X}{Y} = \frac{1}{2}(\kld{X}{Y} + \kld{Y}{X}).
\end{equation}
%
Obviously, $\skld{X}{Y} = \skld{Y}{X}$, so the measure is symmetric.


\subsection{Jensen-Shannon Divergence}

Another widely used symmetric divergence measure derived from KL divergence
is the Jensen-Shannon divergence.  To compare $X$ and $Y$ using Jensen-Shannon
divergence, we first define a new random variable $Z$ with distribution defined by
averaging the distributions of $X$ and $Y$,
%
\[
p_Z(n) = \frac{1}{2}(p_X(n) + p_Y(n)).
\]
%
We then define Jensen-Shannon divergence by taking average divergence
from $Z$,
%
\begin{equation}
\jsd{X}{Y} = \frac{1}{2}(\kld{X}{Z} + \kld{Y}{Z}).
\end{equation}
%
As with the symmetrized KL divergence, Jensen-Shannon divergence is
symmetric by definition.


\subsection{LingPipe KL-Divergence Utilities}

KL-divergence is implemented as a static utility method in the
\code{Statistics} utility class in package \code{com.aliasi.stats}.
The method takes two double arrays representing probability
distributions and measures how much the first is like the second.

In order to give you a sense of KL-divergence, we implement a simple
utility in the demo class \code{KlDivergence} to let you try various
values.  The code just parses out double arrays from teh command line
and sends them to the KL function.
%
\codeblock{KlDivergence.1}

The ant target \code{kl} runs the command with the first two arguments
given by properties \code{p.x} and \code{p.y}.  To calculate the example
from above, we have
%
\commandlinefollow{ant -Dp.x="0.2 0.8" -Dp.y="0.4 0.6" kl}
\begin{verbatim}
pX=(0.2, 0.8)     pY=(0.4, 0.6)     
kld=0.13202999942307514    
skld=0.1415037499278844   
jsd=0.03485155455967723
\end{verbatim}
%
The Jensen-Shannon divergence is less than the symmetrized KL
divergence because the interpolated distribution $p_Z$ is closer to
both of the original distributions $p_X$ and $p_Y$ than they are to
each other.

