\mychapter{Installation}{INSTALLATION}\label{appendix:installation}

\noindent
\firstchar{T}his appendix explains how to install everything you'll
ned to work with the code in this book.

\mysection{Unix Shell Tools}

\noindent
We present examples as run from the shell through the Ant build
environment or directly.  We use Unix-style commands, which should
be familiar to anyone working in 
a Unix, Linux, or Mac OS X environment.

\mysubsection{Cygwin for Windows}

\noindent
If you're working in Windows (XP, Vista or 7), we recommend the 
Cygwin suite of Unix command-line tools for Windows.  They may be
downloaded and installed through Cygwin's home page,

\begin{quote}
\hrefurl{http://www.cygwin.com/}
\end{quote}

The {\tt setup.exe} program is small.  When you run it, it goes out
over the internet to find the packages from registered mirrors.  It
then lists all the packages available.  You can install some or all of
them.  We  Just
run the setup program again 

\mysubsection{Archive and Compression Tools}

\noindent
We will assume you can run {\tt tar} archiving tool, as well as the
unpacking commands {\tt unzip} and {\tt gunizp}.  These may be
installed as part of Cygwin on Windows.

\mysection{Version Control}

\noindent
If you don't live in some kind of version control environment, you should.
Not only can you keep track of your own code across multiple sites and/or
users, you can also keep up to date with all the LingPipe releases and
code from free online version control repositories such as SourceForge.  

We prefer the shell for all of our work, and you can install a
shell-based version of Subversion, the command-name for which is {\tt
svn}.  SVN itself requires a secure shell (SSH) client over which to
run.  These may both be installed through Cygwin for Windows users.


\mysection{Text Editors}

You will need to be able to edit text.  We like to work in the emacs
text editor, which contains too many features to begin to mention here.  
It's as close as you'll get to an IDE in a simple text editor.   We
use the XEmacs distribution of emacs, which is available from its home page,

\begin{quote}
\hrefurl{http://www.xemacs.org/}
\end{quote}


There's an installer for Windows.  We found the install from {\tt
xemacs.org} to work better than the one that comes with Cygwin as far
as interacting with Windows.  We like to work with the Lucida Console
font, which is distributed with Windows; it's the font used for examples
in this textbook.


\mysection{Integrated Development Environment}

\noindent
LingPipe development may be carried out through an integrated
development environment (IDE) that supports Java.  The two most
popular ones are Eclipse and NetBeans.  

\mysubsection{Eclipse Development Environment}

\noindent
Eclipse provides a full range of code checking (lint), auto-completion
and code generation, and debugging facilities.  Eclipse is an open source
project, and a wide range of additional tools are available as plugins.  It
also has modules for languages other than Java, such as C++ and PHP.

The full set of Eclipse downloads is listed on the following page,
%
\begin{quote}
\hrefurl{http://download.eclipse.org/eclipse/downloads/}
\end{quote}

You'll want to make sure you choose the one compatible with the JDK
you are using.  It's mainly designed to work on Windows, but has been
ported to Linux and Mac OS X (though Carbon).


\mysubsection{NetBeans}

\noindent
Unlike Eclipse, the NetBeans IDE is written entirely in Java.  Thus
it's possible to run it under Windows, Linux, Solaris Unix, and Mac OS
X. There is also a wide range of plug-ins available for NetBeans.

You can get NetBeans for free from its home page,
%
\begin{quote}
\hrefurl{http://netbeans.org/}
\end{quote}
%



\mysection{Java Standard Edition 6}

\noindent
The programs in this book are based on Java Standard Edition Version
6.  At the time of writing, this is the only version of Java being
publicly supported by Sun/Oracle.

In order to run the Java programs and their
associated libraries, you will need a Java runtime environment (JRE),
which contanis a Java virtual machine (JVM).  In order to compile Java
programs, you will need a Java development kit (JDK).

Java is available in 32-bit and 64-bit versions.  The 64-bit version
is required to allocate JVMs with heaps larger than 1.5 or 2 gigabytes
(the exact maximum for 32-bit Java depends on the platform).  

Licensing terms for the JRE are at
%
\begin{quote}
\hrefurl{http://www.java.com/en/download/license.jsp}{}
\end{quote}
and for the JDK at
%
\begin{quote}
\hrefurl{http://java.sun.com/javase/6/jdk-6u20-license.txt}
\end{quote}

Java is available for the Windows, Linux, and Solaris operating
systems in both 32-bit and 64-bit versions from
%
\begin{quote}
\hrefurl{http://java.sun.com/javase/downloads/index.jsp}
\end{quote}
%
The Java JDK and JRE are included as part of Mac OS X.  Updates are
available through
%
\begin{quote}
\hrefurl{http://developer.apple.com/java/download/}
\end{quote}
%
Java is updated regularly and it's worth having the latest version.
Updates include bug fixes and often include performance enhancements,
some of which can be quite substantial.

Java must be installed on the operating system so that shell commands
will execute it.  We have to manage multiple versions of Java, so
typically we will define an environment variable {\tt JAVA\_HOME}, and
add {\tt \$\{JAVA\_HOME\}/bin} (Unix) or {\tt \%JAVA\_HOME\%{\textbackslash}bin} (Windows) 
to the {\tt PATH}.  We then set {\tt JAVA\_HOME} to either {\tt
JAVA\_1\_5}, {\tt JAVA\_1\_6}, or {\tt JAVA\_1\_7} depending on use
case.  Note that {\tt JAVA\_HOME} is one level above Java's {\tt bin}
directory containing the executable Java commands.

You can test this with the following command, which
should produce similar results.

\begin{quote}
{\small
\ttfamily
{\it\ttfamily > java -version}
\\
java version "1.6.0\_18"
\\
Java(TM) SE Runtime Environment (build 1.6.0\_18-b07)
\\
Java HotSpot(TM) 64-Bit Server VM (build 16.0-b13, mixed mode)
}
\end{quote}


\mysection{Ant}

\noindent
The build files for the code samples and examples are provided as Ant
files.  Ant build files are coded in XML and use easy to read element
and attribute names for build targets.  Unlike other build systems
such as Make or Maven which try to resolve dependencies and determine
if targets are up to date, Ant always executes all of a target,
including all of the subtargets.

Ant is is an Apache project, and as such, is subject to the Apache license,
\begin{quote}
\hrefurl{http://ant.apache.org/license.html}
\end{quote}
%
Ant is available from 
%
\begin{quote}
\hrefurl{http://ant.apache.org/}
\end{quote}
%
You only need one of the binary distributions, which will
look like {\ttfamily apache-ant-{\it\ttfamily version}-bin.tar.gz}.

First, you need to unpack the distribution.  We like directory
structures with release names, which is how ant unpacks, using
top-level directory names like {\tt apache-ant-1.8.1}.  Then, you need
to put the {\tt bin} subdirectory of the top-level directory into the
{\tt PATH} environment variable so that Ant may be executed from the
command line.

Ant requires the {\tt JAVA\_HOME} environment variable to be set to
the path above the {\tt bin} directory containing the Java
executables.  

Ant's installation instructions suggest setting the {\tt
ANT\_HOME} directory in the same way, and then adding but it's not necessary unless
you will be scripting calls to Ant.

Ant build files may be imported directly into the Eclipse IDE.

\mysection{LingPipe}

\noindent
LingPipe may be downloaded from its web site:
%
\begin{quote}
\hrefurl{http://alias-i.com/lingpipe/}
\end{quote}
%
Other than unpacking, the gzipped tarball, there is nothing required
for installation.  Downloading LingPipe is not technically necessary for
running the examples in this book because the LingPipe library jar is
included with the book's source download.


\mysection{Book Source and Libraries}

\noindent
The code samples from this book are available via anonymous subversion checkout
from the LingPipe sandbox.  Specifically, the entire content of the book,
including code and text, may be checked out anonymously using,

\begin{quote}
{\small\ttfamily
> svn co https://aliasi.devguard.com/svn/sandbox/lpbook
}
\end{quote}
%
The distribution contains all of the code and Ant build files for
running it.  It also contains the \LaTeX\ source for the book itself
as well as the system to extract text from the source for automatic
inclusion in the book.
