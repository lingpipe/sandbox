\chapter{Latent Dirichlet Allocation}\label{chapter:lda}

Latent Dirichlet allocation (LDA) provides a generative statistical
model of an entire corpus of text documents.  At the basis of the LDA
model is the notion that each document is generated from its own
mixture of topics.  For instance, a post on a blog about a dinner
party might 70\% about cooking, 20\% about sustainable farming, and
10\% about entertaining.

Applications of LDA include document similarity measurement, document
clustering, topic discovery, word association for search, feature
extraction for discriminative models such as logistic regression or
conditional random fields, classification of new documents, language
modeling, and many more.

\section{Corpora and Documents}

For LDA, a corpus is nothing more than a collection of documents.
Each document is modeled as a sequence of words (or tokens).  It turns
out that only the count of the words is significant, not their order.

A bag is a data structure like a set, only with (non-negative) counts.
We can think of a bag as a mapping from tokens to their counts. Like
other models that don't distinguish word order, LDA is called a
bag-of-words model.

We often start with a real document collection consisting of character
data.  LingPipe uses tokenizer factories to convert character
sequences to sequences of tokens (see \refchap{tokenization}).
Utilities in the LDA implementation class help convert these to
appropriate representations with the help of symbol tabls (see
\refchap{symbol-tables}).

\subsection{Generalized ``Documents''}

Although LDA is framed in terms of documents and words, it turns out
it only uses the identities of the tokens.  As such, it may be applied
to collections of ``documents'' consisting of any kind of count data,
not just bags of words.  For instance, LDA may be applied to RNA
expression data.


\section{The Generative Model}

LDA is a kind of probabilistic model known as a generative model.
Generative models provide step-by-step characterizations of how to
generate a corpus of documents.%
%
\footnote{LDA does not model the number of documents in a corpus
or the number of tokens in each document.  These values must be
provided as constants.  This is usually not a problem because
the corpus is given as data.}
%
This setup seems strange to many people at first because we are in
practice presenting LDA with a collection of documents and asking it
to infer the topic model, not the other way around.  Nevertheless, the
probabilistic model is generative.

LDA generates each document independently based on the model
parameters.  To generate a document, LDA first generates a topic
distribution for that document.%
%
\footnote{The topic distribution for a document is generated from
a Dirichlet distribution, which is where the model gets its name.}
%
This happens before any of the words in a document are generated.  At
this stage, we might only know that a document is 70\% about politics
and 30\% about the economy, not what words it contains.

After generating the topic distribution for a document, we generate
the words for the document.  

In the case of LDA, we suppose that the number of documents and the
number of tokens in each document is given.  This means they're
not part of 

That means we don't try
to predict how long each document is or use document-length
information for relating documents.  Furthermore, LDA does not We also do not attempt to model
the size of the corpus.


contain are known, but 


LDA is a probabilistic topic (or factor) model.  Historically, LDA
evolved as a Bayesian generalization of the probabilistic latent
semantic indexing (pLSI) model.  pLSI was itself motivated as a
properly discretized probabilistic form of latent semantic indexing
(LSI).  LingPipe's implementation of the LSI factor model is covered
in \refchap{svd}.

The topic model is latent \ie{unknown}.

LingPipe's implementation of LDA is able to start from a corpus of
documents and infer the set of topics in the corpus, the topic
