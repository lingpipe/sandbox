\chapter{Corpora}

In this appendix, we list the corpora that we use for examples in the
text.

\section{20 Newsgroups}

\subsubsection{Contents}

The corpus contains roughly 20,000 posts to 20 different newsgroups.
These are

\begin{center}
\footnotesize
\begin{tabular}{p{0.3\textwidth}p{0.3\textwidth}p{0.3\textwidth}}
\path{comp.graphics} & \path{rec.autos} & \path{sci.crypt}
\\
\path{comp.os.ms-windows.misc} & \path{rec.motorcycles} & \path{sci.electronics}
\\
\path{comp.sys.ibm.pc.hardware} & \path{rec.sport.baseball} & \path{sci.med}
\\
\path{comp.sys.mac.hardware} & \path{rec.sport.hockey} & \path{sci.space}
\\
\path{comp.windows.x}
\\[12pt]
\path{misc.forsale} & \path{talk.politics.misc} & \path{talk.religion.misc}
\\
& \path{talk.politics.guns} & \path{alt.atheism}
\\
& \path{talk.politics.mideast} & \path{soc.religion.christian}
\end{tabular}
\end{center}


\subsubsection{Authors}

It is currently being maintained by Jason Rennie, who speculates that
Ken Lang may be the original curator.

\subsubsection{Licensing}

There are no licensing terms listed and the data may be downloaded
directly.

\subsubsection{Download}

\url{http://people.csail.mit.edu/jrennie/20Newsgroups/}




\section{MedTag}\label{section:corpora-medtag}

\subsubsection{Contents}

The MedTag corpus has two components, MedPost, a part-of-speech-tagged
corpus, and GENETAG, a gene/protein mention tagged corpus.  The data
annotated consists of sentences, and sentence fragments from MEDLINE
citation titles and abstracts.

\subsubsection{Authors}

The MedPost corpus was developed at the U.~S.~National Center for
Biotechnology Information (NCBI), a part of the National Library of
Medicine (NLM), which is itself one of the National Institutes of
Health (NIH).  Its contents are described in
%
\begin{quote}
Smith, L.~H., L.~Tanabe, T.~Rindflesch, and W.~J.~Wilbur.
2005. MedTag: a collection of biomedical annotations.  In {\it
  Proceedings of the ACL-ISMB Workshop on Linking Biological
  Literature, Ontologies and Databases: Mining Biological
  Semantics}. 32--37.
\end{quote}

\subsubsection{Licensing}

The MedPost corpus is a public domain corpus released as a ``United
States Government Work.''

\subsubsection{Download}

\url{ftp://ftp.ncbi.nlm.nih.gov/pub/lsmith/MedTag/medtag.tar.gz}




\section{WormBase MEDLINE Citations}\label{section:corpora-wormbase}

The WormBase web site, \url{http://wormbase.org}, curates and distributes
resources related to the model organism {\it Caenorhabditis elegans},
the nematode worm.  Part of that distribution is a set of citation to
published research on {\it C.~elegans}.  These literature
distributions consist of MEDLINE citations in a simple line-oriented
format.

\subsubsection{Authors}

Wormbase is a collaboration among many cites with many supporters.
It is described in 
%
\begin{quote}
Harris, T.~W. et al. 2010. WormBase: A comprehensive resource for
nematode research. {\it Nucleic Acids Research} {\bf 38}.
\doi{10.1093/nar/gkp952}
\end{quote}

\subsubsection{Download}

\url{ftp://ftp.wormbase.org/pub/wormbase/misc/literature/2007-12-01-wormbase-literature.endnote.gz} (15MB)

