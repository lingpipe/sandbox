\mychapter{Introduction}{INTRODUCTION}

\firstchar{T}his book provides a comprehensive description of how
LingPipe may be used to build natural language processing
applications.  LingPipe is an interconnected suite of natural language
processing software implemented in Java and accessed through its
application programming interface (API).  LingPipe's interfaces are
tailored to abstract over low-level implementation details to enable
components such as tokenizers or classifiers to be swapped in a
plug-and-play fashion.

The presentation here will be hands on and assume the reader is
comfortable reading Java programs.  idioms and higher-level design
patterns will also be presupposed, though trickier aspects of both
these and Java itself will be explained in some detail.

LingPipe contains a mixture of heuristic rule-based components and
statistical components, often implementing the same interfaces, as for
statistical phrase chunkers and regular-expression-based chunkers.

Familiarity with statistical notions such as probability will be
assumed in the description of some statistical models and their
associated algorithms and tuning parameters, but will not be assumed
throughout.  This is not intended to be a statistics or
machine-learning textbook.

We do not presuppose any knowledge of linguistics beyond a simple
understanding of the terms used in dictionaries such as words,
syllables, pronunciations, and parts of speech such as noun and
preposition.



\mysection{The Java Programming Language}

\noindent
We chose Java as the basis for LingPipe because we felt it provided
the best tradeoff among efficiency, usability, portability, and
library availability.

The presentation here assumes the reader has a basic working knowledge
of the Java programming language.  We will focus on a few aspects of
Java that are particularly crucial for processing textual language
data, such as character and string representations, input and output
streams and character encodings, regular expressions, parsing HTML and
XML markup, etc.  In explaining LingPipe's design, we will also delve
into greater detail on general features of Java such as concurrency,
generics, floating point representations, and the collection package.

\mysubsection{Java Standard Edition 6}

\noindent
This book is based on the latest currently supported standard edition
of the Java platform (Java SE), which is version 6.  You will need the
Java development kit (JDK) in order to compile Java programs.  A java
virtual machine (JVM) is required to execute compiled Java programs.
A Java runtime environment (JRE) contains platform-specific support
and integration for a JVM and often interfaces to web browsers for
applet support.

\mysubsection{Downloading Java}

\noindent
Java is available for the Windows, Linux, and Solaris operating
systems in both 32-bit and 64-bit versions.  The 64-bit version is
required to allocate JVMs with heaps larger than 1.5 or 2 gigabytes
(depending on platform).  The JDK, JRE and JVM are updated regularly.
The latest version for Windows, Linux, and Solaris is available from
%
\begin{quote}
\hrefurl{http://java.sun.com/javase/downloads/index.jsp}
\end{quote}
%
The Java JDK and JRE are included as part of Mac OS X.  Updates are
available through
%
\begin{quote}
\hrefurl{http://developer.apple.com/java/download/}
\end{quote}
%
Licensing terms for the JDK are provided at
%
\begin{quote}
\hrefurl{http://java.sun.com/javase/6/jdk-6u20-license.txt}
\end{quote}
%
and for the JRE at
%
\begin{quote}
\hrefurl{http://www.java.com/en/download/license.jsp}{}
\end{quote}

\mysubsection{Learning Java}

\noindent
For learning Java, we recommend the following two books pitched at
beginners and intermediate Java programmers,
%
\begin{itemize}
\item Arnold, Ken, James Gosling and David Holmes.  2005.
{\it The Java Programming Language}, 4th Edition.  Prentice-Hall.
\item Bloch, Joshua. 2008. {\it Effective Java}, 2nd Edition.  Prentice-Hall.
\end{itemize}
%
James Gosling designed the first version of Java and Joshua Bloch was
one of the main architects for the language after that.

We also highly recommend the algorithm section of the TopCoder
contests.  TopCoder provides a complete online Java environment, well
thought out problems at varying specified levels of difficulty, unit
tests that check your submissions, and examples of what others did to
provide code-reading practice.
%
\begin{quote}
\hrefurl{http://www.topcoder.com/tc}
\end{quote}
%





\mysection{Downloading LingPipe}

\noindent
LingPipe may be downloaded from its web site:

\begin{quote}
\hrefurl{http://alias-i.com/lingpipe/}
\end{quote}
