\chapter{Character Data and the Web}\label{chapter:web}

In this chapter we examine the intersection of Java character encoding 
and web technologies by showing the processing steps involved 
in sending data via HTTP between Java programs.
How is character data sent via HTTP?
Unask the question: character data isn't sent via HTTP; \emph{bytes} are sent via HTTP.
It is up to the client and server programs to map these bytes to characters.
As we stated in \refsec{section:char-problems},
data sent and received across a socket is a stream of bytes.
Without knowing the character set and encoding scheme used to generate
these bytes, we cannot reliably interpret the data.

The example programs for this chapter are in the code directory:
%
\displ{\filepath{src/chars/src/com/lingpipe/book/http}}
%
For some of these examples you will need to have a local web server and servlet container.
We use the Apache Tomcat 7 running as a standalone web server.
See the instructions on downloading and installing Tomcat in section \refsec{into-tomcat}.
These examples use classes in the Java APIs \code{javax.servlet} and \code{javax.servlet.http}.
Implementations of these APIs are not distributed as part of the Java 7 core distribution.
Instead we use the implementation provided by Tomcat.
To compile these classes, then, we need to edit the build.xml file and set the
property \code{\charmention{\$}tomcat\_home} accordingly.

\section{The HTTP Protocol}

The HTTP protocol is used to send data between web servers and clients.
Web browers are but one kind of client.
Data can be sent between many web services.
Data interchange formats developed for data feeds include XML formats such as
RSS, ATOM, as well as JSON, the Javascript Object Notation.

An HTTP message consists of a header and an optional message body.
HTTP/1.1 is version of protocol in common use today.

The header consists of a series of fields.
A field is a key-value pair in clear text format separated by a colon character
and terminated by a carriage-return line-feed (CR-LF).
A field may have multiple values, in which case the values are separated by spaces,
therefore a value may not contain spaces.
The key must be ASCII text.
Non-ASCII values are allowed provided they are encoded.%
%
\footnote{See Internet Engineering Task Force (IETF) RFC 5987 for details.}
%
The header is terminated by a blank field, that is, two consecutive CR-LF sequences.
The message body, if any, follows the header.

The header fields of interest are the \code{Content-Length} and \code{Content-Type} fields.
The \code{Content-Length} field specifies the number of bytes in the body.
The \code{Content-Type} header consists of a MIME type optionally followed by 
a semi-colon followed by parameters of the form \code{attribute=value}.
The following example is taken from the HTTP/1.1 specification.
\begin{verbatim}
Content-Type: text/html; charset=ISO-8859-4
\end{verbatim}
This example specifies that message body is HTML encoded using
ISO-8859-4, the Latin-4 encoding set.
The HTTP/1.1 protocol specifies that if the \code{Content-Type} is missing from the header,
the default MIME type is text/plain and the default
character encoding is ISO-8859-1, Latin-1 (see \refsec{latin1}).

The \code{java.net} package contains classes used to implement networking applications.
We start by using \code{java.net.URL} and \code{java.net.URLConnection} objects
to send and recieve HTTP requests and responses via the example class \code{EchoHttpHeader}.
%
\displ{\filepath{src/regex/src/com/lingpipe/book/webchars/EchoHttpHeader.java}}
%
\codeblock{EchoHttpHeader.1}
%
First we create a URL object from the first command-line argument.
If the argument doesn't start with a known protocal such as
\code{http}, \code{ftp}, \code{file}, \code{jar}, or if it is null,
the constructor will throw a \code{MalformedURLException}.
Next we create an \code{URLConnection} by calling the \code{openConnection()}
method on the \code{URL} object.
Despite its name, this method doesn't connect to the specified website, it just
creates a URLConnection object, to which setup parameters and and general 
request properties can be modified and added.
When the \code{connect()} method is invoked, the request is sent and the remote object
becomes available.  
%
\codeblock{EchoHttpHeader.2}
%
The method \code{getHeaderFields} parses the HTTP header into a map of
field names to values.
The first line of the HTTP header is the Status-Line which gives the
HTTP-version, status-code, and  reason-phrase, e.g.
\code{HTTP/1.1 200 OK}.
When we run this program with the wikipedia home page as the URL,
it returns the following headers
%
\commandlinefollow{ant -Durl="http://www.wikipedia.org" http-headers}
\begin{verbatim} 
field name: null	value: HTTP/1.0 200 OK 
field name: Age	value: 2
field name: Content-Length	value: 46388
field name: Last-Modified	value: Fri, 21 Dec 2012 11:10:33 GMT
field name: X-Cache-Lookup	values : HIT from cp1013.eqiad.wmnet:...
field name: Connection	value: keep-alive
field name: Server	value: Apache
field name: X-Cache	values : HIT from cp1013.eqiad.wmnet, MISS ...
field name: Cache-Control	value: s-maxage=3600, must-revalidate, ...
field name: X-Content-Type-Options	value: nosniff
field name: Date	value: Thu, 10 Jan 2013 23:25:36 GMT
field name: Vary	value: Accept-Encoding
field name: Content-Type	value: text/html; charset=utf-8
\end{verbatim}
%
This page is the universal wikipedia page which consists of links to all
the language specific-wikipedias.
Next we request the Japanese wikipedia hompage:
%
\commandlinefollow{ant -Durl="http://ja.wikipedia.org" http-headers}
\begin{verbatim}
field name: null	value: HTTP/1.0 200 OK 
field name: Age	value: 67
field name: Content-Language	value: ja
field name: Content-Length	value: 85505
field name: Last-Modified	value: Thu, 10 Jan 2013 15:15:07 GMT
field name: Connection	value: keep-alive
field name: X-Cache-Lookup	values : MISS from cp1005.eqiad.wmn ...
field name: X-Cache	values : MISS from cp1005.eqiad.wmnet, HIT ...
field name: Server	value: Apache
field name: X-Content-Type-Options	value: nosniff
field name: Cache-Control	value: private, s-maxage=0, max-age=0, ...
field name: Date	value: Thu, 10 Jan 2013 23:03:04 GMT
field name: Vary	value: Accept-Encoding,Cookie
field name: Content-Type	value: text/html; charset=UTF-8
\end{verbatim}
%
The field \code{Content-Language} is specified for the Japanese home page
but not for the universal Wikipedia page.  The contents of these pages are different
and this is reflected in the different values for the \code{Content-Length} field.
Both pages have the same Content-Type value: \code{text/html; charset=UTF-8}.
Using a Unicode charset means that almost all widely used characters in most of the
worlds languages can be used.
On my browser, most of the wikipedia languages display in thier proper font, excepting
Dhivehi, Gutisk, and a few others.

We use the header information to process the message body.
The example class \code{EchoHttpBody} retrieve the bytes from the message body and
uses the \code{charset} parameter in the \code{Content-Type} field to convert them
to characters.  
%
\displ{\filepath{src/regex/src/com/lingpipe/book/webchars/EchoHttpBody.java}}
%
This program reads in the URL from the command line and creates a \code{URLConnection},
as in the previous example.  Next it looks for the \code{Content-Type} field and
parses out the charset name.
%
\codeblock{EchoHttpBody.2}
%
The convenience method \code{URLConnection.getContentType()} returns the entire contents
of the \code{Content-Type} field.
We use a simple regex to extract the value of the charset parameter.
If no header field is present or the charset parameter is not specified
then the encoding defaults to Latin-1.
%
\codeblock{EchoHttpBody.3}
%
Next we read in the bytes from the body and convert them to a string
using the specified charset to correctly map bytes to characters.
%
\codeblock{EchoHttpBody.4}
%
In order to write this string to the terminal we need to use a \code{java.io.Writer}
that has the same character encoding as that of the string.
We construct one using Java's \code{Standard.out}, which is a \code{PrintStream}
that writes to the standard output stream.
The resulting \code{PrintWriter} will print the characters correctly.
Whether or not they display correctly depends on the characteristics of the terminal program
and the available fonts.

To demonstrate, we use this program to fetch the French Wikipedia home page.
Using a Mac terminal.app that is configured to 






\begin{verbatim} 
URL: http://fr.wikipedia.org
Content-Type: text/html; charset=UTF-8
charset: UTF-8
HTTP Response body:
<!DOCTYPE html>
<html lang="fr" dir="ltr" class="client-nojs">
<head>
<title>Wikipédia, l'encyclopédie libre</title>
\end{verbatim}







\section{Data interchange formats}




XML and JSON are common machine and human readable text data interchange formats. 
The purpose of text data interchange formats is not to frustrate the programmer,
rather it is to overcome the problem of proprietary data formats.

When natural language texts are represented as structured documents,
must first recover the document structure as well as the text contents.
This processing may be done either by treating the document as an object
which is fully constructed in memor, the DOM (Document Object Model) approach,
or a streaming approach in which subparts of the documents are processed as
they are encountered.  The streaming approach is the one we consider here.

event-based parsing using callbacks.
parsers keep track of markup, data
events are when markup is encountered

Applications implement handlers 


\section{XML}

Java and XML book.

SAX is a streaming interface for XML, which means that applications using SAX receive event notifications about the XML document being processed an element, and attribute, at a time in sequential order starting at the top of the document, and ending with the closing of the ROOT element. This means that it’s extremely efficient at processing XML in linear time without placing too many demands upon system memory.


\section{JSON}

simple json

\section{HTML}

Processing HTML is challenging because although standards
have been developed for HTML, they are often not met.
Therefore applications which mine HTML documents for text data
must be able to handle ill-formed HTML.
When we need to extract text data from an HTML document,
we use the NekoHTML parser.

http://nekohtml.sourceforge.net/

