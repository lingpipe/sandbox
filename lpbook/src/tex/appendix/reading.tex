\mychapter{Further Reading}{FURTHER READING}\label{appendix:reading}

\noindent
\firstchar{I}n order to fully appreciate contemporary approaches to natural language
processing requires three fairly broad areas of expertise:
linguistics, statistics, and algorithms.  We don't pretend to
introduce these areas in any depth in this book.  Instead, we
recommend the following textbooks, sorted by area, from among the many
that are currently available.

\mysection{Java}

\noindent
For learning Java, we recommend the following two books pitched at
beginners and intermediate Java programmers,
%
\begin{itemize}
\item Arnold, Ken, James Gosling and David Holmes.  2005.
{\it The Java Programming Language}, 4th Edition.  Prentice-Hall.
\\
{\footnotesize Gosling invented Java, and this is the ``official'' introduction to the
language.  It's also very good.}
%
\item Bloch, Joshua. 2008. {\it Effective Java}, 2nd Edition.  Prentice-Hall.
\\
{\footnotesize Bloch was one of the main architects of Java from
version 1.1 forward.  This book is about taking Java program design
to the next level.}
%
\item
Fowler, Martin, Kent Beck, John Brant, William Opdyke, and Don Roberts.
1999.
{\it Refactoring: Improving the Design of Existing Code}.
Addison-Wesley.
\\
{\footnotesize Great book on making code better, both existing
code and code yet to be created.}
\end{itemize}
%

\mysubsection{Books on Java Libraries}

\noindent
There is a substantial collection of widely used libraries for Java.
Here are reference books for the Java libraries we use in this book.
%
\begin{itemize}
%
\item 
Loughran, Steve and Erik Hatcher. 2007
{\it Ant in Action}.  Manning.
\\
{\footnotesize This is technically the second edition of their 2002
book, {\it Java Development with Ant}.  Both authors are committers for
the Apache Ant project.}
%
\item
Tahchiev, Peter, Felipe Leme,
Vincent Massol, and Gary Gregory. 2010.
{\it JUnit in Action}, Second Edition.  Manning.
\\
{\footnotesize A good overview for JUnit.  Should be up-to-date with
the latest attribute-driven versions, unlike either Kent Beck's book
(he's the author of Ant).  As much as we liked {\it The Pragmatic
Programmer}, Hunt and Thomas's book on JUnit is not particularly useful.}
%
\item
McCandless, Michael, Erik Hatcher and Otis Gospodneti\'c.
2010.
{\it Lucene in Action}, Second Edition. Manning.
\\
{\footnotesize A fairly definitive overview of the Apaceh Lucene
search engine by three project committers.}
\end{itemize}


\mysubsection{Practice Makes Perfect}

\noindent
Really, the only way to learn how to code is to practice.  Being a
good coder means two things: being fast and efficient (assuming
accuracy).  By accurate, we mean writing code that does what it's
supposed to do.  By fast, we mean the coding goes quickly.  By efficient,
we mean the resulting code is efficient.

For practice, we highly recommend the algorithm section of the
TopCoder contests.  TopCoder provides a complete online Java
environment, well thought out problems at varying specified levels of
difficulty, unit tests that check your submissions, and examples of
what others did to provide code-reading practice.
%
\begin{quote}
\hrefurl{http://www.topcoder.com/tc}
\end{quote}
%



\mysection{General Programming}

\noindent
We would also recommend a book about programming in general, 
especially for those who have not worked collaboratively with
a good group of professional programmers.
%
\begin{itemize}
\item Hunt, Andrew and David Thomas.
1999.
{\it The Pragmatic Programmer}.
Addison-Wesley.
\\
{\footnotesize Read this book and follow its advice, which is just as relevant now
as when it was written.  The best book on how to program
we know.  A competitor to books like Steve McConnell's {\it Code Complete} (2004)
and books on ``agile'' or ``extreme'' programming.}
\end{itemize}


\mysection{Text Encoding}

\noindent
Unicode is so important it deserves its own section.  There's one definitive
book covering unicode, and it's
%
\begin{itemize}
\item Unicode Consortium, The.  2006.
{\it Unicode Standard, Version 5.0}.
Addison-Wesley.
\end{itemize}
%
One of the best parts of this book is that it's available online fromthe
Unicode Consortium, at
%
\begin{quote}
\hrefurl{http://unicode.org/}
\end{quote}
%
This includes all of the code tables for the latest version (5.2), with
an online search by character code, at
%
\begin{quote}
\hrefurl{http://unicode.org/charts/}
\end{quote}


\mysection{Algorithms}

\noindent
Because natural language processing requires comptuational
implementations of its models, a good knowledge of algorithm analysis
goes a long way in understanding the approaches found in a tool kit
like LingPipe.

\begin{itemize}
%
\item
Cormen, Thomas H., Charles E.\ Leiserson, Ronald L.\ Rivest, and
Clifford Stein. 
2009.
{\it Introduciton to Algorithms}, Third Edition.
MIT Press.
\\
{\footnotesize A thorough yet readable introduction to algorithms.}
%
\item
Durbin, Richard, Sean R.\ Eddy, Anders Krogh, Grame Mitchison.
{\it Biological Sequence Analysis: Probabilistic Models of Proteins
and Nucleic Acids}.
Cambridge University Press.
\\
{\footnotesize Even though it focuses on biology, it has great
descriptions of several key natural language processing algorithms
such as hidden Markov models and probabilistic context-free grammars.}
%
\item
Gusfield, Dan.
1997.
{\it Algorithms on Strings, Trees and Sequences}.
Cambridge University Press.
\\
{\footnotesize Incredibly detailed and thorough introduction to
string algorithms, mainly aimed at computational biology, but
useful for all string and tree processing.}
%
\end{itemize}






\mysection{Probability and Statistics}

\noindent
Most of the tools in LingPipe are statistical in nature.  To fully
understand modern natural language processing it's necessary to
understand the statistical models which make up their foundation.
To do this requires an understanding of the basics of statistics.
Much of the work in natural language processing is Bayesian, so
understanding Bayesian statistics is now a must for understanding
the field.  Much of it also relies on information theory, which is
also introduced in the machine learning textbooks.

\begin{itemize}
%
\item
Bulmer, M.~G.
1979.
{\it Principles of Statistics}.
Dover.
\\
{\footnotesize Packs an amazing amount of information into
a readable little volume, though it assumes some mathematical
sophistication.}
%
\item Cover, Thomas M.\ and Joy A.\ Thomas.
2006.
{\it Elements of Information Theory}, Second Edition.
Wiley.
\\
{\footnotesize This is the classic textbook for information theory.}
%
\item
DeGroot, Morris H.\ and Mark J.\ Schervish.
2002.
{\it Probability and Statistics},  Third Edition.
Addison-Wesley.
\\
{\footnotesize Another introduction to classical frequentist
statistical hypothesis testing and estimation but with some Bayesian
concepts thrown into the mix.}
%
\item
Gelman, Andrew and Jennifer Hill.
2006.
{\it Data Analysis Using Regression and Multilevel/Hierarchical Models}.
Cambridge University Press.
\\
{\footnotesize A great practical introduction to regression from a Bayesian
perspective with an emphasis on multilevel models.  Much less mathematically
demanding than the other introductions, but still very rigorous.}
%
\item
Gelman, Andrew, John B.~Carlin, Hal S.\ Stern, and Donald B.\ Rubin.
2003.
{\it Bayesian Data Analysis}, Second Edition.
Chapman and Hall.
\\
{\footnotesize The bible for Bayesian methods.  Presupposes high
mathematical sophistication and a background in classical probability
and statistics to the level of DeGroot and Schervish or Larsen
and Marx's books.}
%
\item
Larsen, Richard J.\ and Morris L.\ Marx.
2005.
{\it Introduction to Mathematical Statistics and Its Applications}, Fourth Edition.
Prentice-Hall.
\\
{\footnotesize A nice intro to classical frequentist statistics.}
%
\end{itemize}


\mysection{Machine Learning}

\noindent
Machine learning, in contrast to statistics, focuses on implementation
details and tends to concentrate on predictive inference for larger
scale applications.  As such, understanding machine learning requires
a background in both statistics and algorithms.

\begin{itemize}
\item
Bishop, Christopher M.
2007.
{\it Pattern Recognition and Machine Learning}.
Springer.
\\
{\footnotesize Introduction to general machine learning
techniques that is self contained, but assumes more mathematical
sophistication than this book.}
%
\item
Hastie, Trevor, Robert Tibshirani and Jerome Friedman.
2009.
{\it The Elements of Statistical Learning}, Second Edition.
Springer.
\\
{\footnotesize Presupposes a great deal of mathematics, providing
fairly rigorous introductions to a wide range of algorithms and
statistical models.}
%
\item
MacKay, David J.~C.
2002.
{\it Information Theory, Inference, and Learning Algorithms}.
Cambridge University Press.
\\
{\footnotesize More of a set of insightful case studies than a coherent textbook.
Presupposes a high degree of mathematical sophistication.}
%
\item 
Witten, Ian H.\ and Eibe Frank. 2005. {\it Data Mining: Practical
Machine Learning Tools and Techniques} (Second Edition).  Elsevier.
\\
{\footnotesize The most similar to this book, explaining general machine
learning techniques at a high level and 
linking to the Java implementations (from their widely used Weka toolkit).}
%
\end{itemize}




\mysection{Linguistics}

\noindent
Natural language processing is about language.  To fully appreciate
the models in LingPipe, it is necessary to have some background in
linguistics.  Unfortunately, most of the textbooks out there are based
on traditional Chomskyan theoretical linguistics which divorces
language from its processing, and from which natural language
processing diverged several decades ago.

\begin{itemize}
%
\item Bergmann, Anouschka, Kathleen Currie Hall, and Sharon Miriam Ross (editors).
2007.
{\it Language Files}, Tenth Edition.
Ohio State University Press.
\\
{\footnotesize Easy to follow and engaging overview of just about every
aspect of human language and linguistics.  Books on specific areas of
linguistics tend to be more theory specific.}
%
\end{itemize}


\mysection{Natural Language Processing}

\noindent
Natural language processing is largely about designing and coding computer
programs that operate over natural language data.  Because most NLP is
statistical these days, this requires not only a knowledge of linguistics,
but also of algorithms and statistics.

\begin{itemize}
\item
Jurafsky, Daniel, James H.\ Martin.
2008.
{\it Speech and Language Processing}, Second Edition.
Prentice-Hall.
\\
{\footnotesize High-level overview of both speech and language
processing.}
%
\item
Manning, Christopher D., Raghavan Prabhakar, and Hinrich Sch\"utze.
2008.
{\it Introduction to Information Retrieval}.
Cambridge University Press.
\\
{\footnotesize Good intro to IR and better intro to some NLP than
Manning and Sch\"utze's previous book, which is starting to feel
dated.}
%
\item
Manning, Christopher D.\ and Hinrich Sch\"utze.
1999.
{\it Foundations of Statistical Natural Language Processing}.
MIT Press.
\\
{\footnotesize Needs a rewrite to bring it up to date, but still
worth reading given the lack of alternatives.}
\end{itemize}





