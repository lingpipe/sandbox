\chapter{Mathematics}\label{chapter:maths}

This appendix provides a refresher in some of the mathematics
underlying machine learning and natural language processing.

\section{Basic Notation}

\subsection{Summation}

We use summation notation throughout, writing $\sum_{n \in N}
x_n$ for the sum of the $x_n$ values for all $n \in N$.  For instance,
if $N = \setext{1,5,7}$, then $\sum_{n \in N} x_n = x_1 + x_5 + x_7$.

We also write $\sum_{n=a}^b x_n$ for the sum $x_a + x_{a+1}, \ldots,
x_{b-1}, x_{b}$.  If $a = -\infty$ or $b = \infty$, the sums are open
ended.  For instance, we write $\sum_{n=a}^{\infty} x_n$ for the
infinite sum $x_a + x_{a+1} + x_{a+2} + \cdots$.

The boundary condition of $\sum_{n \in N} x_n$ when $N$ is the empty
set is 0, because 0 is the additive unit \ie{$0+n = n+0 = n$}.


\subsection{Multiplication}

We use product notation just like summation notation, swapping the
product symbol $\prod$ for the summation symbol $\sum$.  For instance,
we write $\prod_{n \in N} x_n$ for the product of all the $x_n$ values
for $n \in N$.  We use bounded and infinite products in exactly the
same way as for summation.

The boundary condition for $\prod_{n \in N} x_n$ in the case where $N$
is empty is 1, because 1 is the multiplicative unit \ie{$1 \times n =
  n \times 1 = n$}.




\section{Useful Functions}


\subsection{Exponents}

The exponential function $\exp(x)$ is defined as the unique non-zero
solution to the differential equation $f' = f$.  Thus we have
%
\begin{equation}
\frac{d}{dx}\exp(x) = \exp(x)
\ \ \ \ \ \mbox{and} \ \ \ \ \ 
\int \exp(x) \ dx = \exp(x).
\end{equation}
%
It turns out the solution is $\exp(x) = e^x$, where $e$, known as
Euler's number, is approximately 2.718.

The exponential function may be applied to any real number, but its
value is always positive.  The exponential function has some
convenient properties, such as
%
\begin{equation}
\exp(a + b) = \exp(a) \times \exp(b), \mbox{ and}
\end{equation}
%
\begin{equation}
\exp(a \times b) = \exp(a)^b.
\end{equation}


\subsection{Logarithms}

The natural logarithm, $\log x$ is defined as the inverse of
$\exp(x)$.  That is, $\log x$ is defined to be the $y$ such that
$\exp(y) = x$.  Thus for all $x \in (-\infty,\infty)$ and all $y \in
(0,\infty)$ we have
%
\begin{equation}
\log \exp(x) = x
\ \ \ \ \ \mbox{and} \ \ \ \ \ 
\exp(\log y) = y.
\end{equation}
%

Logarithms convert multiplication into addition and exponentiation
into multiplication, with
%
\begin{equation}
\log (x \times y) = \log x + \log y, \mbox{ and}
\end{equation}
%
\begin{equation}
\log x^y = y \log x.
\end{equation}

We can define logarithms in different bases, where $\log_b x$ is defined
to be the $y$ such that $b^y = x$.  In terms of natural logs, we have
%
\begin{equation}
\log_b x = \log x / \log b.
\end{equation}

\subsection{The Factorial Function}\label{section:stats-factorial}

The factorial function computes the number of different ordered
permutations there are of a list of a given (non-negative) number of
distinct objects.  The factorial function is written $n!$
and defined for $n \in \nats$ by
%
\begin{equation}
n! = 1 \times 2 \times \cdots \times n = \prod_{m=1}^n m.
\end{equation}
%
The convention is to set $0! = 1$ so that the boundary conditions of
definitions like the one we provided for the binomial coefficient work
out properly.


\subsection{Binomial Coefficients}\label{section:stats-binomial-coefficient}

The binomial coefficient has many uses.  We're mainly interested in
combinatorics, so we think of it as providing the number of ways to
select a subset of a given size from a superset of a given size.  The
binomial coefficient is written ${m choose n}$, prounced ``$m$
choose $n$,'' and defined for $m, n \in \nats$ with $n \geq m$ by
%
\begin{equation}
{n \choose m} = \frac{n!}{m! \ (n-m)!}.
\end{equation}


\subsection{Multinomial Coefficients}\label{section:maths-multinomial-coefficient}

The multinomial coefficient generalizes the binomial coefficient to
choosing more than one item.  It tells us how many ways there are to
partition a set into subsets of fixed sizes.  For a set of size $n$,
the number of ways partition it into subsets of sizes
$m_1,\ldots,m_K$, where $m_k \in \nats$ and $n = \sum_{k=1}^K m_k$, is
%
\begin{equation}
{n \choose {m_1, \ldots, m_K}}
= \frac{n!}{\prod_{k=1}^K m_k!}.
\end{equation}


\subsection{The $\Gamma$ Function}\label{section:stats-gamma-function}

The $\Gamma$ function is the complex generalization of the factorial
function (see \refsec{stats-factorial}).  It is written $\Gamma(x)$, 
and we're only concerned with real $x \in [0,\infty)$, where its value
is defined by 
%
\begin{equation}
\Gamma(x) = \int_0^{\infty} t^{x-1} \ \exp(-t) \ dt.
\end{equation}
%
In general, we have
%
\begin{equation}
\Gamma(x+1) = x \ \Gamma(x).
\end{equation}
%
Because we have $\Gamma(1) = 1$, we also have for all $n \in \nats, n > 0$, 
%
\begin{equation}
\Gamma(n) = (n-1)!.
\end{equation}

