\chapter{Suffix Arrays}\label{chap:suffix-arrays}

Suffix arrays provide a convenient data structure for finding repeated
substrings in text.  Suffix arrays efficiently support operations such
as finding all of the repeated substrings in a larger string.
LingPipe provides suffix arrays that operate at either the character
level or the token level.

\section{What is a Suffix Array?}

A simple example will suffice to illustrate the basic workings of
suffix arrays.  Suppose we have the string \stringmention{abracadabra},
which has character indexes as follows.
%
\begin{verbatim}
      abracadabra
      012345678901
      0         1
\end{verbatim}
%

A suffix array represents the suffixes of a string.  For instance, the
string \stringmention{abracadabra} has the following suffixes, given
with their starting position in the string.
%
\begin{center}
\begin{tabular}{rl}
\tblhead{Pos} & \tblhead{Suffix}
\\ \hline
0 & \stringmention{abracadabra} \\
1 & \stringmention{bracadabra} \\
2 & \stringmention{racadabra} \\
3 & \stringmention{acadabra} \\
\end{tabular}
%
\hspace*{0.5in}
%
\begin{tabular}{rl}
\tblhead{Pos} & \tblhead{Suffix}
\\ \hline
4 & \stringmention{cadabra} \\
5 & \stringmention{adabra} \\
6 & \stringmention{dabra} \\
7 & \stringmention{abra} \\
\end{tabular}
%
\hspace*{0.5in}
%
\begin{tabular}{rl}
\tblhead{Pos} & \tblhead{Suffix}
\\ \hline
8 & \stringmention{bra} \\
9 & \stringmention{ra} \\
10 & \stringmention{a} \\
{} & {}
\end{tabular}
\end{center}
%

A suffix array sorts the character indexes based on the strings
corresponding to the suffixes.  The suffix array for our running
example is as follows, listing the suffix array index, the position
in the undelrying string, and the suffix it represents.
%
\begin{center}
\begin{tabular}{rrl}
\tblhead{Idx} & \tblhead{Pos} & \tblhead{Suffix}
\\ \hline
0 & 10 & \stringmention{a} \\
1 & 7 & \stringmention{abra} \\
2 & 0 & \stringmention{abracadabra} \\
3 & 3 & \stringmention{acadabra} \\
4 & 5 & \stringmention{adabra} \\
5 & 8 & \stringmention{bra} \\
\end{tabular}
%
\hspace*{0.5in}
%
\begin{tabular}{rrl}
\tblhead{Idx} & \tblhead{Pos} & \tblhead{Suffix}
\\ \hline
6 & 1 & \stringmention{bracadabra} \\
7 & 4 & \stringmention{cadabra} \\
8 & 6 & \stringmention{dabra} \\
9 & 9 & \stringmention{ra} \\
10 & 2 & \stringmention{racadabra} \\
{} & {}
\end{tabular}
\end{center}
%

The suffix array itself is just the array of first-character
positions for the suffixes.  In our example, the suffix array is
the following array of \code{int} values.
%
\begin{verbatim}
{ 10, 7, 0, 3, 5, 8, 1, 4, 6, 9 , 2 }
\end{verbatim}
%
The indexes are implicit, as is usual for an array, and
the suffix itself may be retrieved given the underlying
string and the position of the first character. 

The utility of a suffix array is that it brings together suffixes that
share common prefixes.  For instance, it is easy to see that there are
two suffixes that start with the sequence \stringmention{bra}, one
starting at position 8 and one starting at position 1.  We can also see
that there are five substrings starting with \stringmention{a},
two starting with \stringmention{abra}.  

Every substring of our original string is representable as a prefix
of a suffix.  for instance, the substring \stringmention{ca} spanning
from position 7 (inclusive) to 9 (exclusive) is the two-character
prefix of the suffix \stringmention{cadabra} starting at position 7.

Thus we can easily find all the matching substrings in a string.  For
instance, if we want to find all the instances of strings of length
three or more that occur two or more times in the text, we just scan
down the suffix array, where we will find \stringmention{abr} and
\stringmention{bra} both occur twice.




\section{Character Suffix Arrays}

LingPipe implements suffix arrays where the symbols are characters in
the class \code{CharSuffixArray} in the package
\code{com.aliasi.suffixarray}.  We cover token suffix arrays in the
next section).

\subsection{Demo Code Walkthrough}

We have provided a demo in class \code{CharSuffixArrayDemo} in
this chapter's package, \code{com.lingpipe.book.suffixarray}.  
The demo class does little more than build a suffix array and then
walk over it extracting the suffixes in order.  The relevant code,
after assigning the input argument to a string \code{text} is
%
\codeblock{CharSuffixArrayDemo.1}
%
The code uses the constructor without a specification on the maximum
length.  Then it walks through all of the positions, up to the length
of the suffix array.  From each, it extracts the position in the
underlying string and the suffix running from that position to the end
of the string.  We have elided the print statements, as usual.

The second part of the demo extracts all of the repeated substrings.
As such, it starts with a loop over the length of matching substrings,
then extracts them inside.
%
\codeblock{CharSuffixArrayDemo.2}
%
We start from the maximum possible match, which is one less
than the length of the string, and repeat for all lengths above 1.
Within the loop, we call the \code{prefixMatches()} method on the
suffix array with the specified length.  This returns a list of
spans represented as integer arrays.  If the size is zero, there
are no repeated substrings of the specified length, and we continue
to the next iteration of the loop.  

If there are matches, we iterate over the arrays of positions.
For each such position, we extract the start and end position,
assigning them to local variables for readability.  Finally, we
loop over the matching positions from first (inclusive) to last
(exclusive), and for each match, pull out the position, then generate
the string corresponding to the match.  We do not show the print
routines for these values, but show examples in the next section.


\subsection{Running the Demo}

We've provided an Ant target \code{char-demo} to run the demo.  
The property \code{text} specifies the text for which to build
a suffix array.  Using our running example, we have
%
\commandlinefollow{ant -Dtext=abracadabra char-demo}
\begin{verbatim}
abracadabra
012345678901
0         1

Idx  Pos  Suffix           Idx  Pos  Suffix
  0   10  a                  6    1  bracadabra
  1    7  abra               7    4  cadabra
  2    0  abracadabra        8    6  dabra
  3    3  acadabra           9    9  ra
  4    5  adabra            10    2  racadabra
  5    8  bra
\end{verbatim}
%
As before, the index is the (implicit) index into the
suffix array, the position indicates the position in the
original string, and the suffix is the range of characters
from the position to the end of the original string.

After writing out the basic suffix array, the demo carries on with
writing out the positiosn of all repeated substrings of length greater
than one.
%
\begin{verbatim}
start=  7 len=  4  abra       start=  7 len=  2  ab
start=  0 len=  4  abra       start=  0 len=  2  ab

start=  7 len=  3  abr        start=  8 len=  2  br
start=  0 len=  3  abr        start=  1 len=  2  br

start=  8 len=  3  bra        start=  9 len=  2  ra
start=  1 len=  3  bra        start=  2 len=  2  ra
\end{verbatim}
%
For instance, this indicates that the substring \stringmention{abra}
appeared twice, once starting at position 7 and once starting at
position 0.  

Although there are only pairs of matches here, the number can
quickly grow with repetitive strings.  The worst case situation for
overlaps is with a sequence of the same character. 
%
\commandlinefollow{ant -Dtext=yyyyy char-demo}
\begin{verbatim}
yyyyy
012345

Idx   Pos  Suffix      Idx   Pos  Suffix
  0     4  y             3     1  yyyy
  1     3  yy            4     0  yyyyy
  2     2  yyy


start=  1 len=  4  yyyy       start=  3 len=  2  yy
start=  0 len=  4  yyyy       start=  2 len=  2  yy
                              start=  1 len=  2  yy
start=  2 len=  3  yyy        start=  0 len=  2  yy
start=  1 len=  3  yyy
start=  0 len=  3  yyy
\end{verbatim}
%
In this example, the suffixes are merely sorted by length, and
there are two matches of length four, three of length three, and
four of length two.









\section{Implementation Details} 

Suffix arrays are constructed in the obvious way.  An array of all
integer positions is created and then sorted based on the suffixes.

The initial implementation in LingPipe 4 is based on \code{Integer}
(object) indexes and a simple \code{Comparator} implementation for
sorting.


Thus the overall construction time is no worse than $\bigO{n \log n}$
comparisons, where $n$ is the length of the string.  Although the
comparisons appear to require another $\bigO{n}$ to carry out, the
operations may be amortized in the following way.  When we are doing
comparisons for sorting, the divide-and-conquer nature of sorting
algorithms allows us to deduce the entire array we're going to sort
shares the same initial characters.  These prefix comparisons can then
be skipped in subsequent calculations for the subblock of data with
shared prefixes.











