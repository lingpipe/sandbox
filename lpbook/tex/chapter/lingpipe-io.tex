\section{Lingpipe I/O Utilities}

LingPipe provides a number of utilities for input and output,
and we discuss the general-purpose ones in this section.  

\subsection{The \code{FileLineReader} Class}\label{section:io-file-line-reader}

To deal with line-based file formats, LingPipe provides the class
\code{FileLineReader}, in the \code{com.aliasi.io} package.  This
class extends \code{LineNumberReader} in \code{java.io}, which itself
extends \code{BufferedReader}, so all the inherited methods are
available.  

For convenience, the \code{FileLineReader} class implements the
\code{Iterable<String>} interface, with \code{iterator()} returning an
\code{Iterator<String>} that iterates over the lines produced by the
\code{readLine()} method inherited from \code{BufferedReader}.  Being
iterable enables for-loop syntax, so that if \code{file} is an
instance of \code{File}, the standard idiom for reading lines is
%
\codeblock{FragmentsIo.14}
%
The elided code block will then perform some operation on each line.

There are also static utility methods, \code{readLines()} and
\code{readLinesArray()}, both taking a file and character encoding
name like the constructor, to read all of the lines of a file into
either an array of type \code{String[]} or a list of type
\code{List<String>}.

\subsection{The \code{Files} Utility Class}

Some standard file operations are implemented as static methods
in the \code{Files} class from the package \code{com.aliasi.util}.

\subsubsection{File Name Parsing}

The \code{baseName(File)} and \code{extension(File)} methods split a
file's name into the part before the last period (\code{.}) and the
part after the last period.  If there is no last period, the base name
is the entire file name.

\subsubsection{File Manipulation}

The method \code{removeDescendants(File)} removes all the files
contained in the specified directory recursively.  The
\code{removeRecursive(File)} method removes the descendants of a file
and the file itself.  The \code{copy(File,File)} method
copies the content of one file into another.

\subsubsection{Bulk Reads and Writes}

There are methods to read bytes, characters, or a string from a file,
and corresponding methods to write.  The methods all pass on any I/O
exceptions they encounter.

The method \code{readBytesFromFile(File)} returns the array of bytes
corresponding to the file's contents, whereas
\code{writeBytesToFile(byte[],File)} writes the specified bytes to the
specified file.

The methods \code{readCharsFromFile(File,String)} and
\code{writeCharsToFile(char[],File,String)} do the same thing for
\code{char} values.  These methods both require the character encoding
to be specified as the final argument.  There are corresponding
methods for strings, \code{readFromFile(File,String)} and
\code{writeStringToFile(String,File,String)}.  Together, we can
use these methods to transcode from one encoding to another,
%
\codeblock{FragmentsIo.15}


\subsection{The \code{Streams} Utility Class}

The class \code{Streams} in the package \code{com.aliasi.io} provides
static utility methods for dealing with byte and character streams.  

\subsubsection{Quiet Close Methods}

The method \code{closeQuietly(Closeable)} closes the specified
closeable object, catching and ignoring any I/O exceptions raised
by the close operation.  If the specified object is null, the
quiet close method returns.

The method \code{closeInputSource(InputSource)} closes the byte
and character streams of the specified input source quietly, as
if by the \code{closeQuietly()} method.

\subsubsection{Bulk Reads}

There are utility bulk read methods that buffer inputs and return them
as an array.  Special utility methods are needed because the amount of
data available or read from a stream is unpredictable.  Bulk writing,
on the other hand, is built into output streams and writers
themselves.

The bulk read method \code{toByteArray(InputStream)} reads all of the
bytes from the input stream returning them as an array.  The method
\code{toCharArray(Reader)} does the same for \code{char} values.

The method \code{toCharArray(InputSource)} reads from an input source,
using the reader if available, and if not, the input stream using
either the specified character encoding or the platform default, and
barring that, reads from the input source's system identifier (URL).

The method \code{toCharArray(InputStream,String)} reads an array of
\code{char} values by reading bytes from the specified input stream
and converts them to \code{char} values using the specified character
encoding. 

\subsubsection{Copying Streams}

The method \code{copy(InputStream,OutputStream)} reads bytes from the
specified input stream until the end of stream, writing what it finds
to the output stream.  The \code{copy(Reader,Writer)} does the same
thing for \code{char} value streams.

\subsection{The \code{Compilable} Interface}\label{section:io-compilable}

LingPipe defines the interface \code{Compilable} in the package
\code{com.aliasi.util} for classes that define a compiled
representation.  

The \code{Compilable} interface has a single method,
\code{compileTo(ObjectOutput)}, used for compiling an object to an
object output stream.  Like all I/O interface methods, it's defined to
throw an \code{IOException}.  The method is used just like the
\code{writeExternal(ObjectOutput)} method from Java's
\code{Externalizable} interface.  It is assumed that only whatever is
written, deserialization will produce a single object that is the
compiled form of the object that was compiled.

Many of LingPipe's statistical models, such as language models,
classifiers and HMMs have implementations that allow training data to
be added incrementally.  The classes implementing these models are
often declared to be both \code{Serializable} and \code{Compilable}.
If they are serializable, the standard serialization followed by
deserialization typically produces an object in the same state and
of the same class as the one serialized.  If a model's compilable,
a compilation followed by deserialization typically produces an
object implementing the same model more efficiently, but without the
ability to be updated with new training data. 


\subsection{The \code{AbstractExternalizable} Class}\label{section:io-abstract-externalizable}

LingPipe provides a dual-purpose class \code{AbstractExternalizable}
in the package \code{com.aliasi.util}.  This class provides static
utility methods for dealing with serializable and compilable objects,
as well as supporting the serialization proxy pattern.

\subsubsection{Base Class for Serialization Proxies}

The abstract class \code{AbstractExternalizable} in package
\code{com.aliasi.util} provides a base class for implementing the
serialization proxy pattern, as described in
\refsec{io-serialization-proxy}.  It is declared to implement
\code{Externalizable}, and the \code{writeExternal(ObjectOutput)}
method from that interface is declared to be abstract and thus must be
implemented by a concrete subclass.  The method
\code{readExternal(ObjectInput)} is implemented by
\code{AbstractExternalizable}, as is the serialization method
\code{readResolve()}.  The other abstract method that must be
implemented is \code{read(ObjectInput)}, which returns an object.
This method is called by \code{readExternal()} and the object it
returns stored to be used as the return value for
\code{readResolve()}.

To see how to use this class as the base class for serialization
proxies, we provide an example that parallels the one we implemented
directly in \refsec{io-serialization-proxy}.  This time, we use
a nested inner class that extends \code{AbstractExternalizable}.

The class is constructed as follows.
%
\codeblock{NamedScoredList.1}
%
Note that the class is defined with a generic and declared to
implement Java's \code{Serializable} interface.  Also note the
explicit declaration of the serial version UID.  The constructor
stores a copy of the input list, the string name, and the double
score, as well as a new object which is not serializable itself.

The two features making this class immutable also prevent it from
being serializable.  First, there is no zero-argument constructor.
Second, the member variables are declared to be final.  Adding a
public zero-arg constructor and removing the final declarations would
allow the class to be serialized in place.

Instead, we use the \code{writeReplace()} method as before to
return the object's serialization proxy, which is also declared
to be generic and provided with the object itself through \code{this}.
%
\codeblock{NamedScoredList.2}

The class \code{Externalizer} is declared as a nested static class
as follows.
%
\codeblock{NamedScoredList.3}
%
Note that the generic argument is different; they can't be the same
because this is a nested static class, not an inner class, so it
doesn't have access to the containing class's generic.  It is defined
to extend the utility class \code{AbstractExternalizable}.  It also
defines its own serial version UID.  It holds a final reference to a
\code{NamedScoredList} for which it is the serialization proxy.  There
is the requisite public no-argument constructor, as well as a
constructor that sets the member variable to its argument.  The no-arg
constructor is used for deserialization and the one-arg constructor
for serialization.

The helper class requires two abstract methods to be overridden.
The first is for writing the object, and gets called as soon as
\code{writeReplace()} creates the proxy.
%
\codeblock{NamedScoredList.4}
%
This uses the contained instance \code{mAe} to access the data that
needs to be written.  Here, it's an object, a string, and a double,
each with their own write methods.  The class is defined to throw
an I/O exception, and will almost always require this declaration
because the write methods throw I/O exceptions.

The second overridden method is for reading.  This just reverses
the writing process, returning an object.
%
\codeblock{NamedScoredList.5}
%
We read in the list, using the generic on the class itself
for casting purposes.  We then read in the string and double
with the matching read methods on object inputs.  Finally, we
return a new named scored list with the same generic.  This
object is what's finally returned on deserialization.

The \code{main()} method for the demo just creates an object
and then serializes and deserializes it using the utility method.
%
\codeblock{NamedScoredList.6}

This may be run using the Ant target \code{ae-demo}.
%
\commandlinefollow{ant ae-demo}
\begin{verbatim}
p1=NSL([a, b, c], foo, 1.4)     p2=NSL([a, b, c], foo, 1.4)
\end{verbatim}

\subsubsection{Serializing Singletons}

This class makes it particularly easy to implement serialization for a
singleton.  We provide an example in \code{SerializedSingleton}.  Following
the usual singleton pattern, we have a private constructor and a public
static constant,
%
\codeblock{SerializedSingleton.1}
%
We then have the usual definition of \code{writeReplace()}, and a
particularly simple implementation of the serialization proxy itself,
%
\codeblock{SerializedSingleton.2}
%
If the nested static class had been declared public, we wouldn't have
needed to define the public no-argument constructor.  The version ID
is optional, but highly recommended for backward compatibility.

To test that it works, we have a simple \code{main()} method,
%
\codeblock{SerializedSingleton.3}
%
which serializes and deserializes the instance using the 
LingPipe utility in \code{AbstractExternalizable}, then tests
for reference equality.

We can run this code using the Ant target \code{serialized-singleton},
which takes no arguments,
%
\commandlinefollow{ant serialized-singleton}
\begin{verbatim}
same=true
\end{verbatim}

\subsubsection{Serialization Proxy Utilities}

The \code{AbstractExternalizable} class defines a number of static
utility methods supporting serialization of arrays.  These methods
write the number of members then each member of the array in turn.
For instance, \code{writeInts(int[],ObjectOutput)} writes an array of
integers to an object output and \code{readInts(ObjectInput)} reads
and returns an integer array that was serialized using the
\code{writeInts()} method.  There are similar methods for \code{double},
\code{float}, and \code{String}.

The method \code{compileOrSerialize(Object,ObjectOutput)} will attempt
to compile the specified object if it is compilable, otherwise it
will try to serialize it to the specified object output.  It will throw
a \code{NotSerializableException} if the object is neither serializable
nor compilable.  The method \code{serializeOrCompile(Object,ObjectOutput)}
tries the operations in the opposite order, compiling only if the
object is nor serializable.

\subsubsection{General Serialization Utilities}

The \code{AbstractExternalizable} class defines a number of
general-purpose methods for supporting Java's serialization and 
LingPipe's compilation.

There are in-memory versions of compilation and serialization.  The
method \code{compile(Compilable)} returns the result of compiling then
deserializing an object. The method
\code{serializeDeserialize(Serializable)} serializes and deserializes
an object in memory.  These methods are useful for testing
serialization and compilation.  These operations are carried out in
memory, which requires enough capacity to store the original object, a
byte array holding the serialized or compiled bytes, and the
deserialized or compiled object itself.  The same operation may be
performed with a temporary file in much less space because the
original object may be garbage collected before deserialization and
the bytes reside out of main memory.

There are also utility methods for serializing or compiling to a file.
The method \code{serializeTo(Serializable,File)} writes
a serialized form of the object to the specified file.  The 
\code{compileTo(Compilable,File)} does the same thing for compilable
objects.

To read serialized or compiled objects back in, the method
\code{readObject(File)} reads a serialized object from a file and
returns it.

The method \code{readResourceObject(String)} reads an object from a
resource with the specified absolute path name (see
\refsec{io-resource-input} for information on resources including
how to create them and put them on the classpath).  This method simply
creates an input stream from the resource then wraps it in an object
input stream for reading.  The
\code{readResourceObject(Class<?>,String)} method reads a (possibly
relative) resource relative to the specified class.  


\subsection{Demos of Serialization and Compilation}

In the demo classes \code{CompilationDemo} and
\code{SerializationDemo}, we provide examples of how to serialize
and deserialize a class.  Both classes are based on a simple
\code{main()} command-line method and both use the same object
to serialize and compile.


\subsubsection{Object Construction}

Suppose we have an instance of a class that's compilable, like
\code{TrainSpellChecker} in LingPipe's \code{spell} package.  We will
assume it is constructed using an appropriate language model \code{lm}
and weighted edit distance \code{dist}, using the following code at
the beginning of the main method of both demo classes.
%
\codeblock{CompilationDemo.0}
%
This code also provides it with a training example so it's not completely
empty later (not that being empty would cause problems).

\subsubsection{Serialization Demo}

To serialize a file directly, we can use the following.
%
\codeblock{SerializationDemo.1}
%
We first create a file input stream, then wrap it in an object output
stream, in both cases assigning to more general variables.  We then
call \code{writeObject()} on the object output with the object to be
serialized as an argument.  This is how all serialization works in
Java at base.  If the object being passed in is not serializable, an
\code{NotSerializableException} will be raised.  We then close the
object output stream to ensure all bytes are written.  

The complementary set of calls reads the object back in.  We create an
input stream using the same file, wrap it in an object input, then
call the \code{readObject()} method on the object input.  Finally, we
cast to the appropriate class, \code{TrainSpellChecker}, and then
close the object input.

The same thing can be accomplished with the static utility methods
\code{serializeTo()} and \code{readObject()}, both in
\code{AbstractExternalizable}.
%
\codeblock{SerializationDemo.2}

It is also possible to serialize and deserialize in memory, which
is particularly useful for testing purposes.  The static utility
method \code{serializeDeserialize()} carries this out, as follows.
%
\codeblock{SerializationDemo.3}

We can run the demo using the Ant target \code{serialize-demo},
as follows.
%
\commandlinefollow{ant serialize-demo}
\begin{verbatim}
tsc.getClass()=class com.aliasi.spell.TrainSpellChecker
o1.getClass()=class com.aliasi.spell.TrainSpellChecker
o2.getClass()=class com.aliasi.spell.TrainSpellChecker
o3.getClass()=class com.aliasi.spell.TrainSpellChecker
\end{verbatim}
%
The output shows the result of calling the general object method
\code{getClass()} on the variables of type \code{Object} read back in.
This approach works in general for inspecting the class of a
deserialized object.


\subsubsection{Compilation Demo}

Assuming we have constructed a trained spell checker in variable
\code{tsc} as shown above, we can use the following code to compile
it.
%
\codeblock{CompilationDemo.1}
%
This involves creating an object output stream wrapping a file output
stream, then calling \code{compileTo()} on the object being compiled
with the object output as argument.  Make sure to close the stream so
that everything's written.  

After creating the object, we reverse the process by creating an
object input that wraps a file input stream.  We then use the
\code{readObject()} method on the object input to read an object,
which we assign to variable \code{o1} of type \code{Object}.  We then
cast the object to a compiled spell checker and make sure to close
the object input.

Exactly the same pair of operations can be carried out using the
static utility methods \code{compileTo()} and \code{readObject()}
with the following code.
%
\codeblock{CompilationDemo.2}
%
If you look at the implementation of these methods, they closely
match what was written above, with the addition of buffering and
more robust closes in finally blocks.

If you don't need an on-disk representation, the single method
\code{compile()} does the same thing in memory, though with more
robust closes.
%
\codeblock{CompilationDemo.3}
%

If we run the demo, using the \code{compile-demo} Ant target, we see
that the objects read back in are compiled spell checkers in each
case.
%
\commandlinefollow{ant compile-demo}
\begin{verbatim}
tsc.getClass()=class com.aliasi.spell.TrainSpellChecker
o1.getClass()=class com.aliasi.spell.CompiledSpellChecker
o2.getClass()=class com.aliasi.spell.CompiledSpellChecker
o3.getClass()=class com.aliasi.spell.CompiledSpellChecker
\end{verbatim}



