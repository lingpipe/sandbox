\mychapter{Introduction}{INTRODUCTION}

\firstchar{L}ingPipe is a library for natural language processing
implemented in Java.  This book explains the tools that are available
in LingPipe and provides examples of how they can be used to build
natural language processing (NLP) applications.

LingPipe's application programming interface (API) is tailored to
abstract over low-level implementation details to enable components
such as tokenizers, feature extractors, or classifiers to be swapped
in a plug-and-play fashion.  LingPipe contains a mixture of heuristic
rule-based components and statistical components, often implementing
the same interfaces, such as chunking or tokenization.

The presentation here will be hands on.  You should be comfortable
reading short and relatively simple Java programs.  Java programming idioms
like loop boundaries being inclusive/exclusive and higher-level design
patterns like visitors will also be presupposed.  More specific
aspects of Java coding relating to text processing, such as streaming
I/O, character decoding, string representations, and regular
expression processing will be discussed in more depth.  We will also
go into some detail on collections, XML/HTML parsing with SAX, and
serialization patterns.

We do not presuppose any knowledge of linguistics beyond a simple
understanding of the terms used in dictionaries such as words,
syllables, pronunciations, and parts of speech such as noun and
preposition.  We will spend considerable time introducing linguistic
concepts, such as word senses or noun phrase chunks, as they relate to
natural language processing modules in LingPipe.

We will do our best to introduce LingPipe's modules and their
application from a hands-on practical API perspective rather than a
theoretical one.  In most cases, such as for logistic regression
classifiers and conditional random field (CRF) taggers and chunkers,
it's possible learn how to effectively fit complex and useful models
without fully understanding the mathematical basis of LingPipe's
estimation and optimization algorithms.  In other cases, such as naive
Bayes classifiers, hierarchical clusterers and hidden Markov models
(HMM), the models are simpler, estimation is a matter of cuonting,
and there is almost no hand-tuning required.

Deeper understanding of LingPipe's algorithms and statistical models
requires familiarity with computational complexity analysis and basic
probability theory including information theory.   We provide
a sequence of suggested readings in 
\hyperref[appendix:reading]{Appendix~\ref*{appendix:reading}}.

After introducing LingPipe's modules and their applications at a
programming level, we will provide mathematically precise definitions.
The goal is a complete description of the mathematical models
underlying LingPipe.  To understand these sections will require a
stronger background in algorithms and statistical modeling than
the other sections of the book.
        


\mysection{The Java Programming Language}

\noindent
We chose Java as the basis for LingPipe because we felt it provided
the best tradeoff among efficiency, usability, portability, and
library availability.

The presentation here assumes the reader has a basic working knowledge
of the Java programming language.  We will focus on a few aspects of
Java that are particularly crucial for processing textual language
data, such as character and string representations, input and output
streams and character encodings, regular expressions, parsing HTML and
XML markup, etc.  In explaining LingPipe's design, we will also delve
into greater detail on general features of Java such as concurrency,
generics, floating point representations, and the collection package.

\mysubsection{Java Standard Edition 6}

\noindent
This book is based on the latest currently supported standard edition
of the Java platform (Java SE), which is version 6.  You will need the
Java development kit (JDK) in order to compile Java programs.  A java
virtual machine (JVM) is required to execute compiled Java programs.
A Java runtime environment (JRE) contains platform-specific support
and integration for a JVM and often interfaces to web browsers for
applet support.

Information on installing Java is available in Appendix~\ref{appendix:installation}.


\mysection{Shell and Apache Ant Build Tool}

\noindent
We primarily work in the shell with a text editor. Instructions for
installing shells for Windows are availabe in Appendix~\ref{appendix:installation}.

though it's
possible to do everything in an integrated development environment for
Java like Eclipse or NetBeans.

We present examples for compilation and for running programs using the
Apache Ant build tool.  This is an XML-based tool for scripting builds
and other operations that's simpler than dependency-based tools such
as the classic Unix make program or Apache's Maven.  



