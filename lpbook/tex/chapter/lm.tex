\chapter{Language Models}\label{chap:lm}

A \techdef{language model} is a model that assigns probabilities to
strings.  Traditionally, these strings are either sequences of bytes,
sequences of characters, or sequences of tokens.  LingPipe provides
language model implementations for sequences of characters and
sequences of tokens.  We are going to focus on character sequences
that make up natural language text, although language models may be
applied to arbitrary sequences.  This text may come from anywhere,
including e-mail or instant messages, blogs, newspapers, fiction
novels, emergency room discharge summaries, job applicant resume's,
etc.

The basic operation of language models in LingPipe is to provide them
training data consisting of a sequence of texts.  After they are
trained, language models may be used to estimate the probability of
texts that were not in the training data.  

\section{Applications of Language Models}

LingPipe uses language models as the basis of several other modules,
including classification, taggers, chunkers, spelling correction, and
significant phrase extractionnn for exploratory data analysis.  In all
cases, these applications use one or more language models to estimate
different kinds of text.  For instance, in language-model based
classification, there is a language model for each category, and texts
are classified into the category which assigns them the highest
probability.  In chunkers, separate language models might be
constructed for person names, company names, place names, and for text
that's not part of a name; these are then combined to analyze new text
for names of entities.  For spelling correction, a single language
model characterizes the correct spellings and is used to weigh
alternative corrections against each other.  For significant phrase
extraction, a background language model characterizes a fixed
comparison text, such as all the text collected from newspapers in a
year, and a foreground model characterizes some other text, such as
the text collected this week; differential analysis is then done to
find phrases that appear more frequently than one would expect in the
foreground model, thus modeling a kind of ``hot topic'' detection.

One of the major applications of language models is in speech
recognition, where they play the same role as the language model in
spelling correction.  

Another application of language models is compression.  Using a
technique such as arithmetic coding, language models provide
state-of-the-art compression.  Although LingPipe does not implement an
arithmetic coder for compression, we will discuss the connection to
compression below when we discuss empirical cross-entropy.

Yet another application of language models is in characterizing
languages or sublanguages.  For instance, we can compare the entropy
of news stories in the {\it New York Times} with biomedical abstracts
in MEDLINE, the surprising result being that MEDLINE is more
predictable at the text level than the {\it Times}).  We can also
compare the entropy of Chinese and English, measuring the average
information content of a Chinese character (of which there are tens of
thousands) with the information content of an English character (of
which there are dozens).

\section{Language Model Interfaces}

There is a rich set of interfaces for language models and related
support classes in the LingPipe package \code{com.aliasi.lm}.  The
basic interfaces are in the \code{LanguageModel} interface, most
being defined as nested interfaces.

\subsection{Predictive: \code{LanguageModel}}

The top-level language model interface, \code{LanguageModel}, defines
methods for estimating probabilities of sequence of characters.  It
defines two methods for estimating the log (base 2) probability of a
sequence of characters, one based on arrays and one based on the
character sequence interface.
%
\begin{verbatim}
double log2Estimate(char[] cs, int start, int end);
double log2Estimate(CharSequence cs);
\end{verbatim}
%
The first method uses an indexed character array to define the slice
of characters from \code{cs[start]} to \code{cs[end-1]}.  For most
methods, the character slice method is more efficient and the
character-sequence method defined by first converting the sequence to
an array of characters.

The reason estimates are returned as logarithms is that we would
otherwise run into underflow.  If each character has an average
probabilty of 0.25 (roughly what they have in open-domain English text
like a newspaper), the probability of a sequence of 600 tokens has a
$2^{-1200}$ probabilty, which is too small to fit into a Java
double-precision floating point value (the minimum value for which is
$2^{-1074}$).

\subsection{Trainable: \code{LanguageModel.Dynamic}}

Language models that may be trained by supplying example text
implement the interface \code{LanguageModel.Dynamic}.  This interface
extends the corpus interface \code{ObjectHandler<CharSequence>},
which specifies the method
%
\begin{verbatim}
void handle(CharSequence cs);
\end{verbatim}
%
The character sequence is considered as training data and used to fit
the model parameters.  See \refsec{corpus-handlers} for more
information on \code{ObjectHandler} and how it fits into corpus
patterns.

There are four other methods named \code{train()}, the first two
of which simply allow training with character sequences.
%
\begin{verbatim}
void train(CharSequence cs);
void train(CharSequence cs, int count);
\end{verbatim}
%
The first method is a legacy method duplicating the functionality of
\code{handle()}.  The second method takes an integer count value which
determines the number of times the character sequence is used for
training.  Although implemented more efficiently, a training call
with a count is equivalent to calling the basic training method the
count number of times.

The last two methods just repeat the character sequence methods
with character slices.


\subsection{Conditionals: \code{LanguageModel.Conditional}}

Conditional language models predict the probability of a character
given the previous characters.  The interface
\code{LanguageModel.Conditional} specifies the method
%
\begin{verbatim}
double log2ConditionalEstimate(CharSequence);
\end{verbatim}
%
This method returns the probabilty of the last character in
the sequence given the initial characters in the sequence.  For
instance, \code{log2ConditionalEstimate("abcd")} returns the
probability of the character \charmention{d} following the
sequence of characters stringmention{abc}.  There is also
a method working on character slices.

\subsection{Tokens: \code{LanguageModel.Tokenized}}

For language models based on tokens, the interface
\code{LanguageModel.Tokenized} supplies a method for
computing the probability of a sequence of tokens,
%
\begin{verbatim}
double tokenLog2Probability(String[] toks, int start, int end);
\end{verbatim}
%
This returns the log (base 2) probability of the specified slice of
tokens.  As usual, the start is inclusive and the end exclusive, so
that the tokens used are \code{toks[start]} to \code{toks[end-1]}.
There is also a method to return the linear probability, which is just
the result of exponentiating the log probability.  

Unlike character-based probabilities, this token probability only
counts the probability of the tokens, not the whitespaces (or other
non-token characters) surrounding them.  

A further disconnect for tokenized language models is that they may
use a tokenizer that reduces the signal, conflating some of the
inputs.  In that case, what is being modeled is the probability of
the sequence of tokens, not the raw character sequence.  For example,
any two character sequences producing the same tokens will produce
the same token probability even if they are different strings.

\subsection{Bounded: \code{LanguageModel.Sequence}}

The interface \code{LanguageModel.Sequence} is just a marker
interface, meaning that it does not specify any new methods other than
that inherited from its superintervace, \code{LanguageModel}.

Sequence-based language model implementations are normalized so
that the sum of the probability of all strings of all lengths is one.%
%
\footnote{Because it's just a marker interface, the requirement is
  conventional rather than expressed in Java's type language.  This is
  similar to interface requirements on collections in the
  \code{java.util} core library.}
%
In symbols, this amounts to
%
\begin{equation}
\sum_{n \geq 0} \hspace*{8pt} \sum_{\text{\code{cs.length()} == n}} \hspace*{3pt} 2^{\text{\code{log2Prob(cs)}}} = 1.
\end{equation}

\subsection{Unbounded: \code{LanguageModel.Process}}

The other type of language model is a process language model, declared
by implementing the marker interface \code{LanguageModel.Process}.  A
process language model treats a sequence of characters as being
generated as a kind of random process that generates a character at a
time without a distinct beginning or end.  

Because they don't model the beginning or end of a string, process
language models are normalized by length so that the sum of the
probabilities of all strings of a given length is one.  In symbols,
for every \code{n}, we have
%
\begin{equation}
\sum_{\text{\code{cs.length()} == n}} \hspace*{3pt} 2^{\text{\code{log2Prob(cs)}}} = 1.
\end{equation}
%


\section{Character Language Models}

LingPipe's language models come in two main flavors, character-based
and token-based.  The character-based models are further subdivided as
to whether they model the start and end of a string or not.






