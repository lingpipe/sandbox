\chapter{Tagging}\label{chap:tagging}

Consider the problem of trying to assign a grammatical category such
as noun or verb to every word in a sentence.  This is what is known as
a tagging problem and the labels assigned to the words are called
part-of-speech tags.  

One productive way of viewing a tagging problem is as a structured
classification problem.  A tagging for a given sequence of tokens
consists of a sequence of tags, one for each token.  These tags are
like the categories assigned by a classifier, only now there is a
whole sequence of such categories that must be assigned.

LingPipe provides two standard statistical taggers, hidden Markov
models (HMM) and conditional random fields (CRF); we cover these in
detail in \refchap{hmm} and \refchap{crf}, respectively.  HMMs provide
a generative model of tagging that is analogous to naive Bayes
classifiers.  CRFs provide a discriminative model of tagging that is
closely related to logistic regression.  

Like their analogous classifiers, LingPipe's taggers implement
not only first-best tagging, but also $n$-best tagging at the
sequence level based on sequence probabilities and $n$-best
tagging at the token level based on marginal probabilities.

Like their analogous classifiers, taggers are defined through
a suite of tagging-specific interfaces and evaluated with
interface-specific evaluators.  Before covering HMMs and CRFs,
we introduce the tagging interfaces and evaluators in this
chapter.


\section{Taggings}

LingPipe provides a base class for representing a complete tagging
of a sequence as well as some specialized subclasses which we use
heavily for $n$-best output and natural-language inputs.

LingPipe locates its general tagging interfaces and evaluators in the
package \code{com.aliasi.tag}.  Within that package, the generic class
\code{Tagging<E>} represents taggings of sequences of type \code{E}.
For example, \code{Tagging<String>} provides taggings over sequences
of strings, such as tokens.  

\subsection{Base Tagging Class}\label{section:tagging-tagger}

A tagging is nothing more than a lightweight wrapper for the sequence
of tags assigned to a sequence of symbols.  Unless you write your own
tagger, you will likely only need tagging objects that are produced by
existing taggers.

\subsubsection{Constructing a Tagging}

The constructor \code{Tagging(List<E> syms, List<String> tags)}
creates a tagging consisting of a list of symbols of type \code{E} and
a parallel list of tags of type \code{String}.  These lists must be
the same length, because the tag at position $i$ is taken to
correspond to the symbol at position $i$; in Java terms,
\code{tags.get(i)} is the tag for symbol \code{syms.get(i)}.  If the tags
are not the same length, the constructor will throw an illegal argument
exception.

\subsubsection{Accessing Tags and Symbols}

The entire list of tags and list of tokens are available through the
methods \code{tags()} and \code{tokens()}.  Both of these methods
return immutable views of the underlying tags and tokens as produced
by the static utility method \code{unmodifiableList(List<T>)} in
Java's utility class \code{Collections} in package \code{java.util}.
The return result is a list, but it throws unsupported operation
exceptions if an attempt is made to modify it using methods such as
\code{add()}.

The tags and tokens may also be accessed on a position-by-position
basis.  The method \code{size()} returns the lengths of the lists of
tags and symbols.  For each integer between 0 (inclusive) and the
size of the lists (exclusive), the methods \code{token(int)} and
\code{tag(int)} return the symbol and tag at a specified position.
If an integer out of this range is suppolied, the methods throw
index out of bounds exceptions.

The object method \code{toString()} may be used to render the entire
tagging in a human-readable format (assuming the symbols and tags are
human readable, of course).

Like most of LingPipe's objects, taggings are immutable in the sense
that once they are constructed, their content is constant in the sense
can't be modified.  One advantage of this design is that all of
the tagging classes are entirely thread safe without any synchronization.


\subsection{Scored Taggings}\label{section:tagging-scored-tagging}

The class \code{ScoredTagging<E>} extends \code{Tagging<E>}, and thus
serves as a tagging itself.  Scored taggings also implement the
general LingPipe interface \code{Scored} in \code{com.aliasi.util}.

\subsubsection{Constructing Scored Taggings}

As with regular taggings, these are most commonly created by taggers
rather than by users.  A scored tagging is constructured using
\code{ScoredTagging(List<E>,List<String>,double)}, with the first two
arguments passed to the supertype constructor
\code{Tagging(List<E>,List<String>)} and the double being used for the
score.

Like base taggings, scored taggings are immutable. 

\subsubsection{Accessing Scores}

The \code{Scored} interface defines a single method \code{score()}
returning a double-precision floating-point value.  The advantage of
encapsulating having a score in an interface is that it makes it
easy to sort \code{Scored} instances by their scores.  For example,
there are predefined comparators for scored objects, accessible
through static methods \code{ScoredObject.comparator()} and
\code{ScoredObject.reverseComparator()} for sorting in increasing or
decreasing order of score respectively.  The class \code{ScoredObject<E>}
provides a general implementation of \code{Scored} that also holds
an immutable reference to an object of type \code{E}.


\subsection{String Taggings}

The class \code{StringTagging} also extends the base
\code{Tagging<E>}, but with the generic type \code{E} fixed to
\code{String}.  In addition to the toekns and tags, a string tagging
stores an underlying character sequence from which the tokens were
derived along with the start and end position of each token in the
sequence.  

\subsubsection{Constructing String Taggings}

A string tagging is construccted using
\code{StringTagging(List<String> toks, List<String> tags, CharSequence
  cs, int[] tokenStarts, int[] tokenEnds)}.  The tokens and tags are
passed to the superclass constructor
\code{Tagging(List<String>,List<String>)}, so must be the same length.
The arrays of token starts and token ends must also be the same length
as the lists of tokens and tags; if they aren't, an illegal argument
exception will be thrown.  The character sequence is copied into a
local string object.

Like the other tagging classes, string taggings are immutable.

\subsubsection{Underlying String and Tokens}

The method \code{characters()} returns a \code{String}
corresponding to the sequence of charcters undelrying this tagging.

The methods \code{tokenStart(int)} and \code{tokenEnd(int)} provide
the position of a token, with the start position being inclusive and
the end position being exclusive.  The method \code{rawToken(int)} is
a convenience method to return the slice of the underlying string
spanning the token at the specified psoition.  As with the tag and
token access methods, these methods throw index out of bounds
execptions if the indices are out of range.

The string tagging class overrides \code{toString()} to provide
human-readable output.


\subsubsection{Equality and Hash Codes}

The basic \code{Tagging} and extended \code{ScoredTagging} classes
define equality by refernece (by not overriding the definition of
\code{equals(Object)} inherited from \code{java.util.Object}).  That
is, a tagging and scored tagging are only equal if they are reference
equal.  

String taggings have an equality method defined in terms of their
contents, with a corresponding hash code method.


\section{Tag Lattices}

The \code{Tagging} class provides an output representating for
first-best taggers that return their best estimate of a tagging.  The
\code{ScoredTagging} class is sufficient for $n$-best output, which
consists of a sequence of taggings with scored attached.  For marginal
taggers, output takes the form of a tag lattice.  A tag lattice
encapsulates a rich computation of probabilities for individual
tags given the rest of the output.

In this section, we describe the abstract class \code{TagLattice<E>}.
Like for regular taggings, the generic type parameter \code{E}
represents the type of symbols being tagged.  Additional methods
provide the means to estimate the probabilities of specific tags.

\subsection{Constructing Tag Lattices}

Tag lattices are constructed by marginal taggers and returned to
clients.  If you are not defining your own marginal tagger, you should
not need to construct a tag lattice.  There is a private
implementation in the \code{tag} package if you'd like to examine a
simple implementation.  

There are concrete implementations for both HMMs and CRFs.  The
HMM-specific subclass is not public, so is not accessible through the
API.  If you're curious, the code is availabe in the class
\code{TagWordLattice} in the package \code{com.aliasi.hmm}.  There is
a public subclass \code{ForwardBackwardTagLattice} in the \code{crf}
package which we describe along with conditional random fields in
\refchap{crf}; it provides some extra functionality beyond that provided
by the abstract base class described in this section.

\subsection{Accessing Tag Probabilities}

The primary service provided by tag lattices is encapsulating
probability estimates for tags for individual symbols.  For instance,
consider the sentence \stringmention{The dog barked.}.  With a
standard English tokenization, such as LingPipe's built in
\code{IndoEuropeanTokenizerFactory}), the sentence consists of four
tokens \stringmention{The}, \stringmention{dog},
\stringmention{barked}, and \stringmention{.}  

Given this sentence, with LingPipe's HMM and CRF taggers, it's
possible to estimate the probabilty that the token
\stringmention{barked} is a past tense verb as well as the probability
that it is a past particple verb.  Note that because
\stringmention{barked} may be used for simple past tense or the past
participle, it's not possible to determine the category by looking at
just the word.  Instead, we want to know what the chance
\stringmention{barked} is a past participle in the context of the
sentence \stringmention{The dog barked.} (the answer is that it should
be pretty low; the existence of a preceding noun and the absence of an
auxiliary verb form of \stringmention{have} are both evidence that the
word in question is a simple past tense usage. 

\subsubsection{Underlying Tokens}

The underlying set of tokens may be retrieved using methods similar to
the simple taggings.  The number of tokens in the input is provided by
the method \code{numTokens()}. The token at a given position is
retrieved using \code{token(int)}, which returns an instance of
\code{E}, the type of symbols in the tagging.  An immutable view of
the entire list of tokens is given by \code{tokenList()}.

\subsubsection{Per-Token Marginal Probability Estimates}

The simplest way to access the probability estimates for each tag for
a given token is through the method \code{tokenClassification(int)}.
Given an index, this method returns an instance of
\code{ConditionalClassification}, which is able to provide
the probability of each category for the token given the context
of the entire input string (see \refsec{conditional-classifier}
for more conditional classifications).

\subsubsection{Tag Set as Symbol Table}

Because a tag lattice provides a probability estimate for each tag for
each token position, it is helpful to be able to access the entire set
of possible tags.  The tags are represented in a symbol table, a copy
or immutable view of which is returned by \code{tagSymbolTable()}; see
\refchap{symbol-tables} for an overview of LingPipe's symbol tables.
The basic idea of the symbol table is that it lets you retrieve an
integer identifier for each tag and vice-versa.

It's also possible to retrieve the sequence of tags in symbol-table
order using the method \code{tagList()}, which returns a list of strings.

\subsubsection{Simple Tag Probabilities}

The method \code{logProbability(int token, int tag)} returns the
(natural) log of the probability of the token with the specified index
being assigned to the tag with the specified identifier.

\subsubsection{Tag Sequence Probabilities}

Because the lattice stores such a rich set of probability information,
it is possible to extend the calculation of the probability of a
single category for a single token to the calculation of a sequence of
categories for a sequence of tokens, all conditional on the entire
input sequence.  The method \code{logProbability(int n, int[] tags)}
returns the (natural) log probability of the sequence of tags starting
at position \code{n}.  The tag positions must all be within the
sequence, so that $0 \leq \mbox{\code{n}}$ and $\mbox{\code{n}} +
\mbox{\code{tags.length}} \leq \mbox{\code{numTokens}}$.  


\section{Taggers}

The \code{com.aliasi.tag} package defines three tagging interfaces,
corresponding to the three kinds of tag outputs.  The two statistical
tagging implementations in LingPipe, hidden Markov models (HMM) and
conditional random fields (CRF), implement all three interfaces.  

The interfaces are broken out to allow simpler run-time interfaces.
For instance, this lets us define separate evaluation interfaces for
each mode of operation, as well as leaving open the option for
taggers that do not implement all three interfaces.

\subsection{First-Best Taggers}

The first-best tagging interface is \code{Tagger<E>}, with the
type parameter \code{E} determining the kind of tokens that
are tagged.  The interface defines a single method,
%
\begin{verbatim}
Tagging<E> tag(List<E> tokens);
\end{verbatim}
%
That is, given a list of tokens of type \code{E}, return a first-best
tagging over that type.  The return type \code{Tagging} is described
in \refsec{tagging-tagger}.

\subsection{$N$-Best Taggers}

The $n$-best tagging interface is \code{NBestTagger<E>}, and
like the first-best interface, the type parameter \code{E}
determines the type of tokens that are tagged.  The $n$-best
tagger interface extends the simple \code{Tagger<E>} interface,
so that any $n$-best tagger may be used as a simple tagger.

There are two methods specified in this interface, differing in terms
of their score normalizations.

The first method is
%
\begin{verbatim}
Iterator<ScoredTagging<E>> tagNBest(List<E>,int);
\end{verbatim}
%
where the input list is the list of tokens to parse and the integer
argument specifies the maximum number of results to generate.  Setting
the maximum number of results lower than \code{Integer.MAX\_VALUE} can
save space during tagging by allowing taggings that are known to be
beyond the result set to be pruned.  

The return result is an iterator over scored taggings, which are
described in \refsec{tagging-scored-tagging}.  Essentially, a scored
tagging combines a tagging with a score.  The interface specifies that
the scored taggings are returned in decreasing order of score.  Thus
the best tagging is returned first, the second-best next, and so on.
This is what allows any $n$-best tagger to implement first-best
tagging; it just returns the first-result from \code{tagNBest()} as
the result for the superinterface's \code{tag()} method.%
%
\footnote{This presupposes that there is always at least one tagging
  result, which is the case for all of the taggers of which we are
  aware.  While the \code{tag()} method could be defined to return
  \code{null}, that would not play nicely with other interfaces which
  expect non-null taggings.}

The second method specifiedin the interface has a different name, but
with identical argument and result types,
%
\begin{verbatim}
Iterator<ScoredTagging<E>> tagNBestConditional(List<E>,int);
\end{verbatim}
%
The difference between this method and the simple \code{tagNBest()}
method is that this method requires the scores to be conditional
probabilities of the tagging given the input tokens.%
%
\footnote{Both the conditional probability normalization and the
order are informal requirements, because they are beyond the expressive
power of Java's type system.}
%
The difference between $n$-best and conditionally normalized $n$-best
taggers is the same as that between scored classifiers and conditional
classifiers (see \refsec{conditional-classifier}).

The \code{tagNBestConditional()} method is specified in the interface
to throw an \code{UnsupportedOperationException} if the method is not
supported.  Thus it's possible to write an unnormalized $n$-best
tagger.  


\subsection{Marginal Taggers}

The third tagging interface is \code{MarginalTagger<E>}.  It produces
an entire tag lattice as output in oen method call, specified by the
single interface method
%
\begin{verbatim}
TagLattice<E> tagMarginal(List<E>);
\end{verbatim}
%
As with the other interfaces, the type parameter \code{E} picks out
the type of the tokens that are being tagged.  

The marginal tagger interface does not extend the basic tagging
interface.  That's because it's not in general possible to extract
a coherent first-best result from a tag lattice, because the
first-best tags on a per-token basis may not be consistent as an
entire tag sequence;  we'll see examples when we consider using taggers
for chunking.


\section{Tagger Evaluators}

Like classifier evaluators, which we discussed in
\refchap{classifier-evaluation}, there is a separate tagger evaluator
for each of the tagging interfaces.  Each of the the three interfaces is
named for the tagger interface it evaluates.

\subsection{First-Best Tagger Evaluation}

The simplest evaluator is the \code{TaggerEvaluator<E>}, which evaluates
taggers over type \code{E} tokens.  

As with the classifier evaluators, a tagger evaluator is constructed
from a tagger.  The constructor,
%
\begin{verbatim}
TaggerEvaluator(Tagger<E>,boolean);
\end{verbatim}
%
takes an argument for the tagger that will be evaluated.  This tagger
may be null at construction time and reset later using the method
\code{setTagger(Tagger<E>)}.  The tagger must be either supplied in the
constructor or set before cases are evaluated.  The second argument
indicates whether the tokens in the test cases are saved along with
the tags.

\subsubsection{Adding Test Cases}

Like the classifier evaluators, the tagger evaluators implement object
handlers, in this case implementing the interface
\code{ObjectHandler<Tagging<E>{>}} (see \refsec{corpus-handlers}),
which specifies the single method
%
\begin{verbatim}
void handle(Tagging<E>);
\end{verbatim}
%
As explained in \refsec{corpus-handlers} and the rest of
\refchap{corpus}, this allows the handlers to be plugged together with
corpora and parsers.

The second way to add test cases is directly through the \code{addCase()}
method, which has signature
%
\begin{verbatim}
void addCase(Tagging<E> ref, Tagging<E> resp);
\end{verbatim}
%
The two tagging arguments represent the gold-standard reference
tagging and the system's response, respectively.


\subsubsection{Basic Input Profile}

A tagger evaluator has various utility methods that provide
information about the data that has been seen so far.  The method
\code{numCases()} returns the number of examples seen so far, which is
just the number of calls to \code{handle()} or \code{addCase()}).
The method \code{numTokens()} represents the total number of tokens
seen so far.  The method \code{tags()} returns a list of the tags
seen so far, each of which is a string.  

The method \code{inputSizeHistogram()} returns a histogram, represented
as an object to counter map of integers,
%
\begin{verbatim}
ObjectToCounterMap<Integer> inputSizeHistogram();
\end{verbatim}
%
The result maps integers representing the number of tokens in an input
to the number of cases seen so far with that many input tokens.

The underlying tagger is returned by the method \code{tagger()} and
the flag indicating whether or not to store tokens is returned
by \code{storeTokens()}.

\subsubsection{Evaluations}

There are several methods that return the evaluation results.  The
simplest is \code{caseAccuracy()}, which returns a \code{double}
between 0 and 1 representing the fraction of cases that were tagged
completely correctly.  A response tagging is case correct if every
tag matches matches the corresponding reference tagging.  

The method \code{tokenEval()} returns an evaluation on a
token-by-token basis.  It treats the tagging of each token as a
classification problem, and hence is able to return an instance of
\code{BaseClassifierEvaluator<E>}, where \code{E} is the type
of the tokens.  This provides all the evaluations that are available
for base classifiers.

The token-by-token evaluation can be restricted to a set of tokens not
in a given set.  The main use for this is to evaluate performance on
tokens that were not seen in the training data.  The method
%
\begin{verbatim}
BaseClassifierEvaluator<E> unknownTokenEval(Set<E>);
\end{verbatim}
%
takes a set of tokens as an argument, and returns an evaluation restricted
to tokens not in the argument set.  If the argument set is the set of
tokens in the training data, the resulting evaluation is on tokens
that are not in the training set.  



