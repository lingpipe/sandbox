\mychapter{Introduction}{INTRODUCTION}

\firstchar{L}ingPipe is a Java-based software framework and
implementation for processing natural (human) language data.  This
book provides a comprehensive tutorial on how to use LingPipe.  

\mysection{The Java Programming Language}

\noindent
The presentation here assumes you have a basic familiarity with the
Java programming language.  We chose Java as the basis for LingPipe
because we felt it provided the best tradeoff between programmer
efficiency, scalability, portability and usability.  

On the other hand, several components of Java relevant for processing
textual data and using LingPipe are discussed in detail, such as
character-level I/O, serialization and generic types.  Several
libraries are also discussed in depth as parts of applications.

\mysubsection{Java Standard Edition 6}

\noindent
This book is based on the latest currently supported standard edition
of the Java platform (Java SE), which is version 6.  You will need the
Java development kit (JDK) in order to compile programs.  The JDK
includes the Java runtime environment (JRE), which you will need to
run programs.  

\mysubsection{Downloading Java}

\noindent
Java is available for the Windows, Linux, and Solaris operating
systems in both 32-bit and 64-bit versions.  Download whichever is
appropriate for your hardware and operating system, keeping in mind
that you will need the 64-bit version to use large (greater than 1.5
gigabyte or so) heap sizes.

Java is available for download for Windows, Linux and Solaris from

\begin{quote}
\hrefurl{http://java.sun.com/javase/downloads/index.jsp}
\end{quote}

The Java JDK and JRE are included as standard features on the
Macintosh operating system as part of Mac OS X.  Updates are
available through Apple's standard developer channel at
%
\begin{quote}
\hrefurl{http://developer.apple.com/java/download/}
\end{quote}
%
and frequently asked questions for Java on Mac OS X are answered at
%
\begin{quote}
\hrefurl{http://developer.apple.com/java/faq/}
\end{quote}

Licesning terms for the JDK are available from
%
\begin{quote}
\hrefurl{http://java.sun.com/javase/6/jdk-6u20-license.txt}
\end{quote}
%
and for the JRE at
%
\begin{quote}
\hrefurl{http://www.java.com/en/download/license.jsp}
\end{quote}

\mysubsection{Learning Java}

For learning Java, we recommend the following two books pitched at
beginners and intermediate Java programmers,
%
\begin{itemize}
\item Arnold, Ken, James Gosling and David Holmes.  2005.
{\it The Java Programming Language}, 4th Edition.  Prentice-Hall.
\item Bloch, Joshua. 2008. {\it Effective Java}, 2nd Edition.  Prentice-Hall.
\end{itemize}
%
James Gosling was the co-inventor of Java and Joshua Bloch was
one of the main architects of the language.





\mysection{Downloading LingPipe}

\noindent
LingPipe may be downloaded from its web site:

\begin{quote}
\hrefurl{http://alias-i.com/lingpipe/}
\end{quote}
