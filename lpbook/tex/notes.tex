
\section{Differential Calculus}

\subsection{Derivatives}

Derivatives measure the slope of a function.  They have a simple
definition as a limit of the slope between two points $x$ and
$x+\delta$ as $\delta \rightarrow 0$.  Given a function $f:\reals
\rightarrow \reals$, we write its derivative using Leibniz's notation
as $\frac{d}{dx}f(x)$ and using Lagrange's notation as $f'(x)$.  The definition is
%
\begin{equation}
f'(x)
= \frac{d}{dx} f(x) 
= \lim_{\delta \rightarrow 0} 
    \frac{f(x+\delta) - f(x)}{\delta}.
\end{equation}
%
If this limit converges at a point $x$, the function $f$ is
said to be differentiable at $x$.

We provide a table of common derivative forms in 

\subsection{Derivates of Common Functions}


%
\begin{equation}
f'(x^n) =
\= n \ x^{n-1}.
\end{equation}
%

For exponential functions, we have
%
\begin{equation}
\frac{d}{dx} \exp(x) = \exp(x).
\end{equation}

For logarithms, we have 
%
\begin{equation}
\frac{d}{dx} \log x = 1/x 
\ \ \ \ \  [\mbox{ for} x > 0] \hfill
\end{equation}

For sums of two functions, we have
%
\begin{equation}
\frac{d}{dx}(f(x) + g(x)) = \frac{d}{dx}f(x) + \frac{d}{dx}g(x)
\end{equation}

For the product of two functions, we have
%
\begin{equation}
\frac{d}{dx} f(x)g(x) = g(x) \frac{d}{dx} f(x) + f(x) \frac{d}{dx}g(x).
\end{equation}

For the composition of two functions, it's easier to resort to
prime notation, 
%
\begin{equation}
\frac{d}{dx} f(g(x)) = \frac{d}{dx} g(x) \times (\frac{d}{dx} h(x))(g(x)).
\end{equation}
%
This latter formula is most naturally expressed using Lagrange's
notation and function composition as $(f \circ g)'(x) = g'(x)
h'(g(x))$.

Multiplication by a constant distributes through differentiation, so
that
%
\begin{equation}
\frac{d}{dx} k f(x) = k \frac{d}{dx} f(x)
\end{equation}
%
Similarly, addition 





\begin{equation}
\frac{d}{dx} c = 0 \mbox{ for any constant } c
\end{equation}


