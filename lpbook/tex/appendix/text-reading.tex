\chapter{Further Reading}\label{appendix:reading}

\firstchar{I}n order to fully appreciate contemporary approaches to natural language
processing requires three fairly broad areas of expertise:
linguistics, statistics, and algorithms.  We don't pretend to
introduce these areas in any depth in this book.  Instead, we
recommend the following textbooks, sorted by area, from among the many
that are currently available.

\section{Unicode}

Given that we're dealing with text data, and that Java uses Unicode
internally, it helps to be familiar with the specification.  Or at
least know where to find more information.  The actual specification
reference book is quite readable.  The latest available in print is
for version 5.0,
%
\begin{itemize}
\bibanno{Unicode Consortium, The.  
{\it The Unicode Standard 5.0}.  
Addison-Wesley.}
{This is the final word on the unicode standard, and
quite readable.}
\end{itemize}
%
You can also find the latest version (currently 5.2) online at
%
\urldisplay{http://unicode.org}
%
This includes all of the code tables for the latest version, which are
available in PDF files or through search, at 
%
\begin{quote}
\hrefurl{http://unicode.org/charts/}
\end{quote}


\section{Java}\label{section:reading-java}

\subsection{Overviews of Java}
%
For learning Java, we recommend the following three books pitched at
beginners and intermediate Java programmers,
%
\begin{itemize}
\bibanno{Arnold, Ken, James Gosling and David Holmes.  2005.
{\it The Java Programming Language}, 4th Edition.  Prentice-Hall.}
{Gosling invented Java, and this is the ``official'' introduction to the
language.  It's also very good.}
%
\bibanno{Bloch, Joshua. 2008. {\it Effective Java}, 2nd Edition.  Prentice-Hall.}
{Bloch was one of the main architects of Java from
version 1.1 forward.  This book is about taking Java program design
to the next level.}
%
\bibanno{Fowler, Martin, Kent Beck, John Brant, William Opdyke, and Don Roberts.
1999.
{\it Refactoring: Improving the Design of Existing Code}.
Addison-Wesley.}
{Great book on making code better, both existing
code and code yet to be created.}
\end{itemize}
%

\subsection{Books on Java Libraries and Packages}\label{appendix:java-libs}

\noindent
There is a substantial collection of widely used libraries for Java.
Here are reference books for the Java libraries we use in this book.
%
\begin{itemize}
%
\bibanno{
Loughran, Steve and Erik Hatcher. 2007
{\it Ant in Action}.  Manning.}
{This is technically the second edition of their 2002
book, {\it Java Development with Ant}.  Both authors are committers for
the Apache Ant project.}
%
\bibanno{Tahchiev, Peter, Felipe Leme,
Vincent Massol, and Gary Gregory. 2010.
{\it JUnit in Action}, Second Edition.  Manning.}
{A good overview for JUnit.  Should be up-to-date with
the latest attribute-driven versions, unlike either Kent Beck's book
(he's the author of Ant).  As much as we liked {\it The Pragmatic
Programmer}, Hunt and Thomas's book on JUnit is not particularly useful.}
%
\bibanno{McCandless, Michael, Erik Hatcher and Otis Gospodneti\'c.
2010.
{\it Lucene in Action}, Second Edition. Manning.}
{A fairly definitive overview of the Apaceh Lucene
search engine by three project committers.}
%
\bibanno{Hunter, Jason and William Crawford. 2001. {\it Java Servlet Programming}, Second Edition. O'Reilly.}
{It only covers version 2.2, so it's a bit out of date
relative to recent standards, but the basics haven't changed, and
although there are more elementary introductions, this book is a very
good introduction to what servlets can do.  It also explains HTTP
itself quite clearly.  The first author is an Apache Tomcat developer
and was involved in developing the original servlet standard.}
%
\bibanno{Goetz, Brian, Tim Peierls, Joshua Bloch, Joseph Bowbeer, David Holmes, and Doug Lea. 2006. {\it
Java Concurrency in Practice}.  Addison-Wesley.}
{This fantastic book covers threading and concurrency
patterns and their application, along with implementations in Java's
\code{java.util.concurrent} package.  Doug Lea was the original author
of \code{util.concurrent} before it was brought into Java itself, and
his previous book, {\it Concurrent Programming in Java}, Second
Edition (1999; Prentice Hall), is wroth reading for a deeper theoretical
understanding of concurrency in Java.}
%
\bibanno{Hitchens, Ron. 2002.  {\it Java NIO}. O'Reilly.}
{A concise and readable introduction to the ``new'' I/O package, \code{java.nio} and
the concepts behind it, like buffers and multicast reads.}
%
\bibanno{McLaughlin, Brett and Justin Edelson. 2006. {\it Java and XML}, Third Edition. O'Reilly.}
{A good introduction to the set of tools available in
Java for parsing and generating XML with SAX and DOM, specifying structure
with DTDs and XML Schema, transforming XML with XSLT, as well as the general 
high-level libraries JAXP and JAXB.}
\end{itemize}

\subsection{General Java Books}

There are also general-purpose books about Java that do
not focus on particular classes or applications.
%
\begin{itemize}
%
\bibanno{Shirazi, Jack. 2003. {\it Java Performance Tuning}, 2nd
  Edition. O'Reilly} 
{Although already dated, it provides a good introduction to
  profiling and thinking about performance tuning.  The author and friends
  also run the web site \code{javaperformancetuning.com}, which is a
  monthly digest of Java performance tuning tips (and advertising).
  If you know C, start with Jon Bentley's classic, {\it Programming
  Pearls} (Addison-Wesley).}
%
\end{itemize}


\subsection{Practice Makes Perfect}

\noindent
Really, the only way to learn how to code is to practice.  Being a
good coder means two things: being fast and efficient (assuming
accuracy).  By accurate, we mean writing code that does what it's
supposed to do.  By fast, we mean the coding goes quickly.  By efficient,
we mean the resulting code is efficient.

For practice, we highly recommend the algorithm section of the
TopCoder contests.  TopCoder provides a complete online Java
environment, well thought out problems at varying specified levels of
difficulty, unit tests that check your submissions, and examples of
what others did to provide code-reading practice.
%
\begin{quote}
\hrefurl{http://www.topcoder.com/tc}
\end{quote}
%



\section{General Programming}

\noindent
We would also recommend a book about programming in general, 
especially for those who have not worked collaboratively with
a good group of professional programmers.
%
\begin{itemize}
\bibanno{Hunt, Andrew and David Thomas.
1999.
{\it The Pragmatic Programmer}.
Addison-Wesley.}
{Read this book and follow its advice, which is just as relevant now
as when it was written.  The best book on how to program
we know.  A competitor to books like Steve McConnell's {\it Code Complete} (2004)
and books on ``agile'' or ``extreme'' programming.}
%
\bibanno{Collins-Susman, Ben, Brian W.\ Fitzpatrick,
and C. Michael Pilato.
2010.
Collins-Sussman et al.'s
\booktitle{Version Control with Subversion}. Subversion 1.6.
CreateSpace.}
{This is the official guide to Subversion from the
authors.  It's also available online at
%
\urldisplay{http://svnbook.red-bean.com/}
}
\end{itemize}


%


\section{Algorithms}

\noindent
Because natural language processing requires comptuational
implementations of its models, a good knowledge of algorithm analysis
goes a long way in understanding the approaches found in a tool kit
like LingPipe.

\begin{itemize}
%
\bibanno{Cormen, Thomas H., Charles E.\ Leiserson, Ronald L.\ Rivest, and
Clifford Stein. 
2009.
{\it Introduciton to Algorithms}, Third Edition.
MIT Press.}
{A thorough yet readable introduction to algorithms.}
%
\bibanno{Durbin, Richard, Sean R.\ Eddy, Anders Krogh, Grame Mitchison.
{\it Biological Sequence Analysis: Probabilistic Models of Proteins
and Nucleic Acids}.
Cambridge University Press.}
{Even though it focuses on biology, it has great
descriptions of several key natural language processing algorithms
such as hidden Markov models and probabilistic context-free grammars.}
%
\bibanno{Gusfield, Dan.
1997.
{\it Algorithms on Strings, Trees and Sequences}.
Cambridge University Press.}
{Incredibly detailed and thorough introduction to
string algorithms, mainly aimed at computational biology, but
useful for all string and tree processing.}
%
\end{itemize}






\section{Probability and Statistics}

\noindent
Most of the tools in LingPipe are statistical in nature.  To fully
understand modern natural language processing it's necessary to
understand the statistical models which make up their foundation.
To do this requires an understanding of the basics of statistics.
Much of the work in natural language processing is Bayesian, so
understanding Bayesian statistics is now a must for understanding
the field.  Much of it also relies on information theory, which is
also introduced in the machine learning textbooks.

\begin{itemize}
%
\bibanno{
Bulmer, M.~G.
1979.
{\it Principles of Statistics}.
Dover.}
{Packs an amazing amount of information into
a readable little volume, though it assumes some mathematical
sophistication.}
%
\bibanno{Cover, Thomas M.\ and Joy A.\ Thomas.
2006.
{\it Elements of Information Theory}, Second Edition.
Wiley.}
{This is the classic textbook for information theory.}
%
\bibanno{DeGroot, Morris H.\ and Mark J.\ Schervish.
2002.
{\it Probability and Statistics},  Third Edition.
Addison-Wesley.}
{Another introduction to classical frequentist
statistical hypothesis testing and estimation but with some Bayesian
concepts thrown into the mix.}
%
\bibanno{Gelman, Andrew and Jennifer Hill.
2006.
{\it Data Analysis Using Regression and Multilevel/Hierarchical Models}.
Cambridge University Press.}
{A great practical introduction to regression from a Bayesian
perspective with an emphasis on multilevel models.  Much less mathematically
demanding than the other introductions, but still very rigorous.}
%
\bibanno{Gelman, Andrew, John B.~Carlin, Hal S.\ Stern, and Donald B.\ Rubin.
2003.
{\it Bayesian Data Analysis}, Second Edition.
Chapman and Hall.}
{The bible for Bayesian methods.  Presupposes high
mathematical sophistication and a background in classical probability
and statistics to the level of DeGroot and Schervish or Larsen
and Marx's books.}
%
\bibanno{Larsen, Richard J.\ and Morris L.\ Marx.
2005.
{\it Introduction to Mathematical Statistics and Its Applications}, Fourth Edition.
Prentice-Hall.}
{A nice intro to classical frequentist statistics.}
%

\bibanno{Agresti, A. 2002. {\it Categorical Data Analysis}, Second Edition. Wiley.}
{A thorough survey of classical approaches to analyzing categorical data,
focusing on contingency tables.}

\end{itemize}


\section{Machine Learning}

\noindent
Machine learning, in contrast to statistics, focuses on implementation
details and tends to concentrate on predictive inference for larger
scale applications.  As such, understanding machine learning requires
a background in both statistics and algorithms.

\begin{itemize}
\bibanno{Bishop, Christopher M.
2007.
{\it Pattern Recognition and Machine Learning}.
Springer.}%
{Introduction to general machine learning
techniques that is self contained, but assumes more mathematical
sophistication than this book.}
%
\bibanno{Hastie, Trevor, Robert Tibshirani and Jerome Friedman.
2009.
{\it The Elements of Statistical Learning}, Second Edition.
Springer.}
{Presupposes a great deal of mathematics, providing
fairly rigorous introductions to a wide range of algorithms and
statistical models.}
%
\bibanno{MacKay, David J.~C.
2002.
{\it Information Theory, Inference, and Learning Algorithms}.
Cambridge University Press.}
{More of a set of insightful case studies than a coherent textbook.
Presupposes a high degree of mathematical sophistication.}
%
\bibanno{Witten, Ian H.\ and Eibe Frank. 2005. {\it Data Mining: Practical
Machine Learning Tools and Techniques} (Second Edition).  Elsevier.}
{The most similar to this book, explaining general machine
learning techniques at a high level and 
linking to the Java implementations (from their widely used Weka toolkit).}
%
\end{itemize}




\section{Linguistics}

\noindent
Natural language processing is about language.  To fully appreciate
the models in LingPipe, it is necessary to have some background in
linguistics.  Unfortunately, most of the textbooks out there are based
on traditional Chomskyan theoretical linguistics which divorces
language from its processing, and from which natural language
processing diverged several decades ago.

\begin{itemize}
%
\bibanno{Bergmann, Anouschka, Kathleen Currie Hall, and Sharon Miriam Ross (editors).
2007.
{\it Language Files}, Tenth Edition.
Ohio State University Press.}
{Easy to follow and engaging overview of just about every
aspect of human language and linguistics.  Books on specific areas of
linguistics tend to be more theory specific.}
%
\end{itemize}


\section{Natural Language Processing}

\noindent
Natural language processing is largely about designing and coding computer
programs that operate over natural language data.  Because most NLP is
statistical these days, this requires not only a knowledge of linguistics,
but also of algorithms and statistics.

\begin{itemize}
\bibanno{Jurafsky, Daniel, James H.\ Martin.
2008.
{\it Speech and Language Processing}, Second Edition.
Prentice-Hall.}
{High-level overview of both speech and language
processing.}
%
\bibanno{Manning, Christopher D., Raghavan Prabhakar, and Hinrich Sch\"utze.
2008.
{\it Introduction to Information Retrieval}.
Cambridge University Press.}
{Good intro to IR and better intro to some NLP than
Manning and Sch\"utze's previous book, which is starting to feel
dated.}
%
\bibanno{Manning, Christopher D.\ and Hinrich Sch\"utze.
1999.
{\it Foundations of Statistical Natural Language Processing}.
MIT Press.}
{Needs a rewrite to bring it up to date, but still
worth reading given the lack of alternatives.}
\end{itemize}

% HACK TO GET RID OF UGLINESS ON MARKED PAGES
% \clearpage
% \thispagestyle{empty}
% \cleardoublepage
