%%%%%%%%%%%%%%%%%%%%%%%%%%%%%%%%%%%%%%%%%%%%%%%%%%%%%%%%%%%%%%%%%%%%%%
% Full Paper LaTeX2e Template for ``Genome Informatics Vol. 17''        %
% Universal Academy Press, Inc.                                      %
%%%%%%%%%%%%%%%%%%%%%%%%%%%%%%%%%%%%%%%%%%%%%%%%%%%%%%%%%%%%%%%%%%%%%%

%%%%%%%%%%%%%%%%%%%  Read General Remarks  %%%%%%%%%%%%%%%%%%%%%%%%%%%
% 1. 11pt fonts should be used.                                      %
% 2. When special style files are used (e.g. epsf.sty),              %
%    please send the style files together.                           %
% 3. Even if color images are used, only black/white images          %
%    will be used for publication.                                   %
% 4. If you want to include figures in your article, please          %
%    use graphicx.sty or graphics.sty. You can take these macros     %
%    from our site in the same directory.(graphics.tar.gz)           %
%%%%%%%%%%%%%%%%%%%%%%%%%%%%%%%%%%%%%%%%%%%%%%%%%%%%%%%%%%%%%%%%%%%%%%

%%%%%  BEGIN Do not change  %%%%%%%%%%%%%%%%%%%%%%%%%%%%%%%%%%%%%%%%%%
\documentclass[twoside,11pt]{article}
\usepackage{graphicx}
\usepackage{url}
\setlength{\topmargin}{-1cm}
\setlength{\oddsidemargin}{-0.5cm}
\setlength{\evensidemargin}{-0.5cm}
\setlength{\textwidth}{17cm}
\setlength{\textheight}{24cm}
\pagestyle{myheadings}
\author{}
\date{}
\setcounter{page}{1}
%%%%%  END Do not change %%%%%%%%%%%%%%%%%%%%%%%%%%%%%%%%%%%%%%%%%%%%%

%%%%%%%%%%%%%%%%%%%     \title    %%%%%%%%%%%%%%%%%%%%%%%%%%%%%%%%%%%%
\title{\bf
Full Paper LaTeX2\boldmath{$\varepsilon$} Template for \\
Genome Informatics 2006
}
%%%%%%%%%%%%%%%%%%%%%%%%%%%%%%%%%%%%%%%%%%%%%%%%%%%%%%%%%%%%%%%%%%%%%%

%%%%%%%%%%%%%%%%%%%   \markboth    %%%%%%%%%%%%%%%%%%%%%%%%%%%%%%%%%%%
\markboth{LastName1 {\em et al.}}
         {Full Paper LaTeX2$\varepsilon$ Template}
% First  parameter: Single author         -> LastName1
%                   Two authors           -> LastName 1 and LastName2
%                   More than two authors -> LastName1 {\em et al.}
% Second parameter: Provide a short running head with length not
%                   greater than 45 letters.
%%%%%%%%%%%%%%%%%%%%%%%%%%%%%%%%%%%%%%%%%%%%%%%%%%%%%%%%%%%%%%%%%%%%%%


%%%%%  BEGIN Do not change  %%%%%
\begin{document}
\maketitle
\thispagestyle{myheadings}
\vspace{-1.8cm}
%%%%%  END Do not change %%%%%%%%

%%%%%%%%%% Author(s) and Email Address(es) %%%%%%%%%%%%%%%
% Remark: When paper is single authored, the affiliation %
%         need not be footmarked.                        %
%%%%%%%%%%%%%%%%%%%%%%%%%%%%%%%%%%%%%%%%%%%%%%%%%%%%%%%%%%
\begin{center}
%\begin{tabular}[t]{c}                      % One person in one line
\begin{tabular}[t]{c@{\extracolsep{2em}}c}  % two persons in one line
%\begin{tabular}[t]{c@{\extracolsep{2em}}c@{\extracolsep{2em}}c}
                                            % three persons in one line
  \bf    FirstName1 MiddleName  LastName1\footnotemark[1]
 &\bf    FirstName2             LastName2\footnotemark[2]
%&\bf    FirstName3 MiddleName3 LastName3\footnotemark[1]
\\
  \small\tt lastname1@ims.u-tokyo.ac.jp
 &\small\tt lastname2@ims.u-tokyo.ac.jp
%&\small\tt lastname3@ims.u-tokyo.ac.jp
\end{tabular}
\smallskip

%%%%%%%%%%%%% More Authors, if any %%%%%%%%%%%%%%%%%%%%%%%%%%%%
\begin{tabular}[t]{c}                        % 1 person/line
%\begin{tabular}[t]{c@{\extracolsep{2em}}c}  % 2 persons/line
%\begin{tabular}[t]{c@{\extracolsep{2em}}c@{\extracolsep{2em}}c}
                                             % 3 persons/line
  \bf    FirstName3             LastName3\footnotemark[3]
%&\bf    FirstName5 MiddleName5 LastName5\footnotemark[3]
%&\bf    FirstName6             LastName6\footnotemark[3]
\\
  \small\tt lastname3@jsbi.org
%&\small\tt lastname5@jsbi.org
%&\small\tt lastname6@jsbi.org
\end{tabular}
\smallskip
%%%%%%%%%%%%%%%%%%%%%%%%%%%%%%%%%%%%%%%%%%%%%%%%%%%%%%%%%%%%%%%



%%%%%%%%%%%%%%%%%%% Affiliation(s) %%%%%%%%%%%%%%%%%%%%%%%%%%%%
\begin{small}
\begin{tabular}{rl}
\footnotemark[1] & \parbox[t]{13cm}{
%%%%%%%%%% Affiliation for footnotemark[1] %%%%%%%%%%%%%%%%%%%%
Human Genome Center, Institute of Medical Science, University of
Tokyo, 4-6-1 Shirokanedai, Minato-ku, Tokyo 108-8639, Japan
%%%%%%%%%%%%%%%%%%%%%%%%%%%%%%%%%%%%%%%%%%%%%%%%%%%%%%%%%%%%%%%
}\\
\footnotemark[2] & \parbox[t]{13cm}{
%%%%%%%%%% Affiliation for footnotemark[2] %%%%%%%%%%%%%%%%%%%%
San Diego Supercomputer Center, University of California at San Diego,
9500 Gilman Dr., La Jolla, CA 92093, USA
%%%%%%%%%%%%%%%%%%%%%%%%%%%%%%%%%%%%%%%%%%%%%%%%%%%%%%%%%%%%%%%
}\\
\footnotemark[3] & \parbox[t]{13cm}{
%%%%%%%%%% Affiliation for footnotemark[3] %%%%%%%%%%%%%%%%%%%%
Institute for Infocomm Research, 21 Hen Mui Keng, Terrace, Singapore
119613
%%%%%%%%%%%%%%%%%%%%%%%%%%%%%%%%%%%%%%%%%%%%%%%%%%%%%%%%%%%%%%%
}\\
\end{tabular}
\end{small}
\bigskip
\end{center}

%%%%%%%%%%%%%%%%%%     Abstract   %%%%%%%%%%%%%%%%%%%%%%%%%%%%%
\begin{abstract}
This document is a LaTeX2$\varepsilon$ template file for
preparing a full paper.   Please read the instructions
carefully and prepare your manuscript.
\end{abstract}
%%%%%%%%%%%%%%%%%%%%%%%%%%%%%%%%%%%%%%%%%%%%%%%%%%%%%%%%%%%%%%%

\noindent {\bf Keywords:}
%%%%%%%%%%%%%%%%%%%%%%%%  Keywords %%%%%%%%%%%%%%%%%%%%%%%%%%%%
aberrant splicing, database, point mutation, scanning model
%%%%%%%%%%%%%%%%%%%%%%%%%%%%%%%%%%%%%%%%%%%%%%%%%%%%%%%%%%%%%%%



%%% BEGIN CONTENTS %%%
\section{Introduction}

The page limit for full paper is TEN.  In case the number
of pages still exceeds this limit after your efforts, please contact
giw2004@ims.u-tokyo.ac.jp for advice.

Do not change the text width, text height, baseline, font size, etc.,
specified in LaTeX2e/LaTeX/MS Word templates.

\section{Method and Results}

Of course you can create more sections with any section titles.

\subsection{Tables}

Table~\ref{table:sample} is an example of a table.

\begin{table}[h]
\caption{Bioinformatics conferences in 2003.}
\label{table:sample}
\begin{center}
\begin{tabular}{lll} \hline
Conference  & Date          & Site \\ \hline
RECOMB      & April 2-5     & Venice \\
ISMB        & August 6-10   & Fortaleza \\
GIW         & December 18-20& Yokohama \\ \hline
\end{tabular}
\end{center}
\end{table}

\subsection{Figures}

When you use figures (Fig.~\ref{figure:sample}, Fig.~\ref{figure:color})
with a special style file, please provide it.

Printed version of ``Genome Informatics Vol. 17, No. 2'' is
monochromatic and has a page limit. But, the electronic version (PDF,
HTML) does not have such restrictions. Therefore, we encourage the
author(s) to send any hypermedia files which can be attached to your
paper. The electronic version will be also published at JSBi Home Page
(\url{http://www.jsbi.org/jsbi_new/gioinfo.html}).

\begin{figure}
\begin{center}
\begin{tabular}{cc}
\begin{minipage}[t]{7cm}
\begin{center}
\includegraphics[scale=0.8]{sample.eps}
\end{center}
\caption{This is a sample figure. }
\label{figure:sample}
\end{minipage}
&
\begin{minipage}[t]{8cm}
\begin{center}
\includegraphics[width=7cm]{JSBi.eps}
\end{center}
\caption{This is a color sample figure. }
\label{figure:color}
\end{minipage}
\end{tabular}
\end{center}
\end{figure}


\subsection{Citation and References}

Reference style and citation should strictly follow this
template~\cite{akutsu98,doi99,motowani94,sw81,JSBi}.

\begin{enumerate}
\item References should be arranged in the alphabetical order of
authors and all references SHOULD be cited.
\item  Style (see examples in {\bf References} below).
\begin{enumerate}
\item Journal: author(s), title, journal name,
vol(issue):startpage-endpage, year.
\item Proceedigns: author(s), title, Proc. conference name, publisher
(if possible), startpage-endpage, year.
\item  Book: author(s), title, publisher, year.
\item  Other: author(s), title, any helpful information, year.
\item  URL: \url{http://(URL site)}.  URLs should be listed in the last.
\end{enumerate}
\item Provide all author names.  If the number of authors is
too large, you may reduce appropriately (such as
Venter, J.C., Adams, M.D., Myers, E.W., Li, P.W., {\em et al.},
The sequence of the human genome,
{\em Science}, 291(5507):1304--1351, 2001.)
\end{enumerate}

\section{Discussions}

Authors are recommended to use spell checker program and grammar check
program for improving your final version.

\begin{thebibliography}{99}
%%%%%%%% Proceedings %%%%%%%%%%%%%%%%%%
\bibitem{akutsu98}
Akutsu, T., Kuhara, S., Maruyama, O., and Miyano, S.,
Identification of gene regulatory networks by strategic gene
disruptions and gene overexpressions,
{\em Proc. 9th ACM-SIAM Symp. Discrete Algorithms}, 695--702, 1998.

%%%%%%%% Genome Informatics Series %%%%
\bibitem{doi99}
Doi, K. and Imai, H.,
A greedy algorithm for minimizing the number of primers in
multiple PCR experiments,
{\em Genome Informatics}, 10:73--82, 1999.

%%%%%%%% Book %%%%%%%%%%%%%%%%%%%%%%%%%
\bibitem{motowani94}
Motowani, R. and Raghavan, P.,
{\em Randomized Algorithms}, Cambridge University Press, 1994.

%%%%%%%% Journal article %%%%%%%%%%%%%%
\bibitem{sw81}
Smith, T.F. and Waterman, M.S.,
Identification of common molecular subsequences,
{\em J. Mol. Biol.}, 147(1):195--197, 1981.

%%%%%%%% URL %%%%%%%%%%%%%%
\bibitem{JSBi}
\url{http://www.jsbi.org/}

\end{thebibliography}
\end{document}
%%% END CONTENTS %%%




















