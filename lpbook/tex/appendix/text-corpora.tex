\chapter{Corpora}

In this appendix, we list the corpora that we use for examples in the
text.

\section{Canterbury Corpus}\label{section:corpora-canterbury}

The Canterbury corpus is a benchmark collection of mixed text and
binary files for evaluating text compression.  

\subsubsection{Contents}

The files contained in the archive are the following.

\begin{center}
\begin{tabular}{lllr}
\tblhead{File} & \tblhead{Type} & \tblhead{Description} & \tblhead{Bytes} 
\\ \hline
\code{alice29.txt} & text & English text & 152,089 \\
\code{asyoulik.txt} & play & Shakespeare & 125,179 \\
\code{cp.html}  & html & HTML source & 24,603 \\ 
\code{fields.c} & Csrc & C source &  11,150 \\
\code{grammar.lsp} & list & LISP source &  3,721 \\ 
\code{kennedy.xls} & Excl & Excel Spreadsheet &  1,029,744 \\
\code{lcet10.txt} & tech & Technical writing &  426,754 \\
\code{plrabn12.txt} & poem & Poetry &  481,861 \\
\code{ptt5} & fax & CCITT test set &  513,216 \\ 
\code{sum} & SPRC & SPARC Executable &  38,240 \\ 
\code{xargs.1} & man & GNU manual page &  4,227
\end{tabular}
\end{center}

\subsubsection{Authors}

Ross Arnold and Tim Bell.

\subsubsection{Licesning}

The texts are all public domain.

\subsubsection{Download}

\url{http://corpus.canterbury.ac.nz/descriptions/}

Warning: you should create a directory in which to unpack the
distribution because the tarball does not contain any directory
structure, just a bunch of flat files.



\section{20 Newsgroups}\label{section:corpora-20-newsgroups}

\subsubsection{Contents}

The corpus contains roughly 20,000 posts to 20 different newsgroups.
These are

\begin{center}
\footnotesize
\begin{tabular}{p{0.3\textwidth}p{0.3\textwidth}p{0.3\textwidth}}
\path{comp.graphics} & \path{rec.autos} & \path{sci.crypt}
\\
\path{comp.os.ms-windows.misc} & \path{rec.motorcycles} & \path{sci.electronics}
\\
\path{comp.sys.ibm.pc.hardware} & \path{rec.sport.baseball} & \path{sci.med}
\\
\path{comp.sys.mac.hardware} & \path{rec.sport.hockey} & \path{sci.space}
\\
\path{comp.windows.x}
\\[12pt]
\path{misc.forsale} & \path{talk.politics.misc} & \path{talk.religion.misc}
\\
& \path{talk.politics.guns} & \path{alt.atheism}
\\
& \path{talk.politics.mideast} & \path{soc.religion.christian}
\end{tabular}
\end{center}


\subsubsection{Authors}

It is currently being maintained by Jason Rennie, who speculates that
Ken Lang may be the original curator.

\subsubsection{Licensing}

There are no licensing terms listed and the data may be downloaded
directly.

\subsubsection{Download}

\url{http://people.csail.mit.edu/jrennie/20Newsgroups/}




\section{WormBase MEDLINE Citations}\label{section:corpora-wormbase}

The WormBase web site, \url{http://wormbase.org}, curates and distributes
resources related to the model organism {\it Caenorhabditis elegans},
the nematode worm.  Part of that distribution is a set of citation to
published research on {\it C.~elegans}.  These literature
distributions consist of MEDLINE citations in a simple line-oriented
format.

\subsubsection{Authors}

Wormbase is a collaboration among many cites with many supporters.
It is described in 
%
\begin{quote}
Harris, T.~W. et al. 2010. WormBase: A comprehensive resource for
nematode research. {\it Nucleic Acids Research} {\bf 38}.
\doi{10.1093/nar/gkp952}
\end{quote}

\subsubsection{Download}

\url{ftp://ftp.wormbase.org/pub/wormbase/misc/literature/2007-12-01-wormbase-literature.endnote.gz} (15MB)

