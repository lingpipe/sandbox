\chapter{Regular Expressions}\label{chapter:regex}

Generalized regular expressions, as implemented in Java and all the
scripting languages like Perl and Python, provide a general means for
describing spans of Unicode text.  Given a regular expression (regex)
and some text, basic regex functionality allows us to test whether a
string matches the regex or find non-overlapping substrings of a
string matching the regex.

\section{Matching Complete Strings}

Despite the rich functionality of regexes, Java's
\code{java.util.regex} package contains only two classes,
\code{Pattern} and \code{Matcher}.  An instance of \code{Pattern}
provides a stateless representation of a regular expression.  An
instance of \code{Matcher} represents the state of the matching
of a regular expression against a string.

It's easiest to start with an example of using a regular expression
for matching, which we wrote as a
\code{main()} method in the class \code{RegexMatch}, the work of
which is done by
%
\codeblock{RegexMatch.1}
%
First, we read the regex from the first command-line argument, then
the text from the second argument.  We then use the regular expression
to compile a pattern, using the static factory method
\code{pattern.compile()}.  This pattern is reusable.  We next
create a matcher instance, using the method \code{matcher()} on the
pattern we just created.  Finally, we assign a boolean variable
\code{matches} the value of calling the method \code{matches()} 
on the matcher.  And then we print out the result.

Regular expressions may consist of strings, in which case they
simply carry out exact string matching.  For example, the regex
\code{aab} does not match the string \stringmention{aabb}.  
There is an Ant target \code{regex-match} which feeds the command-line
arguments to our program.  For the example at hand, we have
%
\commandlinefollow{ant -Dregex="aab" -Dtext="aabb" regex-match}
\begin{verbatim}
Matches=false
\end{verbatim}
%
On the other hand, the regex \code{abc} does match
the string \stringmention{abc}.
%
\commandlinefollow{ant -Dregex="abc" -Dtext="abc" regex-match}
\begin{verbatim}
Matches=true
\end{verbatim}

\section{Finding Matching Substrings}

The second main application of regular expressions is to find
substrings of a string that match the regular expression.  The main
method in our class \code{RegexFind} illustrates this.  We read in two
command-line aguments into string variables \code{regex} and \code{text}
as we did for \code{RegexMatch} in the previous section.  We begin
by compling the pattern and creating a matcher for the text just
as in \code{RegexFind}.
%
\codeblock{RegexFind.1}
%
The first call of the \code{find()} method on a matcher returns
\code{true} if there is a substring of the text that matches the
pattern.  If \code{find()} returned \code{true}, then the are method
\code{group()} returns the substring that matched, and the methods
\code{start()} and \code{end()} return the span of the match, which is
from the start (inclusive) to the end (exclusive).

Subsequent calls to \code{find()} return \code{true} if there is a
match starting on or after the end position of the previous calls.
Thus the loop structure in the program above is the standard idiom
for enumerating all the matches.  

As an example, we use the Ant target \code{regex-find}, which takes
the same arguments as \code{regex-match}.
%
\commandlinefollow{ant -Dregex=aa -Dtext=aaaaab regex-find}
%
\begin{verbatim}
Found |aa| at (0,2)
Found |aa| at (2,4)
\end{verbatim}
%
As in other locations, we use vertical bars to denotes string
boundaries in program output.  The string \stringmention{aa} actually
shows up in four distinct locations in \stringmention{aaaaab}, at
spans (0,2), (1,3), (2,4), and (3,5).  Running find only returns two
of them.  The matcher works from the start to the end of the string,
returning the first match it finds after the first call to
\code{find()}.  In this case, that's the substring of
\stringmention{aaaaab} spanning (0,2).  The second call to
\code{find()} starts looking for a match at position 2, succeeding
with the span (2,4).  Next, it starts looking at position 4, but there
is no substring starting on or after position 4 that matches
\stringmention{aa}.

Matchers may be reused.  The method \code{reset()} may be called on a
matcher to reset it to behave as if it had just been constructed with
its given pattern and character sequence.  The character sequence may
be changed.  The patesequence may also be changed on the fly.  Although
matchers may be reused, the typical usage pattern is to construct them
for a text, use them once, then discard them.


\section{Regular Expression Syntax}

Regular expressions are constructed from basic regular expressions
using a few simple operations.  

\subsection{Empty Regex}

The simplest regular expression is the empty regular expression.  It
only matches the empty string.  For example, the expression
\code{Pattern.compile("").matcher("").matches()} would evaluate to
\code{true}.  Using our test code from the command-line proves tricky,
because we have to escape the quotes if we want to assign environment
variables, as in 
%
\commandlinefollow{ant -Dregex={\bk}"{\bk}" -Dtext={\bk}"{\bk}"  regex-match}
\begin{verbatim}
Regex=||    Text=||    Matches=true
\end{verbatim}
%
\commandlinefollow{ant -Dregex={\bk}"{\bk}" -Dtext=a regex-match}
\begin{verbatim}
Regex=||    Text=|a|    Matches=false
\end{verbatim}


\subsection{Characters}

A regular expression may consist of a single character.  A regular
expression consisting of a single character only matches that single
character.  For instance,
\code{Pattern.compile("a").matcher("a").matches()} would evaluate to
\code{true}.  In code, 
%
\commandlinefollow{ant -Dregex=b -Dtext=b regex-match}
\begin{verbatim}
Regex=|b|    Text=|b|    Matches=true
\end{verbatim}
%
\commandlinefollow{ant -Dregex=b -Dtext=c regex-match}
\begin{verbatim}
Regex=|b|    Text=|c|    Matches=false
\end{verbatim}

\subsubsection{Unicode Escapes}

Arbitrary unicode characters may be escaped in the same way as in
Java, using \code{{\bk}u}\codeVar{hhhh} for the character with Unicode
code point \unicode{\codeVar{hhhh}}.  This is confusing because
Unicode characters may also be included in Java programs using the
same syntax.  For instance, the expressions
\code{Pattern.compile("{\bk}u004E")} and \code{Pattern.compile("N")}
are the same, because Java replaces the Unicode escape
\code{{\bk}u004E} with \code{N} in the literal.  In order to compile
the regular expression \code{{\bk}u004E}, we need to escape the
backslash in Java, using \code{Pattern.compile("{\bk}{\bk}u004E")},
which is not the same as \code{Pattern.compile("N")}.

One important use of Unicode escapes is to match the period character,
which is a reserved character representing wildcards (see
\refsec{regex-wildcard}).  The period character is \unicode{002E}, 
so the regex \code{{\bk}u002E} matches it. 

%
\commandlinefollow{ant -Dregex="{\bk}u002E" -Dtext="." regex-match}
%
\begin{verbatim}
Regex=|\u002E|     Text=|.|     Matches=true
\end{verbatim}
%
Note that we did not have to escape the backslash because it is not a
Java literal nor is it in a context where the shell interprets it as
an escape.  The actual regex is printed as output.  In a Java program,
we'd still have to write \code{Pattern.compile("{\bk}{\bk}u002E")}.


\subsubsection{Other Escapes}

There are also other built-in escapes, such as \code{{\bk}n} for
newline and \code{{\bk}{\bk}} for a backslash character.  The expression
\code{Pattern.compile("{\bk}{\bk}")} will throw an exception, because
it is attempting to compile the regular expression \code{{\bk}}.  To
compile the regular expression \code{{\bk}{\bk}}, we need to write
\code{Pattern.compile("{\bk}{\bk}{\bk}{\bk}")}.  

This is all further confused by the command-line shell, which has its
own escape sequence.  The DOS The general principle is that you need to encode
the string at each level in such a way that the string you need at the
next level passes through.  For instance, consider this puzzler,
%
\commandlinefollow{ant -Dregex="\bk\bk\bk\bk" -Dtext="\bk\bk" regex-match}
%
\begin{verbatim}
Regex=|\\|    Text=|\|    Matches=true
\end{verbatim}
%
We had to escape the backslashes on the command-line with another
backslash because backspace-quote (\code{{\bk}"}) is an escape
sequence for the quote character.  Double backslashes are an escape
for backslash, which itself doesn't normally need an escape.  The
regex that actually gets passed to \code{compile()} is a pair of
backslash characters, \code{\bk\bk}.  The text that is matched is a
single backslash, \code{\bk}.  The match succeeds because the regex
\code{\bk\bk} is the escaped backslash character matcher.

\subsection{Concatenation}

Two regular expressions \codeVar{x} and \codeVar{y} may be
concatenated to produce a compound regular expression \codeVar{xy}.
The regex \codeVar{xy} matches any string that can be decomposed into
a string that matches \codeVar{x} followed by a string that matches
\codeVar{y}.

In the simplest case, we have a concatenation of characters, which
matches any string consisting of the exact same characters.  

\subsection{Begin and End of Line}


The caret (\code{\^{}}) is a regular expression that matches the end
of a line and the dollar sign (\code{\$}) a regular expression
matching the beginning of a line.  For example, 
%
\commandlinefollow{ant -Dregex="\^{}abc\$" -Dtext=abc regex-match}
\begin{verbatim}
Regex=|^abc$|    Text=|abc|    Matches=true
\end{verbatim}
%
\commandlinefollow{ant -Dregex=ab\$c -Dtext=abc regex-match}
\begin{verbatim}
Regex=|ab$c|    Text=|abc|    Matches=false
\end{verbatim}
%
The begin- and end-of-line regexes do not themselves consume any
characters.

These begin-of-line and end-of-line regexes do not match new lines
inside of strings unless the \code{MULTILINE} flag is set (see
\refsec{pattern-modes}).  By default, these regexes match new-line
sequences for any platform (Unix, Macintosh, or Windows).  If the
\code{UNIX\_LINES} flag is set, they only match Unix newlines (see
\refsec{pattern-modes}).


\subsection{WildCards}\label{section:regex-wildcard}

A singe period (\code{.}) is a regular expression that matches a
single characer.  For instance, the regex \code{.} matches the string
\stringmention{a}, and \code{a.c} matches \stringmention{abc}.%
%
\commandlinefollow{ant -Dregex=a.c -Dtext=abc regex-match}
\begin{verbatim}
Regex=|a.c|    Text=|abc|    Matches=true
\end{verbatim}
%

Whether or not a wildcard matches an end-of-line sequence depends on
whether or not the \code{DOTALL} flag is set (see
\refsec{pattern-modes}); if it is set, the wildcard matches
end-of-line sequences.  The following two expressions evaluate to
\code{true},
%
\begin{verbatim}
Pattern.compile(".").matcher("A").matches()
Pattern.compile(".",DOTALL).matcher("\n").matches()
\end{verbatim}
%
whereas the following expression evaluates to \code{false},
%
\begin{verbatim}
Pattern.compile(".").matcher("\n").matches()
\end{verbatim}
%




\subsection{Disjunction}

Two regular expressions \codeVar{x} and \codeVar{y} may be disjoined
to produce a compound regular expression
\codeVar{x}\code{|}\codeVar{y}.  The regex
\codeVar{x}\code{|}\codeVar{y} matches any string that matches either
\codeVar{x} or \codeVar{y} (or both).  For example,
%
\commandlinefollow{ant -Dregex="ab|cd" -Dtext=ab regex-match}
\begin{verbatim}
Regex=|ab|cd|
Text=|ab|
Matches=true
\end{verbatim}
%
\commandlinefollow{ant -Dregex="ab|cd" -Dtext=cd regex-match}
\begin{verbatim}
Regex=|ab|cd|    Text=|cd|    Matches=true
\end{verbatim}
%
\commandlinefollow{ant -Dregex="ab|cd" -Dtext=bc regex-match}
\begin{verbatim}
Regex=|ab|cd|    Text=|bc|    Matches=false
\end{verbatim}
%
The vertical bars used to display the regex are not part of it; thus
the regular expression is \code{ab|cd}, not \code{|ab|cd|}.  The regex
itself needs to be quoted on the command line so the vertical bar is
not interpreted as a Unix pipe.

Even though the order of disjunctions doesn't matter in a logical
expression.  And as far as matching whole strings, the order of
disjunctions doesn't matter.  But for finds, disjunctions are
evaluated left-to-right and the first match returned.
%
\commandlinefollow{ant -Dregex="ab|abc" -Dtext=abc regex-find}
\begin{verbatim}
Found |ab| at (0,2)
\end{verbatim}
%
\commandlinefollow{ant -Dregex="abc|ab" -Dtext=abc regex-find}
\begin{verbatim}
Found |abc| at (0,3)
\end{verbatim}
%

Because complex disjunctions are executed through backtracking,
disjunctions that have to explore many branches can quickly lead to
inefficient regular expressions.

Combinatorics can quickly spiral out of control with disjunctions.
Even a regex as short as \code{(a|b)(c|d)(e|f)} has $2^3 = 8$
possible paths.  

\subsection{Kleene Star and Other Quantifiers}

A regular expression \codeVar{x} may have the Kleene-star operator
applied to it to produce the regular expression \codeVar{x}\code{*}.
The regex \codeVar{x}\code{*} matches a string if the string is
composed of a sequence of zero or more matches of \codeVar{x}.

A regular expression may be followed by an asterisk (\code{*}),
pronounced Kleene-star (after Stephen Kleene, the mathematician who
devised regular languages).






\subsection{Parentheses and Groups}

In all cases, parentheses may be used around a regular expression to
clarify how the operations attach.




There is a rich syntax for constants denoting classes of char
constructing regular expressions out of
simpler regular expressions.

\subsection{Pattern Match Flags}\label{section:pattern-modes}

There are a set of flags, all represented as static integer constants
in \code{Pattern}, which may be supplied to the pattern compiler to
control the matching behavior of the compiled pattern.  The most
important of these for our purposes are the following.
%
\begin{center}
\begin{tabular}{lp{0.65\textwidth}}
\tblhead{Constant Flag} & \tblhead{Description} 
\\ \hline { } \\[-6pt]
\code{DOTALL}
& Allows the wild card (\code{.}) expression to match any
character, including line terminators.  If this mode is not
set, the wild card does not match line terminators.
\\[4pt]
\code{MULTLINE}
& Allows the begin-of-line (\code{\^{}}) and end-of-line (\code{\$}) to
match internal line terminators.  If this is not set, they only
match before the first and after the last character.  The full Java
notion of newline is used unless the flag \code{UNIX\_LINES} is also set.
\\[4pt]
\code{UNIX\_LINES}
& Treat newlines as in Unix, with a single \unicode{000A},
\unicodedesc{line feed}, Java string escape \code{{\bk}n},
representing a new line.  This affects the behavior of wildcard
(\code{.}) and begin-of-line (\code{\^{}}) and end-of-line 
(\code{\$}) expressions. 
\\[4pt] 
\code{CASE\_INSENSITIVE} 
& Matching is insensitive to case.  Case folding is only for
the ASCII charset (\unicode{0000} to \unicode{007F}) unless
the flag \code{UNICODE\_CASE} is also set, which uses the Unicode
definitions of case folding.
\\[4pt]
\code{UNICODE\_CASE}
& If this flag and the \code{CASE\_INSENSITIVE} flag is set,
matching ignores case distinctions as defined by Unicode
case folding rules.
\\[4pt]
\code{CANON\_EQ} 
& Matches are based on canonical equivalence in Unicode 
(see \refsec{unicode-normalization-forms}).
\end{tabular}
\end{center}

The flags are defined using a bit-mask pattern whereby multiple
flags are set by taking their bitwise or value.  For instance,
the expression \code{DOTALL~|~MULTILINE} allows the wildcard
to match anything and the line-terminator expressions to match
internal newlines.  More than two flags may be set this way,
so we could further restrict to Unix newlines with
\code{DOTALL~|~MULTILINE~|~UNIX\_LINES}.

Even stronger matching than canonical equivalence with unicode case
folding can be achieved by writinger the regular expression using
compatibility normal forms and using the ICU package to perform a
compatibility decomposition on the input (see
\refsec{unicode-normalization-forms}).


\subsection{Pattern Construction Exceptions}

Attempting to compile a regular expression with a syntax error raises
a runtime exception, \code{PatternSyntaxException} from the
\code{java.util.regex} package.  The parser tries to provide a helpful
warning message.  For instance, here's what we see if we inadvertently
add a right parenthesis to our regex.
%
\commandlinefollow{ant -Dregex=aa) -Dtext=aaab regex-find}
\begin{verbatim}
Exception in thread "main" 
    java.util.regex.PatternSyntaxException:  
    Unmatched closing ')' near index 1
\end{verbatim}







