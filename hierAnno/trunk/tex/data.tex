
\subsection{4-Test HIV Diagnosis}

Data from Alvord et al.~(1988) cited in (Qu et al.~1996) and Yang and
Becker (1997).  The four tests are (A) radioimmunoassay (RIA)
utilizing recombinant agl21, (B) RIA utilizing purified HIV p24, (C)
RIA utilizing purified HIV gpl20, and (D) enzyme-linked immunosorbent
assay.

\begin{tabular}{cccc|r}
{\it A}, {\it B}, {\it C}, {\it D} & {\it Count}
\\ \hline
0000 & 170
\\
0001 & 15
\\
0100 & 6
\\
1000 & 4
\\
1001 & 17
\\
1011 & 83
\\
1100 & 1
\\
1101 & 4
\\
1111 & 128
\\
\end{tabular}



\subsection{Uterine Carcinoma Diagnosis}

Holmquist et al. (1967) had seven pathologists independently classfy
118 hisotological slides.  Agresti and Lang (1993) dichotomized
ratings, with 1 representing carcinoma and 0 no carcinoma.

\begin{tabular}{cr}
{\it Pathologists} & {\it Count} \\
0000000 & 34 \\
0000100 & 2 \\
0100000 & 6 \\
0100001 & 1 \\
0100100 & 4 \\
0100101 & 5 \\
1000000 & 2 \\
1010101 & 1 \\
1100000 & 2 \\
1100001 & 1 \\
1100100 & 2 \\
1100101 & 7 \\
1100111 & 1 \\
1101001 & 1 \\
1101101 & 2 \\
1101111 & 3 \\
1110101 & 13 \\
1110111 & 5 \\
1111101 & 10 \\
1111111 & 16 \\
\end{tabular}



\subsection{{\it Strongyloides} Infection Diagnosis}

Joseph, Gyorkas and Coupal (1995) analyze data on 162 Cambodian
refugees arriving in Montreal, Canada between July 1982 and February
1983.  A serology test and stool examination were used.  The parasites
are visible in a stool sample positive result, so false positives are
rare.  It is also known that stool examination is highly dependent
on level of infection.

\begin{tabular}{lr}
{\it Serology}, {\it Stool} & {\it Count} \\
11 & 38 \\
10 & 87 \\
01 & 2 \\
00 & 35 \\
\end{tabular}

\subsection{{\it Leishmania infantum} Infection Diagnosis}

\subsubsection{1999 Study}

Boelaert et al.~(1999) report a study of 151 street dogs at risk for
{\it Leishmania infantum} infection in Tunisia by the Institute Pasteur de
Tunis and the Prince Leopolod Institute of Tropical Medicine in
Antwerp.  Serological tests include an indirect immunofluorescence
antibody test (IFAT), an enzyme-linked immunosorbent assay (ELISA),
and a direct agglutination test (DAT).  Clinical status, and parasitological
data is also included as annotations.

\begin{tabular}{lllllr}
{\it C}, {\it E}, {\it D}, {\it I}, {\it P} & {\it Count} \\
00000 & 105 \\
00100 & 8 \\
01000 & 8 \\
01011 & 1 \\
01100 & 1 \\
01110 & 7 \\
01111 & 3 \\
10000 & 9 \\
10111 & 2 \\
11110 & 2 \\
11111 & 5 \\
\end{tabular}

The tests are clinical, ELISA, DAT, IFAT, and parasitology,
respectively.  A priori, parasitology has high specificity and low
sensitivity.

Their statistical analysis follows Qu, Tan and Kutner, assuming
independence of tests and differing sensitivity/specificity for the
models.  That is, two latent classes, for infected and not infected.
They also considered three latent classes, with symptomatically and
asymptomatically infected classes subdividing the infected class.
A direct effect between two tests was the best fit.


\subsubsection{2008 Study}

Menten et al. (2008) performed a follow-on study using DAT, rk39, and
KAtex tests, along with a parasitology test for 291 dogs.

\begin{tabular}{lr}
{\it D}, {\it r}, {\it K}, {\it P} & {\it Count} \\
1111 & 51 \\
1110 & 1 \\
1101 & 4 \\
1100 & 16 \\
1011 & 15 \\
1010 & 1 \\
1001 & 2 \\
1000 & 5 \\
0111 & 7 \\
0110 & 4 \\
0101 & 1 \\
0100 & 15 \\
0011 & 1 \\
0010 & 1 \\
0001 & 1 \\
0000 & 166
\end{tabular}

\subsection{Thought Disorder Diagnosis}

Broemeling (2001) studied four psychiatric residents diagnosing twelve
patients for an unspecified ``thought disorder''.

\begin{tabular}{lr}
{\it Diagnoses} & {\it Count} \\
1111 & 2 \\
1101 & 1 \\
0111 & 2 \\
1100 & 1 \\
0101 & 1 \\
0010 & 2 \\
0100 & 1 \\
0000 & 2 \\
\end{tabular}

Used a Dirichlet prior on all $2^4$ possible outcomes with
a count of 1 for each outcome.
