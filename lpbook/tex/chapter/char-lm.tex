\chapter{Character Language Models}\label{chap:char-lm}

A \techdef{language model} is a statistical construct that assigns
probabilities to strings of symbols.  Traditionally, these strings are
either sequences of bytes, sequences of characters, or sequences of
tokens.  LingPipe provides language-model implementations for
sequences of characters and sequences of tokens (see
\refchap{tokenization} for more on tokenization in LingPipe).  In this
chapter, we introduce the basic interfaces and character language
models.  In \refchap{token-lm}, we introduce token language models.

The bytes, characters or tokens modeled by language models may
represent arbitrary sequences of symbols.  With token-based models,
the symbol set need not even be finite. For instance, we can model XML
documents as sequences of characters, proteins as sequences of amino
acids, or texts as sequences of tokens.  In this chapter, we are going
to focus on texts represented as sequences of Unicode characters.  the
text itself may come from anywhere, including e-mail or instant
messages, blogs, newspapers, fiction novels, emergency room discharge
summaries, people names or names of companies, etc.

The basic operation of language models in LingPipe is to provide them
training data consisting of a sequence of texts.  After they are
trained, language models may be used to estimate the probability of
texts that were not in the training data.  

\section{Applications of Language Models}\label{section:char-lm-apps}

LingPipe uses character language models as the basis of
implementations of several interfaces, including classifiers, taggers,
chunkers, and spelling correctors.  There are further applications for
token language models discussed in \refchap{token-lm}.  In all cases,
these applications use one or more language models to estimate
different kinds of text.  For instance, in language-model based
classification, there is a language model for each category, and texts
are classified into the category which assigns them the highest
probability.  In chunkers, separate language models might be
constructed for person names, company names, place names, and for text
that's not part of a name; these are then combined to analyze new text
for names of entities.  For spelling correction, a single language
model characterizes the correct spellings and is used to weigh
alternative corrections against each other.

Yet another application of language models is in characterizing
languages or sublanguages.  For instance, we can compare the entropy
of news stories in the {\it New York Times} with biomedical abstracts
in MEDLINE, the surprising result being that MEDLINE is more
predictable at the text level than the {\it Times}).  We can also
compare the entropy of Chinese and English, measuring the average
information content of a Chinese character (of which there are tens of
thousands) with the information content of an English character (of
which there are dozens).

\subsection{Applications Beyond LingPipe}

There are many uses of language models that are not supported
by LingPipe.  Yet the language modeling components are the same
and LingPipe supports appropriate interfaces.  

One of the major applications of language models is in speech
recognition, where they play the same role as the language model in
spelling correction.

Another application of language models is compression.  Using a
technique such as \techdef{arithmetic coding}, language models provide
state-of-the-art compression.  Although LingPipe does not implement an
arithmetic coder for compression, we will discuss the connection to
compression when we discuss language model evaluation.  It turns
out there is a deep connection between predictability expressed
probabilistically and compressibility.


\section{The Basics of $N$-Gram Language Models}

LingPipe's character- and token-based language models are based on
\techdef{$n$-grams}.  An $n$-gram is nothing more than a sequence of
$n$ symbols.  In this chapter, we consider character-based $n$-grams;
in \refchap{token-lm}, we consider tokens as symbols.  For instance,
the triple (\charmention{a}, \charmention{b}, \charmention{c}) is a
character trigram (3-gram), whereas the pair (\stringmention{in},
\stringmention{Brooklyn}) is a token bigram (2-gram).

The basis of an $n$-gram language model is storing counts for
$n$-grams seen in a training corpus of text.  These are then used to
infer the probability of a new sequence of symbols.  They key
technical feature of $n$-gram models is that they model the
probability of a character in a sequence based only on the previous
$n-1$ characters.  Models such as these with finite memories are
called \techdef{Markov chains}.

The chain rule allows us to model the probability of a sequence of
characters based on a model of a single character following a sequence
of characters before it.  For instance, with the character sequence
\stringmention{tooth} using 3-grams (trigrams), we decompose the
total probability of a sequence into the product of a sequence of
character probabilities, with
%
\begin{equation}
p(\stringmention{tooth})
= p(\stringmention{t})
\times
p(\stringmention{o}|\stringmention{t})
\times
p(\stringmention{o}|\stringmention{to})
\times
p(\stringmention{t}|\stringmention{oo})
\times
p(\stringmention{h}|\stringmention{ot}).
\end{equation}
%
To prevent the possibilty of underflow from multiplying long sequences
of numbers less than 1, we almost always work on the log scale.  This
also has the pleasant property of converting multiplication to
addition, yielding
%
\begin{equation}
\log p(\stringmention{tooth})
= \log p(\stringmention{t})
+
\log p(\stringmention{o}|\stringmention{t})
+ 
\log p(\stringmention{o}|\stringmention{to})
+
\log p(\stringmention{t}|\stringmention{oo})
+ 
\log p(\stringmention{h}|\stringmention{ot}).\label{eq:char-lm-log-scale}
\end{equation}


\section{Character-Level Language Models and Unicode}

LingPipe's character-based language models are models of sequences of
Java \code{char} values, which represent UTF-16 byte pairs (see
\refsec{char-primitive}).  When LingPipe's language models were coded,
every Unicode code point could be coded in a single UTF-16 byte pair.
As we discussed, this is enough to cover Unicode's basic multilingual
plane (BMP), which includes most characters for most languages.%
%
\footnote{As discussed in \refsec{char-primitive}, Unicode 4.1,
  introduced characters with code points beyond the BMP.  To code such
  characters requires two 16-bit \code{char} values, known as a
  surrogate pair.  For the purposes of LingPipe's language models,
  characters are 16 bit UTF-16 values and a code point beyond the BMP
  is considered two characters.  What this will mean in practice is
  that LingPipe's language models will assign non-zero probability to
  sequences of \code{char} values that do not correspond to legal
  UTF-16 encodings, and thus don't correspond to Unicode strings.  In
  the case of spelling correction, we have to be careful not to
  suggest corrections that are illegal sequences of \code{char}
  values.  In other cases, we just lose some modeling efficiency (in
  the statistical sense of efficiency, not the run-time sense).}


\section{Language Model Interfaces}\label{section:char-lm-interfaces}

There is a rich set of interfaces for language models and related
support classes in the LingPipe package \code{com.aliasi.lm}.  The
basic interface is \code{com.aliasi.lm.LanguageModel} interface, and
other interfaces are nested within \code{LanguageModel}.

\subsection{Predictive: \code{LanguageModel}}

The top-level language model interface, \code{LanguageModel}, defines
methods for estimating probabilities of sequence of characters, which
is also known as prediction.  It defines two methods for estimating
the log (base 2) probability of a sequence of characters, one based on
character array slices and one based on the character sequence
interface.
%
\begin{verbatim}
double log2Estimate(char[] cs, int start, int end);
double log2Estimate(CharSequence cs);
\end{verbatim}
%
The first method uses an indexed character array to define the slice
of characters from \code{cs[start]} to \code{cs[end-1]}.  For most
methods, the character slice method is more efficient and the
character-sequence method is implemented by first converting the
sequence to an array of characters.

The reason estimates are returned as logarithms is that we would
otherwise run into underflow.  If characters have an average
probabilty of 0.25 (this is roughly the case for open-domain English
text), the probability of a sequence of 600 tokens has a $0.25^{600} =
2^{-1200}$ probabilty, which is too small to fit into a Java
double-precision floating point value (the minimum value for which is
$2^{-1074}$; see \refsec{java-numbers}).

\subsection{Trainable: \code{LanguageModel.Dynamic}}

Language models that may be trained by supplying example text
implement the interface \code{LanguageModel.Dynamic}.  This interface
extends the corpus interface \code{ObjectHandler<CharSequence>},
which specifies the method
%
\begin{verbatim}
void handle(CharSequence cs);
\end{verbatim}
%
The character sequence is considered as training data and used to fit
the model parameters.  See \refsec{corpus-handlers} for more
information on \code{ObjectHandler} and how it fits into corpus-based
usage patterns.

There are four other methods named \code{train()}, the first two
of which simply allow training with character sequences.
%
\begin{verbatim}
void train(CharSequence cs);
void train(CharSequence cs, int count);
\end{verbatim}
%
The first method is a legacy method duplicating the functionality of
\code{handle()}.  The second method takes an integer count value which
determines the number of times the character sequence is used for
training.  Although implemented more efficiently, a training call
with a count is equivalent to calling the basic training method the
count number of times.

The last two methods just repeat the character sequence methods
with character slices.


\subsection{Conditional Probabilities: \code{LanguageModel.Conditional}}

Conditional language models predict the probability of a character
given the previous characters.  The interface
\code{LanguageModel.Conditional} specifies the method
%
\begin{verbatim}
double log2ConditionalEstimate(CharSequence);
\end{verbatim}
%
This method returns the probabilty of the last character in the
sequence given the initial characters in the sequence.  For instance,
\code{log2ConditionalEstimate("abcd")} returns the probability of the
character \charmention{d} following the sequence of characters
\stringmention{abc}.  

There is also a method working on character slices.  It doesn't make
sense to call this method with an empty sequence.

The character language models all support this method, but it is not
supported by the token-based language models, which work token by
token, not character by character.

\subsection{Sequence versus Process Language Models}

Language models are divided into two types, sequence models and
process models.  Conceptually, sequence models distribute probability
to all strings of all lengths.  As such, they may be used to compare
the probabilities of words of different lengths.  Process models
form the basis of LingPipe's taggers and chunkers.

In contrast, process models distribute probability only among strings
of a given length.  Process models may only be used to compare the
probabilities of two strings of the same length.  LingPipe implements
sequence models on top of process models as described below.  Process
models also form the basis of spelling correction and like sequence
models, may be used as the basis of classifiers.

The distinction between sequence and process models is implemented
using \techdef{marker interfaces}.  A marker interface does not
specify any method signatures.  The most well-known marker interface
in Java itself is \code{java.io.Serializable}, though
\code{java.util.EventListener} is also a marker interface.


\subsubsection{Bounded: \code{LanguageModel.Sequence}}

The interface \code{LanguageModel.Sequence} is just a marker
interface, meaning that it does not specify any new methods other than
that inherited from its super-interface, \code{LanguageModel}.

Sequence language-model implementations are normalized so that the sum
of the probability of all strings of all lengths is one.%
%
\footnote{Because it's just a marker interface, the requirement is
  conventional rather than expressed in Java's type language.  This is
  similar to interface requirements on collections in the
  standard \code{java.util} library.}
%
In symbols, this amounts to
%
\begin{equation}
\sum_{n \geq 0} \hspace*{8pt} \sum_{\text{\code{cs.length()} == n}} \hspace*{3pt} 2^{\text{\code{log2Prob(cs)}}} = 1.
\end{equation}

\subsubsection{Unbounded: \code{LanguageModel.Process}}

The second type of language model is a process language model.  Such a
language model implements the marker interface
\code{LanguageModel.Process}.  A process language model treats a
sequence of characters as being generated as a kind of random process
that generates a character at a time without a distinct beginning or
end.

Because they don't model the beginning or end of a string, process
language models are normalized by length so that the sum of the
probabilities of all strings of a given length is one.  In symbols,
for every \code{n}, we have
%
\begin{equation}
\sum_{\text{\code{cs.length()} == n}} \hspace*{3pt} 2^{\text{\code{log2Prob(cs)}}} = 1.
\end{equation}
%


\section{Process Character Language Models}

As we saw in the last section, language models have a particularly
simple interface.  We just give them text to train and then they
can model new text. 

The simplest language model implementation supplied by LingPipe is
\code{NGramProcessLM}, in package \code{com.aliasi.lm}.  This class
implements several interfaces, including several of the basic language
model interfaces, \code{LanguageModel}, the unbounded interface,
\code{LanguageModel.Process}, the trainable interface
\code{LanguageModel.Dynamic}, and conditional prediction interface
\code{LanguageModel.Conditional} (see \refsec{char-lm-interfaces} for
descriptions of these interfaces).

In addition to the LM interfaces, the \code{NGramProcessLM} class also
implements \code{ObjectHandler<CharSequence>} (see
\refsec{corpus-handlers} for a description of the handler interface).
This allows language models to be integrated into the general
parser/handler/corpus interfaces (see \refchap{corpus} for more
details).

Furthermore, process character LMs are both serializable (see
\refsec{io-object-data-io}) and compilable (see
\refsec{io-compilable}).  As we will see later in this section,
compilation produces a very efficient, but no longer dynamic
version of the language model, whereas serialization will allow
it to be written out and read back in.


\subsection{Process Character Language Model Demo}\label{section:char-lm-process-demo}

Perhaps the simplest way to get a feel for how language models work is
to code one up.  We provide an example in the class
\code{ProcessLmDemo} in this chapter's package,
\code{com.lingpipe.book.charlm}.  

\subsubsection{Code Walkthrough}

As with all of our demos, it's set up as a simple \code{main()} method
that reads its arguments from the command line.  Specifically, we need
an integer n-gram length (\code{ngram}), a character sequence on which
to train (\code{textTrain}) and a character sequence on which to test
(\code{textTest}), which are supplied as the first three arguments.
The training and evaluation are then almost trivial.
%
\codeblock{ProcessLmDemo.1}
%
We start by creating a new instance of an \code{NGramProcessLM}
supplying the n-gram length to the constructor.  We then take the
training text string and pass it to the language model's
\code{handle(CharSequence)} method, which adds the text to the
training data.  Finally, we compute the log (base 2) of the
probaiblity of the test text using the language model's
\code{log2Estimate(CharSequence)} method.  

\subsubsection{Running the Demo}

The Ant target \code{process-demo} runs the command, using
the property \code{ngram} for $n$-gram length, the property
\code{text.train} for the training text, and the property
\code{text.test} for the test text.  It is thus run as
follows.
%
\commandlinefollow{ant -Dngram=5 -Dtext.train="abracadabra" -Dtext.test="candelabra" process-demo}
\begin{verbatim}
ngram=5    train=|abracadabra|    test=|candelabra|
log2 p(test|train)=-69.693
\end{verbatim}
%
The command-line arguments are first printed (with vertical bars to
denote the boundaries of the strings) and then the log probability of
the test sequence given the training data is shown.

One way of thinking of language models is as measuring a kind
of text similarity.  The more similar the test text is to the
training text, the higher its probability.

If the test text is very predictable given the training text, the log2
probability will be higher (higher log probabilities have smaller
absolute values, because logarithms of values in $[0,1]$ are always
negative).  For instance, consider the following case.
%
\commandlinefollow{ant -Dngram=2 -Dtext.train="ababababab" -Dtext.test="ababab" process-demo}
\begin{verbatim}
ngram=2    train=|ababababab|    test=|ababab|
log2 p(test|train)=-3.060
\end{verbatim}

\subsection{Length Effects}

Shorter texts will usually have lower probabilities in a process
language model.  That's because the sum of probabilities of all
strings of a fixed length sum to 1, such as all seven-character
strings.  There are many many more seven-character strings than
six-character strings.  The number of strings of a given length grows
exponentially (as long as the alphabet contains more than a single
character).  It's important to keep in mind that we shouldn't be
comparing probabilities of strings of different lengths using a
process language model.  Conveniently, LingPipe's built-in
applications of language models ensure that the probabilities are used
appropriately.

As an example of the length effect, consider the same training text as
above with a test text that's a subsequence of the test text above:
%
\commandlinefollow{ant -Dngram=2 -Dtext.train="ababababab" -Dtext.test="abab" process-demo}
\begin{verbatim}
ngram=2    train=|ababababab|     test=|abab|
log2 p(test|train)=-2.419
\end{verbatim}
%
Note that -2.419 is greater than -3.060.  In linear probability terms,
that's $2^{-2.419} = 0.225$, whereas $2^{-3.060} = 0.120$.  In other
words, the model, after training, says that \stringmention{abab} is
almost twice as likely as \stringmention{ababab}.

For process language models, which don't model boundaries, prefixes of
a string always have higher probabilities.  This is not necessarily
the case for arbitrary substrings, nor will it be the case for the
boundary language models we consider in the next section.

\subsubsection{Characters not in the Training Data}

A language model will assign probabilities to any texts of any length.
The characters don't need to have been seen in the training data.  For
instance, consider our first example above, where the characters
\charmention{e}, \charmention{n}, and \charmention{l} were new to the
training data.  Characters that were not seen in the training data tend
to have fairly low probabilities.  For instance, consider
%
\commandlinefollow{ant -Dngram=2 -Dtext.train="ababababab" -Dtext.test="xyxy" process-demo}
\begin{verbatim}
ngram=2    train=|ababababab|    test=|xyxy|
log2 p(test|train)=-71.229
\end{verbatim}
%
Again, it's clear why we use the log scale --- $2^{-71}$ is a very low
probability indeed.

\subsubsection{Effect of $n$-Gram Length}

The $n$-gram length is the most important of the tuning parameters; we
consider the rest of which we turn to in the next section.  A longer
$n$-gram will provide a tighter model of the training text than
shorter $n$-grams.  The longer the $n$-gram, the higher the
probability assigned to the training text itself as well as to very
similar text (again measuring by $n$-gram frequency).  For instance,
consider

\commandlinefollow{ant -Dngram=2 -Dtext.train="abcdef" -Dtext.test="abcdef" process-demo}
\begin{verbatim}
ngram=2    train=|abcdef|    test=|abcdef|
log2 p(test|train)=-11.334
\end{verbatim}

and

\commandlinefollow{ant -Dngram=3 -Dtext.train="abcdef" -Dtext.test="abcdef" process-demo}
\begin{verbatim}
ngram=3    train=|abcdef|    test=|abcdef|
log2 p(test|train)=-10.884
\end{verbatim}

The improvement is relatively small.  If we had repeated the training
text many times or if it had been longer, the effect would be
stronger.  Here, if you actually go up to 4-grams, the probability
estimate gets worse.  The reason for this is the default smoothing
parameters, which we discuss in \refsec{char-lm-smoothing}.

One thing to beware of is that increasing the $n$-gram size tends to
have a dramatic impact on the amount of memory required.  This is
apparent when you consider that for basic English, there are only a
few dozen characters and thus only a few dozen unigrams (1-grams).
But there are thousands of bigrams (2-grams), and millions of 5-grams
and 6-grams.  Even with 40 characters, there are over 2.5 million
potential 4-grams.  Luckily, natural language text is fairly
predictable, so the set of $n$-grams actually observed tends to grow
more slowly than this worst case may lead you to believe.
Furthermore, the number of $n$-grams in any given text is always
bounded by the length of the text, because each position contributes
at most one $n$-gram of a given length.


\section{Sequence Character Language Models}

In this section, we describe the second kind of character language
model available in LingPipe, the sequence models.  Sequence models are
applied when there are either boundary or length effects.  They
provide a properly normalized probability model to compare the
probabilities of strings of different lengths; the process models only
allow comparison of probabilities of strings of the same length.  The
sequence models in LingPipe are built on top of process language
models and use exactly the same interfaces for training and evaluation.

Sequence character language models are the basis for generating words
in LingPipe's hidden Markov model taggers and chunkers.  They are also
used for smoothing token generation in token-based language models.  
Sequence character language models are also used to model
sequences for spell checking.  Like process language models, sequence
models may also be used as the basis of classifiers.  \

The right language model to use depends on the data.  If there are
boundary effects and the sequences are relatively short, such as words
or sentences as opposed to full documents, use sequence language
models.  If you need to compare strings of different lengths, use
sequence language models.  Otherwise, use process models.

\subsection{Boundary and Length Effects}

The process character language models, which we saw in the last
section, treat the stream of characters as an ongoing process with no
start and no end.  One consequence of the process model is that the
initial and final characters in a language model are not treated any
differently than those in the middle. 

Suppose we wanted to build a language model for words, or for
sentences.  Note that a subsequence of a word is not necessarily a
word; for instance, \stringmention{saril}, which is a substring of
\stringmention{necessarily}, is not an English word.  Being like a
word is graded, though, with \stringmention{saril} being more word
like than the substring \stringmention{bseq} of
\stringmention{subsequence}.

Although it seems obvious, it is important statistically that each
word or sentence is of finite length (though there is, in principle,
no bound on this length other than the ability of our interlocuters to
understand us and the amount of time we have).  Words in most
languages are not uniform from start to end.  For instance, in
English, there are suffixes like \stringmention{ing} and
\stringmention{s} that appear far more often at the ends of words than
at the beginning of words.  

Another effect arises at the syllable level in English from
pronounciation.  Syllables tend to follow the \techdef{sonority
  hierarchy}, meaning that sonority tends to rise toward the middle of
the syllable and fall at the end.  For instance, vowels like
\stringmention{o} are the most sonorous, then nasals like
\stringmention{n} and \stringmention{m}, then consonants like
\charmention{r}, then stops like \charmention{t}.  For instance, the
one syllable sequences \stringmention{tram} and \stringmention{mart}
are words, and both have sonorities that increase toward the voewl and
then fall.  In contrast, \stringmention{rtam} is not pronounceable in
English, with the usual reason given that it violates the syllable
sonority principle, because \charmention{r} is more sonorous than
\charmention{t}.  Different langauges have different tolerances for
syllabel sonority.  On the one hand, Japanese typically only allows
open syllables, which means no final consononants other than the most
sonorant, the nasals like \charmention{n}).  At the other extreme,
languages like Slovenian don't even require a full vowel in a syllable
and stack up consonants between syllables in a way that is hard for
speakers of languages like Japanese or even English to pronounce.

In English sentences, we tend to find the first word capitalized more
often than would be expected if the generation of characters was
completely uniform.  At the other end, sentence terminating
punctuation characters, like the period and question mark, show up
more frequently at the end of sentences than elsewhere.  Perhaps more
importantly for statistical modeling follows from this, namely that
non-punctuation characters tend not to show up at the ends of
sentences in well-edited text.

\subsection{Coding Sequences as Processes}

A standard trick used in both computer languages and mathematical
models for converting a process-type model into a bounded
sequence-type model is to add distinguished end-of-string characters.
For instance, this is the technique used by the C programming language
to encode strings.  The byte zero (\code{0x00}) is used to encode the
end of a string.  Thus (ASCII) strings in C are stored in arrays one
byte longer than the string itself.  Similarly, lines in many
file-oriented formats are delineated by end-of-line characters, such
as the carriage return (\unicode{000D}).  In order to present multiple
sequences, characters such as the comma are used in comma-separated
values (CSV) file formats used for spreadsheets.  These commas are not
part of the strings themselves, but act as separators.%
%
\footnote{In standard comma-separated value files, strings that
  contain commas may be quoted.  Then, to include quotes, an escape
  mechanism is used, typically using a sequence of two double quotes
  within a quoted string to represent a quote.  Getting the details of
  these encodings right is essential for preserving the true values of
  strings.}

To encode sequence models as processes, we add a distinguished
end-of-string character.  It is treated like another character in that
at each point, there is a probability of generating the end-of-string
character given a sequence of previous characters.  After genearing
the end-of-string character, the sequence is terminated and no more
characters may be generated.  This leads to a bound on the otherwise
ongoing character generation process.  If the probabilty of generating
the next character includes the possiblity of generating the
end-of-string character, then the result is a bounded model that's
properly normalized over all strings.  That is, the sum of the
probabilities of all strings sums to 1, so we have a proper model of
strings.  The process model generates sequences in which the sum of
the probabilities of strings of a given length was 1.  The upshot of
this is that there's no way to use the process language models to
compare string probabilities to each other when the strings are
different lengths.

The use of a distinguished end-of-string character also handles
boundary effects.  Because we have to generate the end-of-string
character given the suffix of the word, we can model the fact that
strings are more likely to end following some character sequences than
others.  For instance, if we use \charmention{\$} as our end-of-string
marker (fllowing regular expression notation), then we might find
\stringmention{runs\$} to be more likely than \stringmention{ru\$}
because of the probability of generating \charmention{\$} given
\stringmention{runs} can be much higher than the probabilty of
generating \charmention{\$} given \stringmention{ru\$};\ in symbols, we
can have%
%
\footnote{Although not precise in most contexts, the notation $x >> y$
  indicates that $x$ is much greater than $y$ on some implicit scale.
  Typically this notation is only used when $x$ is at least one or
  more orders of magnitude (factors of ten) greater than $y$.}
%
\begin{equation}
p(\charmention{\$}|\stringmention{runs}) >> p(\charmention{\$}|\stringmention{ru}).
\end{equation}

We also use a distinguished begin-of-string character which
we use to condition the probabilities for the string's prefix.
Suppose we use \charmention{{\^{}}} as our begin-of-string character.
What we do is reduce the probability of generating a string
like \stringmention{runs} to that of generating 
\stringmention{runs\$} given \charmention{\^{}}, thus setting
%
\begin{equation}
p_{\mbox{\footnotesize seq}}(\stringmention{runs})
= p_{\mbox{\footnotesize proc}}(\stringmention{runs\$}|\stringmention{\^{}}).
\end{equation}
%
This defines the probability of \stringmention{runs} in the
sequence model, $p_{\mbox{\footnotesize seq}}()$, in terms of
a conditional probability calculation in a process model,
$p_{\mbox{\footnotesize proc}}()$.

\subsection{Sequence Character Language Model Demo}

The demo code is in the class file \path{SequenceLmDemo} in this
chapter's package, \code{com.lingpipe.book.charlm}.  The code
is almost identical to the process LM demo introduced in
\refsec{char-lm-process-demo}.

\subsubsection{Code Walkthrough}

There are three command line arguments, populating the \code{ngram}
integer, \code{csvTrain} as a string and \code{textTest} as a string.

For the sequence language model demo, the training data is assumed to
be a sequence of texts separated by commas (with no allowance for
commas in the strings for this demo).  The work is done in
constructing a trainable instance of the boundary language model,
training on each text, then estimating the probability of the test
text.
%
\codeblock{SequenceLmDemo.1}
%
The method \code{split(String)} defined in \code{java.lang.String}
splits the string into substrings at positions matching the string
separator argument.  Note that the separators are removed from the
strings.  It would be more efficient, given that we are splitting
on a character, to do this explicitly and use the \code{train()}
method taking an array slice; that way, there is no copying at all.

As usual, we have not shown the print statements or the boilerplate
for the \code{main()} method or arguments.

\subsubsection{Running the Demo}

The Ant target \code{seq-demo} runs the demo, with arguments
specified as properties \code{ngram} for $n$-gram length, 
\code{csv.train} for the comma-separated training texts,
and \code{text.test} for the test text.  For example,
%
\commandlinefollow{ant -Dngram=4 -Dcsv.train="runs,jumps,eating,sleeping" -Dtext.test="jumps" seq-demo}
\begin{verbatim}
ngram=4   train=|runs|jumps|eating|sleeping|   test=|jumps|
log2 p(test|train)=-9.877
\end{verbatim}
%
Thus the log (base 2) probability of seeing the string
\stringmention{jumps} in the model trained on the four specified words
is roughly -10, which translates to $2^{-10}$, or about 1/1000 on the
linear scale.  The reason the probability is so low is due to the
smoothing in the model (see \refsec{char-lm-smoothing}).

In contrast to the process model, in the sequence model, prefixes
are not always more probable than longer strings.  For
instance, consider \charmention{jump} instead of \code{jumps}.
%
\commandlinefollow{ant -Dngram=4 -Dcsv.train="runs,jumps,eating,sleeping" -Dtext.test="jump" seq-demo}
\begin{verbatim}
ngram=4   train=|runs|jumps|eating|sleeping|   test=|jump|
log2 p(test|train)=-13.037
\end{verbatim}
%
Under this model (4-grams with default parameters) with the specified
four training instances, the probability of \stringmention{jumps} is
about 8 times as likely as the probability of \stringmention{jump}
($2^{-13}$ is approximately 1/8000).

One of the most useful parts of the sequence model is how it
generalizes endings and beginnings.  For instance,
\stringmention{running} is estimated to be fairly probable (log
probabilty $-21.5$).  Even an unseen word such as
\stringmention{blasting} (log probability $-46.4$) is modeled as more
likely than substrings like \stringmention{blast} (log probability
$-46.6$), because the model has learned that words are likely to end in
\stringmention{ing}. 


\subsubsection{Configuring the Boundary Character}

The boundary character in the bounded language models may be specified
by the client.  Because the begin-of-sentence character is not
generated, the same character may be used for both boundaries without
loss of generality.  The default for the boundary character is the
reserved Unicode character \unicode{FFFF}.  The boundary character
must be set in the constructor, as it remains immutable during
training and/or compilation.


\section{Tuning Language Model Smoothing}\label{section:char-lm-smoothing}

Language models provide two parameters for tuning in addition to the
length of $n$-grams.  One of these controls the degree to which
lower-order $n$-grams are blended with higher-order $n$-grams for
smoothing.  The second controls the number of characters in the base
(zero order) model.  We describe both in this section.

Operationally, the tuning parameters may be set at any time before
compilation.  These will dynamically change the behavior of the
dynamic language models.  But once a model is compiled, the tuning
parameters may not be changed.  This suggests a strategy of training
a large model, then evaluating the tuning parameters on held out data
(not part of training) to choose which is best.  

\subsection{The Problem with Maximum Likelihood}

Recall that maximum likelihood estimation provides estimates for model
parameters in such a way as to maximize the probability of the
training character sequence(s).  These estimates, for our simple
language models, would be based on ratios of empirical counts.  In
particular, they will provide zero probabilty to a character given a
context if that character was never seen in that context in the
training data.  The problem with this is that we are likely to run
into subsequences of characters in new test data that were not in the
training data.  We would like to assign these unseen sequences
non-zero probabilities so that every documents we observe in testing are
assigned non-zero probabilities.  If we don't, our higher-level
applications of language models to tagging or spell checking or
classification would almost certainly break down.

For example, suppose we're trying to model the probability of the
letter \charmention{r} following the sequence \stringmention{the new
  cha} with a 7-gram model.  In a pure maximum likelihood 7-gram, the
probability of seeing the letter \charmention{r} will be given by
dividing the number of times \stringmention{ew char} occurred in the
training corpus by the number of times \stringmention{ew cha} showed
up followed by any character.  For instance, there might be two
instances of \stringmention{ew chai}, perhaps as a part of
\stringmention{new chair} (furniture) or \stringmention{new chai}
(beverage) and one of \stringmention{ew chas}, perhaps as part of
\stringmention{grew chastened}, in the corpus.  In this case, the
probability of the next letter being \charmention{i} is estimated at
2/3 and the probability of the next letter being \charmention{s} at
1/3.  Note that there is no probability left over for the possibility
of \charmention{r} being next.  The maximum likelihood estimate for an
$n$-gram model assigns zero probability to any string containing an
$n$-gram that was not seen during training!  While this may not be a
problem in a short-context genetic modeling problem, it is a huge
problem for natural language texts, where there is a long tail of rare
$n$-grams which causes $n$-grams not seen during training to pop up in
test sets.  An easy example is numbers.  If we have 8-grams, only
8-digit numbers showing up in the training set are assigned non-zero
probability during prediction (testing).  There are $10^8$, that is
100 million, 8-digit numbers.  

\subsection{Witten-Bell Smoothing}

To get around the novel $n$-gram problem, LingPipe employs a common
language modeling technique known as \techdef{interpolative smoothing} to
ensure that every sequence of characters is assigned a non-zero
probability.  

LingPipe uses an approached to interpolative smoothing is a slight
generalization of the general method developed by 
Ian Witten and Timothy Bell in the context of text compression.%
%
\footnote{Witten, Ian H.\ and Timothy C.~Bell. 1991. The zero-frequency
  problem: estimating the probabilities of novel events in adaptive
  text compression. {\it IEEE Transactions on Information Theory} {\bf 37}(4).}
%
The idea behind Witten-Bell smoothing for $n$-grams involves blending
order $n$ predictions with order $n-1$ predictions, all the way down
to a basic uniform distribution at order 0.  Thus even if we haven't
ever seen \stringmention{ew char} in the corpus, we will still assign
it a non-zero probability.  That probability will be a weighted
average of the 7-gram estimate of \charmention{r} given
\stringmention{ew cha}, the 6-gram estimate of \charmention{r} given
\stringmention{w cha}, the 5-gram estimate of \charmention{r} given
\stringmention{ cha} (with an initial space), the 4-gram estimate of
\charmention{r} given \stringmention{cha}, the 3-gram estimate of
\charmention{r} given \stringmention{ha}, the 2-gram estimate of
\charmention{r} given \stringmention{a}, the 1-gram estimate of
\charmention{r} given \stringmention{} (empty context).  Even if
the higher-order $n$-grams provide zero probability estimates, the
lower-order $n$-grams are likely to have been trained on the suffix
of the sequence.  But even if the character is novel, we bottom out
in a uniform distribution, as we describe in the next section.

\subsection{Uniform Base Distribution}

This recursion bottoms out at the uniform estimate, which we consider
a 0-gram estimate for the sake of terminological consistency.  In the
uniform estimate, each character is given equal probabilty.  If there
are 256 possible characters, as in Latin 1, then each has a 1/256
probability estimate in the uniform base case.

Because we have a uniform base case, LingPipe's character language
models require the number of possible characters to be specified in
the constructor.  The default is \code{Character.MAX\_VALUE}, which is
$2^{16}-1$, which is a slight overestimate of the number of legal Unicode
UTF-16 values (see \refsec{char-primitive}).  The number of characters
can be set lower if we know we are dealing with ASCII or Latin 1 or
another restricted subset of Unicode characters.  

As long as we choose the number of characters to be greater than or
equal to the number of unique characters seen in the combined training and
test set, the probability estimates will be proper (that is, sum to
1 over all strings for sequence models and over all strings of a given
length for process models).


\subsection{Weighting the $n$-Grams}

Interpolated $n$-gram models take a weighted average of the
predictions of (maximum likelihood estimated) $n$-grams,
$(n-1)$-grams, $(n-2)$-grams, and so on, eventually bottoming out with
the uniform distribution over characters.  What we haven't said is how
the weights for each model are determined.  We'll provide precise
mathematical definitions in \refsec{char-lm-math}.  For now, we will
provide a more descriptive explanation.

The first thing to note is that the weighting is context dependent in
that it depends on the characters in the context from which we are
predicting the next character.  That is, if we are predicting the next
character from context \stringmention{ew char}, we will likely have
different weightings of the $n$-grams than if we predict from context
\stringmention{t of th}.

The weight assigned to the order $n$-gram model in the interpolated
model is based on two factors determined from the context for the
prediction and one parameter in the constructor.  For deciding how
heavily to weight an $n$-gram context of previous characters in
prediction, we take into account how many times that context has been
seen in the training data.  The more we have seen a given context of
previous characters in the training data, the more heavily its prediction
is weighted.  Basically, this just says that when we have more data,
we can trust the prediction more.  

The second factor for weighting $n$-grams is how many different
outcomes we've seen given that context.  If we have a context with
many different outcomes, we will weight it less for prediction. 

In addition to these contextual factors, LingPipe generalizes Witten
and Bell's original method of interpolation with a parameter that
controls the weighting between higher-order contexts and lower-order
contexts.  This parameter, which is named \code{lambdaFactor} in the
code, may be optionally supplied during construction.  In Witten and
Bell's original approach, the lambda factor was implicitly set to 1.0,
which is rarely an optimal value.  This interpolation factor
determines how much to weight higher-order contexts relative to
lower-order contexts.  The higher the value of this factor, the more
agressively we smooth.  In other words, with high values of the lambda
factor, the more heavily the lower-order $n$-grams are weighted.  The
interpolation factor defaults to the length of the $n$-gram, which
seems to work well in a large number of contexts.

\section{Underlying Sequence Counter}

The boundary language models are based on the sequence models.  The
sequence model probabilities are based on keeping a count of character
subsequences.  For both boundary and sequence models, the object that
manages the counts is accessible through the method
\code{substringCounter()}.  This returns an instance of
\code{TrieCharSeqCounter}.  The name reflects the data structure used,
a \techdef{trie}.  

With the sequence counter, raw counts of arbitrary substrings in
the training data may be queried and updated.  It is unlikely that
a client of the language model class will need to directly modify
the underlying counter by any operation other than pruning (see
\refsec{char-lm-pruning}).  

All of the counts are based on \code{long} values.  The sequence
counter efficiently stores values for low count sequences using
shorter integer representations.%
%
\footnote{Carpenter, Bob.  2005.  Scaling high-order character
language models to gigabytes. {\it ACL Software Workshop}.}


\section{Learning Curve Evaluation}

In this section, we provide a demo implementation of process languauge
models that operate online, one character at a time.  As each
character in a text is visited, it is predicted given the characters
that have come before.  After that, it is added to the training data.
This allows us to plot a learning curve that shows how our language
model is doing. 

\subsection{Connection to Compression}

It turns out that using a technique known as arithmetic compression
(which LingPipe does not support), it is possible to use the
probabilistic predictions of a language model to perform compression
of arbitrary sequences of bytes, such as files.  The key result is
that for every character $c$ we encounter after context $s$, we can
represent it using a number of bits equal to $\log_2 p(c|s)$.  That
is, the better we can predict the next character, as measured by the
probability assigned to it, the better we can perform compression.
This makes sense, as this is a measure of predictability, and
predictable texts are easily compressible no matter what compression
technique is used (language models, Zip-style, Burrows-Wheeler style,
etc.)  

Given the fundamental relation between prediction and compression, 
the number of bits required to compress a sequence is just the
sum of the log (base 2) probability estimates for the symbols in
the sequence.

Average, or total log probability is also the standard (and most
useful) overall measure of language model performance.  Because we can
learn on the fly, we can plot out the learning curve of average bits
per symbol as we move through the sequence.  By inspecting the
resulting learning curves, we can tell how the model improves with
more training data.  In every LingPipe application, such as tagging,
classification and spelling correction, language models with better
held-out predictive accuracy provide better accuracy to the
application of which they are a part.


\subsection{Code Walkthrough}

This demo introduces a few new tools from LingPipe, which we will
explain as we go.  Two of these new methods are language-model
specific, including a method to train a single character given a
context and a method to estimate the probability of a character given
the previous characters.

Because we are thinking of our text as a streaming sequence of
characters, we use a process language model.  

As usual, we have a number of parameters supplied to the \code{main()}
method by the command line.  In order, we assume an integer
\code{ngram} for the $n$-gram length, an integer \code{numChars} for
the total number of unique characters possible (not necessarily
observed) in the train and test sets, a double \code{lambdaFactor} for
the interpolation ratio, and a file \code{dataDir} for the directory
in which to find the test texts, arranged with one test per file.
This will allow you to easily test your own data by just pointing this
demo at the directory containing the text.

To start, we iterate over the files in the data directory, extracting
their characters and setting up a language model with the parameters
specified on the command line.
%
\codeblock{LmLearningCurve.1}
%
The characters are extracted using LingPipe's static utility method
\code{readCharsFromFile}, which is in the
\code{com.aliasi.util.Files}.  Note that we have specified the
character encoding ISO-8859-1, aka Latin1.  The reason we do this is
that Latin1 characters stand in a one-to-one relationship with bytes.
By specifing Latin1, each character corresponds to a single byte with
the same value.  This lets us model arbitrary byte sequences using
characters.  

We then set up an instance of \code{OnlineNormalEstimator}, which
is a class in \code{com.aliasi.stats} that computes running averages
and variances, as we will see in the next block.

Next, we iterate over the characters in the text, first estimating 
each character's probability given the previous characters, then
adding the character to the training data (given its context).
%
\codeblock{LmLearningCurve.2}
%
First, we estimate the log (base 2) probability of the next character
given the characters we have already seen.  This method takes a
character array (\code{cs}), a start position (\code{0}), and an end
position (\code{n+1}).  The value returned is the probability of the
$n$-th character (\code{cs[n]}) given the previous characters
(\code{cs[0]} through \code{cs[n-1]}).  We use the value \code{0} as
the start position because the method is clever enough to only use as
long a context for prediction as it stores.

Next, we add the character to the training data.  This is done with
the model method \code{trainConditional()}, which is supplied with several
arguments.  First, \code{cs}, is the character slice.  Next is the
starting position from which to train.  Because we are using
$n$-grams, this is the current position minus the $n$-gram length, but
given that it cannot be less than 0, we use \code{max(0,n - ngram)}
as our starting position.  The third argument, here \code{n}, specifies
the final position which is trained.  The fourth and final argument,
here \code{max(0,n - 1)}, indicates the end of the context.  Here,
we just train a single character at a time, so the context ends one
position before the character we're training (\code{n - 1}).

Finally, we increment our average and deviation accumulator and
extract the current average and standard deviation.
%
\codeblock{LmLearningCurve.3}
%
This involves calling the \code{handle(double)} method on the online
normal estimator, which adds the predicted log probability to the set
of observed probabilities.  After that, we pull the current sample
mean (otherwise known as the average) and the sample standard
deviation (with an unbiased estimate) from the online normal
estimator.  These sample means and deviations are based on the
characters we have seen so far.

As usual, we have suppressed the various print statements.


\subsection{Running the Online Learning Curve Demo}

The Ant target \code{learning-curve} calls the demo, with
parameters supplied by properties \code{ngram} for the maximum
$n$-gram size, \code{num.chars} for the total number of
unique characters, \code{lambda.factor} for the interpolation
ratio, and \code{data.dir} for the directory in which the
data is found.

For test data, we use the Canterbury corpus (which we describe in
\refsec{corpora-canterbury}).  The Canterbury corpus is a standard
corpus introduced by Ross Arnold and Timothy Bell (of Witten and Bell
smoothing) to evaluate compression algorithms.  The Canterbury corpus
contains a mixture of document types, including novels, plays and
poetry, as well as C and Lisp source code, a Unix man page, a
Microsoft Excel binary spreadsheet (\code{.xls} format), and
executables for the SPARC architecture.  In \refsec{corpora-canterbury},
we describe the files more fully along with their sizes.

\commandlinefollow{ant -Dngram=6 -Dnum.chars=256 -Dlambda.factor=6.0 -Ddata.dir=../../data/canterbury learning-curve}
\begin{verbatim}
PROCESS LM PARAMETERS
    ngram=6
    numchars=256
    lambdaFactor=6.0
    data.dir=C:\lpb\data\canterbury

Processing File=..\..\data\canterbury\alice29.txt
   log2Prob[c[0]]=-8.000        mean=-8.000   sd= 0.000
   log2Prob[c[5000]]=-3.018     mean=-3.352   sd= 2.692
   log2Prob[c[10000]]=-3.382    mean=-2.972   sd= 2.575
   log2Prob[c[15000]]=-0.011    mean=-2.798   sd= 2.559
   log2Prob[c[20000]]=-5.706    mean=-2.701   sd= 2.533
...
   log2Prob[c[140000]]=-0.852   mean=-2.113   sd= 2.351
   log2Prob[c[145000]]=-2.414   mean=-2.105   sd= 2.348
   log2Prob[c[150000]]=-0.410   mean=-2.102   sd= 2.350
   log2Prob[c[152088]]=-24.140  mean=-2.103   sd= 2.352

Processing File=..\..\data\canterbury\asyoulik.txt
   log2Prob[c[0]]=-8.000        mean=-8.000   sd= 0.000
   log2Prob[c[5000]]=-0.386     mean=-3.397   sd= 2.462
   log2Prob[c[10000]]=-3.013    mean=-3.072   sd= 2.400
...
   log2Prob[c[120000]]=-1.099   mean=-2.342   sd= 2.309
   log2Prob[c[125000]]=-0.047   mean=-2.337   sd= 2.310
   log2Prob[c[125178]]=-0.001   mean=-2.337   sd= 2.310
...
\end{verbatim}
%
We've instrumented the code so that every 5000 characters it prints
out the prediction for that character and the running mean and
standard deviation from the set of samples so far.  We see that for
file \code{alice29.txt}, which is the text of Lewis Carroll's book
{\it Alice in Wonderland}, the final average log (base 2) probability
per character after all 152,088 characters have been processed is
-2.103.  This corresponds to about 2 bits per character, which is a
typical result for language models of text given this little training
data (a book is not much on the scale of the text available for
training, such as years of newswirte or MEDLINE, which run to
gigabytes, or the web, which runs to terabytes if not petabytes).

Also note the rather large sample standard deviation, which
indicates that there is a vast range of probability estimates.  For
instance, character 15,000 is predicted with log probabilty $-0.011$,
which is almost certainty, whereas character 20,000 is predicted with
probabilty $-5.7$, which isn't much better than chance given the set
of characters.  

Processing the entire corpus takes less than a minute on my aging
workstation.  The final results for the various files are as follows
(see \refsec{corpora-canterbury} for details on the contents of the files).
%
\begin{center}
\begin{tabular}{lllrrr}
\tblhead{File} & \tblhead{Bytes} & \tblhead{Avg} & \tblhead{SD} & \tblhead{GZip}
\\ \hline
\code{alice29.txt}  &    152,089 & -2.103 &  2.352 & 2.85 \\
\code{asyoulik.txt} &    125,179 & -2.337 &  2.310 & 3.12 \\
\code{cp.html}      &     24,603 & -2.285 &  2.725 & 2.59 \\ 
\code{fields.c}     &     11,150 & -2.020 &  2.584 & 2.24 \\
\code{grammar.lsp}  &      3,721 & -2.434 &  2.945 & 2.65 \\ 
\code{kennedy.xls}  &  1,029,744 & -1.753 &  3.632 & 1.63 \\
\code{lcet10.txt}   &    426,754 & -1.882 &  2.299 & 2.71 \\
\code{plrabn12.txt} &    481,861 & -2.204 &  2.200 & 3.23 \\
\code{ptt5}         &    513,216 & -0.829 &  2.484 & 0.82 \\
\code{sum}          &     38,240 & -2.702 &  3.451 & 2.67 \\ 
\code{xargs.1}      &      4,227 & -2.952 &  2.649 & 3.31 
\end{tabular}
\end{center}
%
We have also presented compressions results based on GZip, expressed
as the number of bits per byte required for the compressed file.%
%
\footnote{Given on the Canterbury corpus page at
\url{http://corpus.canterbury.ac.nz/details/cantrbry/RatioByRatio.html}.}
%
Note that for language data, our 6-gram model with default parameters
is far superior to GZip, with GZip being close or slightly better on
some of the binary data, like the spreadsheet.

These are not optimal parameters for our model, but rather defaults.
We recommend using 5-grams or 6-grams for English text in situations
where there is relatively little training data (as in these examples).
We also recommend setting the interpolation ratio to the $n$-gram length.
We encourage you to try diffeerent parameters, especially for $n$-gram
length and interpolation ratio; you should leave the number of
characters fixed at 256, which is the maximum number of Latin1
characters.



\section{Pruning Counts}\label{section:char-lm-pruning}

One problem with the naive application of $n$-gram character language
models is that the size of the model increases with the amount of
training data.%
%
\footnote{This is a general property of so-called
  \techdef{non-parameteric models}.  This terminology is confusing,
  because non-parameteric models rather than being parameter free,
  have an unbounded number of parameters.  The unboundedness comes
  from the fact that such models store the data they are trained on.
  Another example of a non-parametric model in LingPipe is
  $K$-nearest-neighbor classifification, which stores all of
  the training instances to compare to the test instances.}
%
Although the model size increases sublinearly due to repeated
subsequences,%
%
\footnote{Sublinear growth means that the model size is less than a
  constant factor times the size of the training data, meaning less
  than $\bigO{n}$.  The reason storage grows non-linearly is due to
  repeated $n$-grams, which must repeat with a finite.  As a
  non-linguistic example, binary search grows sublinearly with size,
  being $\bigO{\log n}$.}
%
it can grow very large relative to available storage with
long $n$-grams.

One way to keep the memory constrained is to \techdef{prune} the
underlying counts from time to time (the etymology of the term is that
it's typically used for tree-like data structures, such as the ones
used for language models).  Pruning works by removing all strings
whose count falls below some minimum value.  If there are millions of
such contexts, removing the ones with low count often has very little
effect on the overall probabilities assigned by the model.  One reason
for this is that the low counts tend to be associated with long
sequences that are low frequency.  These tend to have shorter
sequences which are almost as representative, but slightly higher
frequency.  Another reason is that the low frequency counts are for
low frequency training events which tend to occur less often in new
data than high frequency training events; if not, there's something
wrong with the training data that's making it unlike the test data.



\section{Compling and Serializing Character LMs}

Both implementations of character language models,
\code{NGramProcessLM} and \code{NGramBoundaryLM}, implement Java's
\code{Serializable} interface (see \refsec{io-serializable}) and
LingPipe's \code{Compilable} interface (see \refsec{io-compilable}).

\subsection{Compilation}

Like many of LingPipe's statistical models, the character language
model implementations come in two flavors.  First, there are the two
classes we've discussed already, \code{NGramProcessLM} and
\code{NGramBoundaryLM}, which implement handlers for character
sequences for training.  These two classes are both compilable (see
\refsec{io-compilable} for a general overview of the \code{Compilable}
interface in LingPipe).  Compiling an instance of one of these classes
produces instances of \code{CompiledNGramProcessLM} and
\code{CompiledNGramBoundaryLM}.  Although based on shared abstract
superclasses, these abstract bases are not public because there is a
sufficiently rich set of interfaces (see \refsec{char-lm-interfaces}
for the complete list).

Compiled character language models are much much faster than the
dynamic language models that allow training.  The reason for this is
threefold.  First, all of the smoothing is done at compile time, so
only a single $n$-gram need be evaluated, the one with the longest
matching context.  Second, a suffix tree data structure is used to
make it efficient to find the longest suffix for sequences of
evaluations.  Third, all of the logarithms are precomputed, so only
addition is required at run time (see \refeq{char-lm-log-scale}).  

Compiled language models only support the estimation interfaces,
though they support both the joint (sequence of characters) and
conditional (one character given previous characters) forms of
estimation.  The only other method they support is
\code{observedCharacters()}, which returns an array containing all of
the characters that have been seen (including the boundary character
for sequence models).

The classes that depend on character language models use the dynamic
forms of language models for training and the compiled form in compiled
models.

\subsection{Serialization}

Character language models are also serializlable.   The deserialized
version is an instance of the same class as the language model that
was serialized, and will have exactly the same set of stored $n$-grams
and predictions.  

A typical use case is to train a language model on some data, then
compile it.  But that does not allow further training.  If the model
is also serialized, then it may be deserialized, trained with further
data, then compiled for use and serialized for later training.


\section{Thread Safety}

As with most of LingPipe's classes, language models are thread safe
under read-write synchronization.  Specifically, any number of
operations that read from language models, including all of the
estimation and sequence counter operations may be run concurrently
with each other. Write operations on the other hand, like training,
pruning, and setting configuration parameters like the interpolation
ratio, must operate exclusively of all other read or write operations.



\section{The Mathematical Model}\label{section:char-lm-math}

The mathematics of character language models are straightforward.
We'll begin with the process language model and move onto the
bounded language model.

\subsection{Strings}

First we'll establish some notation for talking about strings
mathematically.  Suppose we have an alphabet $\Sigma$ of characters.%
%
\footnote{For character language models, $\Sigma$ will consist of
  UTF-16 byte pairs.  If we stay within the basic multilingual plane
  (BMP) of Unicode, these will correspond to Unicode characters (code
  points).  If the characters used go beyond the BMP, some of these
  may correspond to the high or low half of a surroage pair.}
%
Let $\Sigma^n$ be the set of all sequences of such characters of
length $n$, which are called strings.  In symbols,
%
\begin{equation}
\Sigma^n = \setcomp{(s_1,\ldots,s_N)}{s_n \in \Sigma \mbox{ for } 1 \leq n \leq N}.
\end{equation}
%
Given a string, we often write $\strlen{s} = n$ if $s \in \Sigma^n$ and
say that $s$ is of length $n$.

Next, we take $\Sigma^*$ to be the set of all strings, defined by
%
\begin{equation}
\Sigma^* = \bigcup_{n  \in \nats} \Sigma^n.
\end{equation}
%
We will use the symbol $\emptystring$ is used for the \techdef{empty
  string}, which is defined as the unique sequence of length 0.

The fundamental operation over strings is \techdef{concatenation}.
The concatenation of two strings $s = (s_1,\ldots,s_M)$ of length
$M$ and $t = (t_1,\ldots,T_N)$ of length $N$ is taken to be the
%
\begin{equation}
\strcat{s}{t} = (s_1,\ldots,S_M,t_1,\ldots,T_N).
\end{equation}

Strings have several pleasant algebraic properties.   The
empty string is the identity for concatenation, so that
%
\begin{equation}
\strcat{\emptystring}{s} = \strcat{s}{\emptystring} = s.
\end{equation}
%
Concatenation is associative, so that
%
\begin{equation}
\strcat{(\strcat{s}{t})}{u} = \strcat{s}{(\strcat{t}{u})}.
\end{equation}
%
Thus the order of concatenation of a sequence of strings doesn't
matter.  Together, these two properties define what is known as a
\techdef{monoid}.  Note that the natural numbers ($\nats$) under
addition (+) form a monoid with identity zero (0).  By analogy to the
natural numbers, if $s$ is a string, we write $s^n$ for the string
consisting of $n$ concatenations of $s$.  Thus $s^0 = \emptystring$
is the empty string.

Of course, unlike the case for arithmetic addition, concatenation is
not a symmetric operation, so that we don't, in general, have
$\strcat{s}{t} = \strcat{t}{s}$. 




\subsection{The Chain Rule}

The basic idea of $n$-gram language models is to use the chain rule,
which is derivable from the definition of joint and conditional
probabilities, to factor the probabiltiy of a string into the product
of the probabilities of each of its characters given the characters
that came before it.  In symbols, suppose we have a string
$s_1,\ldots,s_N$ of length $N$.  The chain rule tells us we can
compute its probability as
%
\begin{equation}
p(s_1,\ldots,s_k) = p(s_1) \times p(s_2|s_1) \times \cdots \times p(s_N|s_1,\ldots,s_{N-1}).\label{eq:char-lm-chain-rule}
\end{equation}
%

\subsection{Markov Chains}

The point of using a limited-context-length $n$-gram is that you only
look back at most $(n-1)$ characters.  To specify that our models have
limited context, we truncate the contexts in the chain rule, giving us
%
\begin{equation}
p(s_N|s_1,\ldots,s_{N-1}) = p(s_k|s_{k-n+1},\ldots,s_{k-1}).\label{eq:char-lm-finite-context}
\end{equation}
%
Note that unlike the chain rule, which holds for any probabilities at
all in full generality, the use of limited contexts is a modeling
decision.  We know that this decision is only an approximation for
natural language, which exhibits much longer range dependencies.  For
instance, if a person named \stringmention{Tony} is mentioned in an
article, the name is likely to show up again much later in the
document.  

Combining the chain rule in \refeq{char-lm-chain-rule} with our
limited context model \refeq{char-lm-finite-context}, we have
%
\begin{equation}
p(s_1,\ldots,s_k) 
= p(s_1) \times p(s_2|s_1) \times \cdots \times p(s_k|s_{k-n+1},\ldots,s_{k-1}).
\end{equation}
%
For instance, if we have a 2-gram model and string \stringmention{abra},
our model factors the probabilty of \stringmention{abra} as
%
\begin{equation}
p(\stringmention{abra}) 
= p(\stringmention{a}) 
\times p(\stringmention{b}|\stringmention{a}) 
\times p(\stringmention{r}|\stringmention{b}) 
\times p(\stringmention{a}|\stringmention{r}).
\end{equation}
%
Assuming that the contexts we use for prediction are finitely bounded,
meaning that predicting the next character involves only a fixed
finite window, we have what is known as a \techdef{Markov model}.  A
particular sequence of characters generated from such a model is then
called a \techdef{Markov chain}.  In a Markov model of character
string generation, each character is generated based on finitely many
previous characters.  Note that being Markovian doesn't say how the
probabilty of the next character is estimated, but only that it
depends on at most a bounded finite history.

\subsection{Training Data Sequence Counts}

Now suppose we have seen training data in the form of a set of $M$
strings $\sigma_1,\ldots,\sigma_M$.  Given such a training set, let
$\cnt{s}$ be the number of times the string $s$ appeared as a
substring of one of the training strings $\sigma_m$.  For instance, if
the training strings are \stringmention{abe} and
\stringmention{abracadabra}, the substring \stringmention{ab} shows up
3 times, once in the first string and twice in the second.

Let $\extcnt{\sigma}$ be the count of all extensions by one character
of the string $\sigma$, that is
%
\begin{equation}
\extcnt{\sigma} = \sum_{s' \in \Sigma} \cnt{\sigma \cdot s}.\label{eq:char-lm-extcnt}
\end{equation}
%
For instance, in the training data, the count of \stringmention{ra} is
2, but the extension count of \stringmention{ra} is only 1, because
the second training string ends with \stringmention{ra} which is not
extended.


\subsection{Maximum Likelihood Estimates}

We'll use the notation $\pmle(s_n|s_1,\ldots,s_{n-1})$ for
the maximum likelihood estimate of the conditional probabilty of a
seeing the character $s_n$ given the previous $(n-1)$ characters
$s_1,\ldots,s_{n-1}$.  The maximum likelihood estimate model is
derived in the usual way, by counting in the training data, giving us
%
\begin{equation}
\pmle(s_n|s_1,\ldots,s_{n-1})
= \frac{\cnt{s_1,\ldots,s_{n-1},s_{n}}}
       {\extcnt{s_1,\ldots,s_{n-1}}}.
\end{equation}
%
That is, we count up all the times we've seen the character in
question following the string and divide by the number of times
the string was extended by any character.

These maximum likelihood estimates will almost certainly assign zero
probability to many characters in even relatively modest length
$n$-grams. 

\subsection{Witten-Bell Smoothing}

We follow Witten and Bell's approach to smoothing these maximum
likelihood estimates through interpolation.  The basic definition is
recursive, interpolating the maximum likelihood estimate at length $n$
with the smoothed estimate of length $(n-1)$.  In symbols, we let
$\lambda_{s_1,\ldots,s_{n-1}} \in [0,1]$ be the weight of the maximum
likelihood estimate for predicting the next character given that the
previous characters were $s_1,\ldots,s_{n-1}$.  The recursive case
of the definition is then given by
%
\begin{align}
p(s_n|s_1,\ldots,s_{n-1})  = \ & \lambda_{s_1,\ldots,s_{n-1}} \times \pmle(s_n|s_1,\ldots,s_{n-1})
\\[4pt]
&{} + (1 - \lambda_{s_1,\ldots,s_{n-1}}) \times p(s_n|s_1,\ldots,s_{n-2}).\nonumber
\end{align}
%
On each application of this rule, the context is shortened by one
character.

Next, we need to define the interpolation ratios $\lambda_s \in [0,1]$
for arbitrary strings $s \in \Sigma^*$.  Witten and Bell define the
interpolation ratio by setting
%
\begin{equation}
\lambda_s 
= 
\frac{\extcnt{s}}
     {\extcnt{s} + (L \times \numexts{s})},
\end{equation}
%
where $L \geq 0$ is the interpolation parameter specified in the
language model constructor, $\extcnt{s}$ is defined as in
\refeq{char-lm-extcnt}, and $\numexts{s}$ is defined as the
number of unique characters were observed following the context $s$,
%
\begin{equation}
\numexts{s} = \cardinality{\setcomp{c}{\cnt{\strcat{s}{c}} > 0}}
\end{equation}
%
where $\cardinality{x}$ is the cardinality (number of members) of the
set $x$.

The recursion bottoms out by interpolating with the uniform
distribution $\punif(s_n)$.  If $K$ is the total number
of characters, then the uniform distribution over characters is
defined by
%
\begin{equation}
\punif(s_n) = 1/K.
\end{equation}
%
Thus the recursion bottoms out by interpolating the
maximum likelihood unigram model with the uniform distribution, 
%
\begin{equation}
p(s_n) = \lambda_{\emptystring} \times \pmle(s_n)
+ (1 - \lambda_{\emptystring}) \times \punif(s_n),
\end{equation}
%
where the interpolation ratio $\lambda_{\emptystring}$ is defined as
usual.

\subsection{Process Language Models}

The language models we have just defined following Witten and
Bell's definitions are process language models.%
%
\footnote{The terminology, which is not standard in the
  language-modeling literature, derives from the fact that generating
  the sequence of characters forms what is known as a \techdef{random
    process.}}
%
For process language models, if we let $p(s)$ be the probability of a
string $s$, then
%
\begin{equation}
\sum_{s \in \Sigma^n} p(s) = 1.\label{eq:char-lm-process-norm}
\end{equation}
%


\subsection{Sequence Language Models}

Unlike the process language models, for which the probabilities of all
strings of a given length sum to 1, a bounded language model is a
proper model of arbitrary character sequences.  Thus a bounded
language model, with probability $p(s)$, is such that
%
\begin{equation}
\sum_{s \in \Sigma^*} p(s) = 1.\label{eq:char-lm-seq-norm}
\end{equation}
%
The only difference between this equation and \refeq{char-lm-process-norm}
is that the sum is over all strings, $\Sigma^*$, rather than over
strings of a fixed length $n$, $\Sigma^n$.

Thus we think of sequence language models as generating finite
sequences of characters then stopping; a process language model
goes on generating characters indefinitely.

We follow the fairly standard, though often undocumented, procedure of
definining bounded language models in terms of process language
models.  Suppose we are definining sequences over the character set
$\Sigma$.  We will choose a new character, $b$, called the boundary
character, such that $b \not\in \Sigma$.  We then define a process
language model over the extended alphabet $\Sigma \cup \setext{b}$ and
use it to define our sequence language model distribution.

Despite adding a new boundary symbol $b$, our probabilities are still
taken to be defined over strings in $\Sigma^*$.  Specifically, we
define the probability of a string $s = s_1,\ldots,s_M \in \Sigma^*$
as
%
\begin{equation}
p_B(s_1,\ldots,s_M) = p_B(s_1,\ldots,s_M,b|b),\label{eq:char-lm-seq-def}
\end{equation}
%
where $p_B()$ is the sequence language model probability distribution
and $p()$ is the underlying process language model.  In words, we
start with the boundary character $b$ as context, then generate our
string $s = s_1,\ldots,s_M$, then generate a final boundary character
$b$.  Note that we chose $s$ from the unaugmented set of strings
$\Sigma^*$.

Because we now have a termination condition for our character
generating process, it is fairly straightforward to verify that the
process generates a probability distribution normalized over all
strings, as shown in \refeq{char-lm-seq-norm}.  We start with the boundary
character, though don't generate it, which is why it's in the context
in \refeq{char-lm-seq-def}.  We then continue generating, at each point
deciding whether to generate the boundary character $b$ and stop or to
generate a non-boundary character and keep going.  Because the sum of
each next character in the augmented alphabet $\Sigma \cup \setext{b}$
sums to 1.0, the normalization in \refeq{char-lm-seq-norm} holds.











