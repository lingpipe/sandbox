\chapter{Java Basics}\label{chapter:java}

In this chapter, we go over some of the basics of Java that are
particularly relevant for text and numerical processing.  

\section{Numbers}\label{section:java-numbers}

In this section, we explain different numerical bases, including
decimal, octal, hexadecimal and binary.  

\subsection{Digits and Decimal Numbers}

Typically we write numbers in the Arabic form (as opposed to, say, the
Roman form), using decimal notation.  That is, we employ sequences of the
ten digits 0, 1, 2, 3, 4, 5, 6, 7, 8, 9.  

A number such as 23 may be decomposed into a 2 in the ``tens place''
and a 3 in the ``ones place''.  What that means is that $23 = (2 \times
10) + (3 \times 1)$; similarly $4700 = (4 \times 1000) + (7 \times
100)$.  We can write these equivalently as $23 = 2 \times 10^1 + 3 \times 10^0$
and $4700 = (4 \times 10^3) + (7 \times 10^2)$.  Because of the base
of the exponent, decimal notation is also called ``base 10''.

The number 0 is special.  It's the additive identity, meaning
that for any number $x$, $0 + x = x + 0 = x$.  

We also conventionally use negative numbers and negation.  For
instance, -22 is read as ``negative 22''.  We have to add 22 to it to
get back to zero.  That is, negation is the additive inverse, so that
$x + (-x) = (-x) + x = 0$.

The number 1 is also special.  1 is the multiplicative identity,
so that $1 \times x = x \times 1 = x$.  Division is multiplicative
inverse, so that for all numbers $x$ other than 0, $\frac{x}{x} = 1$.

We also use fractions, written after the decimal place.  Fractions are
defined using negative exponents.  For instance $.2 = 2 \times 10^{-1}
= \frac{2}{10^1}$, and $.85 = .8 \times 10^{-1} + .5 \times 10^{-2}$.

For really large or really small numbers, we use scientific notation,
which decomposes a value into a number times an exponent of 10.  For
instance, we write 4700 as $4.7 \times 10^3$ and 0.0047 as $4.7 \times
10^{-3}$.  In computer languages, 4700 and 0.0047 are typically
written as \code{4.7E+3} and \code{4.7E-3}.  Java's floating point
numbers may be input and output in scientific notation.

\subsection{Bits and Binary Numbers}

Rather than the decimal system, computers work in the binary system,
where there are only two values.  In binary arithmetic, bits play the
role that digits play in the decimal system.  A \techdef{bit} can have
the value 0 or 1.  

A number in \techdefs{binary notation}{binary} consists of a sequence
of bits (0s and 1s).  Bits in binary binary numbers play the same role
as digits in decimal numbers; the only difference is that the base is
2 rather than 10.  For instance, in binary, $101 = (1 \times 2^3) + (0
\times 2^2) + (1 \times 2^0)$, which is 7 in decimal notation.  Fractions
can be handled the same way as in decimal numbers.  Scientific notation
is not generally used for binary numbers.

\subsection{Octal Notation}

Two other schemes for representing numbers are common in computer
languages, octal and hexadecimal.  As may be gathered from its name,
\techdef{octal}{octal} is base 8, conventionally written using the
digits 0--7.  Numbers are read in the usual way, so that octal 43 is
expanded as $(4 \times 8^1) + (3 \times 8^0)$, or 35 in decimal
notation.

In Java (and many other computer languages), octal
notation\index{octal!notation} is very confusing.  Prefixing a numeric
literal with a \code{0} (that's a zero, not a capital o) leads to it
being interpreted as octal.  For instance, the Java literal \code{043}
is interpreted as the decimal 35.

\subsection{Hexadecimal Notation}

\techdefs{Hexadecimal}{hexadecimal} is base 16, 
and thus we need some additional symbols.  The first 16 numbers in hex
are conventionally written

\displ{0, 1, 2, 3, 4, 5, 6, 7, 8, 9, A, B, C, D, E, F.}

In hexadecimal, the value of A is what we'd write as 10 in decimal
notation.  Similarly, C has the value 12 and F the value 15.  We read
off compound numbers in the usual way, so that in hex, $93 = (9 \times
16^1) + (3 \times 16^0)$, or 138 in decimal notation.  Similarly, in
hex, the number $\mbox{\rm B2F} = (11 \times 16^2) + (2 \times 16^1) + (15 \times
16^0)$, or 2863 in decimal notation.

In Java (and other languages), hexadecimal
numbers\index{hexadecimal!notation} are distinguished by prefixing
them with \code{0x}.  For example,
\code{0xB2F} is the hexadecimal equivalent of the decimal 2863.

\subsection{Bytes}\label{section:java-bytes} 

The basic unit of organization in contemporary computing systems is
the byte.  By this we mean it's the smallest chunk of data that may be
accessed programmatically, though in point of fact, hardware often
groups bytes together into larger groups that it operates on all at
once.  For instance, 32-bit architectures often operate on a sequence
of 4 bytes at once and 64-bit architectures on 8 bytes at once.  The
term ``word'' is ambiguously applied to a sequence of two bytes, or to
the size of the sequence of bytes at which a particular piece of
hardware operates.

A \techdef{byte} is a sequence of 8 bits, and is sometimes
called an octet for that reason.  Thus there are 256 distinct bytes,
ranging from \code{00000000} to \code{11111111}.  The bits are read
from the high (left) end to the low (right end).

In Java, the \code{byte} primitive type\index{Java!primitive!byte} is
signed.  Bytes between \code{00000000} and \code{01111111} are
interpreted as a binary number with decimal value between 0 and 127
(inclusive).  If the high order bit is 1, the value is interpreted as
negative.  The negative value is value of the least significant 7 bits
minus 128.  For instance,
\code{10000011} is interpreted as $3 - 128 = -125$, because
\code{00000011} (the result of setting the high bit to \code{0}) is interpreted as 3.

The unsigned value of a byte \code{b} is returned by \code{(b~<~0 ?\
(b~+~256) :\ b)}.  

The primitive data type for computers is a sequence of bytes.  For
instance, the contents of a file is a sequence of bytes, and so
is the response from a web server to an HTTP request for a web page.
Most importantly for using LingPipe, sequences of characters are
represented by sequences of bytes.

\subsection{Code Example: Integral Number Bases}

There is a simple program in the code directory for this chapter
%
\displ{\filepath{src/java/src/com/lingpipe/book/basics/ByteTable.java}}
%
for displaying bytes, their corresponding unsigned value, and the
conversion of the unsigned value to octal, hexadecimal and binary
notations.  The work is done by the loop
%
\codeblock{ByteTable.1}
%
This code may be run from Ant by first changing into this chapter's
directory and then invoking the \code{byte-table} target,
%
\commandlinefollow{cd \rootdir/src/java}%
\commandline{ant byte-table}
%
The output, after trimming the filler generated by Ant, looks like
%
\begin{verbatim}
BASES
 10    -10    8  16         2

  0      0    0   0         0
  1      1    1   1         1
  2      2    2   2        10
...
  9      9   11   9      1001
 10     10   12   a      1010
 11     11   13   b      1011
...
127    127  177  7f   1111111
128   -128  200  80  10000000
129   -127  201  81  10000001
...
254     -2  376  fe  11111110
255     -1  377  ff  11111111
\end{verbatim}

\subsection{Other Primitive Numbers}

In addition to bytes, Java provides three other primitive integer
types.  Each has a fixed width in bytes.  Values of type \code{short}
occupy 2 bytes, \code{int} 4 bytes, and \code{long} 8 bytes.  All of
them use the same signed notation as \code{byte}, only with more bits.

\subsection{Floating Point}\index{arithmetic!floating point}

Java has two floating point types, \code{float} and \code{double}.  A
\code{float} is represented with 4 bytes and said to be single
precision, whereas a \code{double} is represented with 8 bytes and
said to be double precision.

In addition to finite values, Java follows the IEEE~754 floating point
standard in providing three additional values.  There is positive
infinity, conventionally $\infty$ in mathematical notation, and
referenced for floating point values by the static constant
\code{Float.POSITIVE\_INFINITY} in Java.  There is also negative
infinity, $-\infty$, referenced by the constant
\code{Float.NEGATIVE\_INFINITY}.  There is also an ``undefined''
value, picked out by the constant \code{Float.NaN}.  There are
corresponding constants in \code{Double} with the same names.

If any value in an expression is \code{NaN}, the result is \code{NaN}.
A \code{NaN} result is also returned if you try to divide 0 by 0,
subtract an infinite number from itself (or equivalently add a positive
and negative infinite number), divide one infinite number
by another, or multiple an infinite number by 0.

Both \code{n/Double.POSITIVE\_INFINITY} and
\code{n/Double.NEGATIVE\_INFINITY} evaluate to 0 if \code{n} is finite
and non-negative.  Conversely, \code{n/0} evaluates to $\infty$ if
\code{n} is positive and $-\infty$ if \code{n} is negative.
The result of multiplying two infinite number is $\infty$ if they are
both positive or both negative, and $-\infty$ otherwise.  If
\code{Double.POSITIVE\_INFINITY} is added to itself, the result is
itself, and the same for \code{Double.NEGATIVE\_INFINITY}.  If one is
added to the other, the result is \code{NaN}.  The negation of an
infinite number is the infinite number of opposite sign.  

Monotonic transcendental operations like exponentiation and logarithms
also play nicely with infinite numbers.  In particular, the log of a
negative number is \code{NaN}, the log of 0 is negative infinity, and
the log of positive infinity is positive infinity.  The exponent of
positive infinity is positive infinity and the exponent of negative
infinity is zero.



\section{Objects}

Every object in Java inherits from the base class \code{Object}.
There are several kinds of methods defined for \code{Object}, which
are thus defined on every Java object.

\subsection{Equality}

Java distinguishes reference equality, written \code{==}, from object
equality.  Reference equality requires that the two variables refer to
the exact same object.  Some objects may be equal, even if they are
not identical.  To handle this possibility, the \code{Object} class
defines a method \code{equals(Object)} returning a \code{boolean}
result.  

In the \code{Object} class itself, the equality method
delegates to reference equality.  Thus subclasses that do not override
\code{Object}'s implementation of equality will have their equality
conditions defined by reference equality.

Subclasses that define their own notion of equality, such as
\code{String}, should do so consistently.  Specifically, equality
should form an equivalence relation.  First, equality should
be reflexive, so that every object is equal to itself; 
\code{x.equals(x)} is always true for non-null \code{x}.  Second,
equality should be symmetric, so if \code{x} is equal to \code{y},
\code{y} is equal to \code{x}; \code{x.equals(y)} is true if
and only if \code{y.equals(x)} is true.  Third, equality should be
transitive, so that if \code{x} is equal to \code{y} and \code{y} is
equal to \code{z}, then \code{x} is equal to \code{z};
\code{x.equals(y)} and \code{y.equals(z)} both being true imply
\code{x.equals(z)} is true.

\subsection{Hash Codes}

The \code{Object} class also defines a \code{hashCode()} method,
returning an \code{int} value.  This value is typically used in
collections as a first approximation to equality.  Thus the
specification requires consistency with equality in that if two
objects are equal as defined by \code{equals()}, they must have the
same hash code.

In order for collections and other
hash-code dependent objects to function properly, hash codes should be
defined

Although the implementation of \code{hashCode()} is not determined for
\code{Object} by the language specification, it is required to be 
consistent with equality.  Thus any two objects that are reference
equal must have the same hash code. 

\subsection{String Representations}

The \code{Object} class defines a \code{toString()} method returning a
\code{String}.  The default behavior is to return the name of the
class followed by a hexadecimal representation of its hash code.  This
method should be overridden in subclasses primarily to help with
debugging.  Some subclasses use \code{toString()} for real work, too,
as we will see with \code{StringBuilder}.

\subsection{Other Object Methods}

\subsubsection{Determining an Object's Class}

The runtime class of an object is returned by the \code{getClass()}
method.  This is particularly useful for deserialized objects and
other objects passed in whose class might not be known.  It is neither
very clear nor very efficient to exploit \code{getClass()} for 
branching logic.  Note that because generics are erased at runtime,
there is no generic information in the return.

\subsubsection{Finalization}

At some point after there are no more references to an object left
alive in a program, it will be finalized by a call to \code{finalize()}
by the garbage collector.  Using finalization is tricky and we will not
have cause to do so in this book.  For some reason, it was defined
to throw arbitrary throwables (the superinterface of exceptions). 

\subsubsection{Threading}

Java supports multithreaded applications as part of its language
specification.  The primitives for synchronization are built on top of
the \code{Object} class.  Each object may be used as a lock to
synchronize arbitrary blocks of code using the \code{synchronized}
keyword.  Methods may also be marked with \code{synchronized}, which
is equivalent to explicitly marking their code block as synchronized.

In addition, there are three \code{wait()} methods that cause the
current thread to pause and wait for a notification, and two
\code{notify()} methods that wake up (one or all) threads waiting
for this object.  

Rather than dealing with Java concurrency directly, we strongly
recommend the \code{java.util.concurrent} library for handling
threading of any complexity beyond exclusive synchronization.

\subsection{Object Size}

There's no getting around it -- Java objects are heavy.  The lack of a
C-like \code{struct} type presents a serious obstacle to even
medium-performance computing.  To get around this problem, in places
where you might declare an array of \code{struct} type in C, LingPipe
employs parallel arrays rather than arrays of objects.

The language specification does not describe how objects are
represented on disk, and different JVMs may handle things differently.
We'll discuss how the 32-bit and 64-bit Sun/Oracle JVMs work.

\subsubsection{Header}

Objects consist of a header and the values of their member variables.
The header itself consists of two references.  The first establishes
the object's identity.  This identity reference is used for reference
equality (\code{==}) and thus for locking, and also as a return value
for the \code{hashCode()} implementation in \code{Object}.  Even if an
object is relocated in memory, its reference identity and locking
properties remain fixed.  The second piece of information is a
reference to the class definition.

A reference-sized chunk of memory is used for each piece of
information.  In a a 32-bit architecture, references are four bytes.
In 64-bit architectures, they are 8 bytes by default.  

As of build 14 of Java SE 6, the \code{-XX:+UseCompressedOops} option
may be used as an argument to \code{java} on 64-bit JVMs to instruct
them to compress their ordinary object pointers (OOP) to 32-bit
representations.  This saves an enormous amount of space, cutting the
size of a simple \code{Object} in half.  It does limit the heap size
to 32 gigabytes (around four billion objects), which should be sufficient for most
applications.  As of build 18 of Java SE 6, this flag is on by default
based on the maximum heap size (option \code{-Xmx}).

\subsubsection{Member Values}

Additional storage is required for member variables.  Each primitive
member variable requires at least the amount of storage required for
the primitive \eg{1 byte for \code{byte}, 2 bytes for \code{short},
and 8 bytes for \code{double}}.  References will be 4 bytes in
a 32-bit JVM and a 64-bit JVM using compressed pointers, and 8 bytes
in the default 64-bit JVM.

To help with word alignment, values are ordered in decreasing order of
size, with doubles and longs coming first, then references, then ints
and floats, down to bytes.

Finally, the object size is rounded up so that its overall size
is a multiple of 8 bytes.  This is so all the internal 8-byte
objects inside of an object line up on 8-byte word boundaries.

Static variables are stored at the class level, not at the individual
object level.

\subsubsection{Subclasses}

Subclasses reference their own class definition, which will in turn
reference the superclass definition.  They lay their superclass member
variables out the same way as their superclass.  They then lay out
their own member variables.  If the superclass needed to be padded, some
space is recoverable by reordering the subclass member variables
slightly.

\subsection{Number Objects}

For each of the primitive number types, there is a corresponding class
for representing the object.  Specifically, the number classes for
integers are \code{Byte}, \code{Short}, \code{Integer}, and
\code{Long}.  For floating point, there is \code{Float} and
\code{Double}.  These objects are all in the base package
\code{java.lang}, so they do not need to be explicitly imported before
they are used.  Objects are useful because many of the underlying Java
libraries, particular the collection framework in \code{java.util},
operate over objects, rather than over primitives.%
%
\footnote{Open source libraries are available for primitive collections.
The Jakarta Commons Primitives and Carrot Search's
High Performance Primitive Collections are released under the Apache license,
and GNU Trove under the Lessger GNU Public License (LGPL).}

Each of these wrapper classes holds an immutable reference to an
object of the underlying type.  Thus the class is said to ``box'' the
underlying primitive (as in ``put in a box'').  Each object has a
corresponding method to return the underlying object.  For instance,
to get the underlying byte from a \code{Byte} object, use
\code{byteValue()}.

Two numerical objects are equal if and only
if they are of the same type and refer to the same primitive value.
Each of the numerical object classes is both serializable and
comparable, with comparison being defined numerically.


\subsubsection{Constructing Numerical Objects}

Number objects may be created using one-argument constructors.  For
instance, we can construct an object for the integer 42 using
\code{new~Integer(42)}.  

The problem with construction is that a fresh object is allocated for
each call to \code{new}.  Because the underlying references are
immutable, we only need a single instance of each object for each
underlying primitive value.  The current JVM implementations are smart
enough to save and reuse numerical objects rather than creating new
ones.  The preferred way to acquire a reference to a numerical object
is to use the \code{valueOf()} static factory method from the
appropriate class.  For example, \code{Integer.valueOf(42)} returns an
object that is equal to the result of \code{new~Integer(42)}, but is
not guaranteed to be a fresh instance, and thus may be reference equal
(\code{==}) to other \code{Integer} objects.

\subsubsection{Autoboxing}\label{section:java-autoboxing}

Autoboxing automatically provides an object wrapper for a primitive
type when a primitive expression is used where an object is expected.
For instance, it is legal to write
%
\begin{verbatim}
Integer i = 7;
\end{verbatim}
%
Here we require an \code{Integer} object to assign, but the expression
\code{7} returns an \code{int} primitive.  The compiler automatically
boxes the primitive, rendering the above code equivalent to
%
\begin{verbatim}
Integer i = Integer.valueOf(7);
\end{verbatim}

Autoboxing also applies in other contexts, such as loops, where
we may write
%
\begin{verbatim}
int[] ns = new int[] { 1, 2, 3 };
for (Integer n : ns) { ... }
\end{verbatim}
%
Each member of the array visited in the loop body is autoboxed.
Boxing is relatively expensive in time and memory, and should be
avoided where possible.

The Java compiler also carries out auto-unboxing, which allows things like
%
\begin{verbatim}
int n = Integer.valueOf(15);
\end{verbatim}
%
This is equivalent to using the \code{intValue()} method,
%
\begin{verbatim}
Integer nI = Integer.valueOf(15);
int n = nI.intValue();
\end{verbatim}


\subsubsection{Number Base Class}

Conveniently, all of the numerical classes extend the abstract class
\code{Number}.  The class \code{Number} simply defines all of the
get-value methods, \code{byteValue()}, \code{shortValue()}, ...,
\code{doubleValue()}, returning the corresponding primitive type.
Thus we can call \code{doubleValue()} on an \code{Integer} object; the
return value is equivalent to casting the \code{int} returned by
\code{intValue()} to a \code{double}.



\section{Arrays}

Arrays in Java are objects.  They have a fixed size that is determined
when they are constructed.  Attempts to set or get objects out of
range raises an \code{IndexOutOfBoundsException}.  This is much better
behaved than C, where out of bounds indexes may point into some other
part of memory that may then be inadvertently read or corrupted.

In addition to the usual object header, arrays also store an integer
length in 4 bytes.  After that, they store the elements one after the
other.  If the values are \code{double} or \code{long} values or
uncompressed references, there is an additional 4 bytes of padding
after the length so that they start on 8-byte boundaries.


\section{Synchronization}

LingPipe follows to fairly simple approaches to synchronization, which
we describe in the next two sections.  For more information on
synchronization, see Goetz et al.'s definitive book {\it Java
  Concurrency in Practice} (see \refappendix{java-libs} for a full
reference).

\subsection{Immutable Objects}

Wherever possible, instances of LingPipe classes are immutable.  After
they are constructed, immutable objects are completely thread safe in
the sense of allowing arbitrary concurrent calls to its methods.

\subsubsection{(Effectively) Final Member Variables}

In LingPipe's immutable objects, almost all member variables are
declared to be final or implemented to behave as if they are declared
to be final.  

Final objects can never change.  Once they are set in the constructor,
they always have the same value.

LingPipe's effectively final variables never change value.  In this
way, they display the same behavior as the hash code variable in the
implementation of Java's \code{String}.  In the source code for
\code{java.lang.String} in Java Standard Edition version 1.6.18,
authored by Lee Boynton and Arthur van Hoff, we have the following
four member variables:
%
\begin{verbatim}
private final char value[];
private final int offset;
private final int count;
private int hash;
\end{verbatim}
%
The final member variables \code{value}, \code{offset}, and
\code{count} represent the character slice on which the string is
based.  This is not enough to guarantee true immutability, because the
values in the value array may change.  The design of the \code{String}
class guarantees that they don't.  Whenever a string is constructed,
its value array is derived from another string, or is copied.  Thus no
references to \code{value} ever escape to client code.

The \code{hash} variable is not final.  It is initialized lazily when
the \code{hashCode()} method is called if it is has not already been
set.
%
\begin{verbatim}
public int hashCode() {
    int h = hash;
    if (h == 0) {
        int off = offset;
        char val[] = value;
        int len = count;
        for (int i = 0; i < len; i++) {
            h = 31*h + val[off++];
        }
        hash = h;
    }
    return h;
}
\end{verbatim}
%
Note that there is no synchronization.  The lack of synchronization
and public nature of \code{hashCode()} exposes a race condition.
Specifically, two threads may concurrently call \code{hashCode()} and
each see the hash code variable \code{hash} with value 0.  They would
then both enter the loop to compute and set the variable.  This is
safe because the value computed for \code{hash} only depends on
the truly final variables \code{offset}, \code{value}, and \code{count}.


\subsubsection{Safe Construction}

The immutable objects follow the safe-construction practice of not
releasing references to the object being constructed to escape during
construction.  Specifically, we do not supply the \code{this} variable
to other objects within constructors.  Not only must we avoid sending
a direct reference through \code{this}, but we can also not let inner
classes or other data structures containing implicit or explicit
references to \code{this} to escape.


\subsection{Read-Write Synchronization}\label{section:java-read-write-synchronization}

Objects which are not immutable in LingPipe require read-write
synchronization.  A read method is one that does not change the
underlying state of an object.  In an immutable object, all methods
are read methods.  A write method may change the state of an object.

Read-write synchronization allows concurrent read operations, but ensures
that writes do not execute concurrently with reads or other writes.

\subsubsection{Java's Concurrency Utilities}

Java's built-in \code{synchronized} keyword and block declarations use
objects for simple mutex-like locks.  Non-static methods declared to
be synchronized use their object instance for synchronization, whereas
static methods use the object for the class.  

In the built-in \code{java.util.concurrent} library, there are a
number of lock interfaces and implementations in the subpackage
\code{locks}.

The \code{util.concurrent} base interface \codeVar{Lock} generalizes Java's
built-in \code{synchronized} blocks to support more fine-grained forms
of lock acquisition and release, allowing operations like acquisitions
that time out.  

The subinterface \code{ReentrantLock} adds reentrancy to the basic
lock mechanism, allowing the same thread to acquire the same lock more
than once without blocking.

The \code{util.concurrent} library also supplies an interface
\codeVar{ReadWriteLock} for read-write locks.  It contains two
methods, \code{readLock()} and \code{writeLock()}, which return
instances of the interface \codeVar{Lock}.

There is only a single implementation of \codeVar{ReadWriteLock}, in
the class \code{ReentrantReadWriteLock}.  Both of the locks in the
implementation are declared to be reentrant (recall that a subclass
may declare more specific return types than its superclass or the
interfaces it implements).  More specifically, they are declared to
return instances of the static classes \code{ReadLock}
and \code{WriteLock}, which are nested in {ReentrantReadWriteLock}.

The basic usage of a read-write lock is straightforward, though
configuring and tuning its behavior is more involved.  We illustrate
the basic usage with the simplest possible case, the implementation of
a paired integer data structure.  The class has two methods, a paired
set and a paired get, that deal with two values at once.  

Construction simply stores a new read write lock, which itself
never changes, so may be declared final.
%
\codeblock{SynchedPair.1}
%

The set method acquires the write lock before setting both values.
%
\codeblock{SynchedPair.2}
%
Note that it releases the lock in a \code{finally} block.  This is
the only way to guarantee, in general, that locks are released even if
the work done by the method raises an exception.

The get method is implemented with the same idiom, but using the
read lock.
%
\codeblock{SynchedPair.3}
%
Note that it, too, does all of its work after the lock is acquired,
returning an array consisting of both values.  If we'd tried to have
two separate methods, one returning the first integer and one
returning the second, there would be no way to actually get both
values at once in a synchronized way without exposing the read-write
lock itself.  If we'd had two set and two get methods, one for each
component, we'd have to use the read-write lock on the outside to
guarantee the two operations always happened together.  

We can test the method with the following command-line program.
%
\codeblock{SynchedPair.4}
%
It constructs a \code{SynchedPair} and assigns it to the variable
\code{pair}, which is declared to be final so as to allow its usage in
the anonymous inner class runnable.  It then creates an array of threads,
setting each one's runnable to do a sequence of paired sets where the
two values are the same.  It tests first that the values it reads are
coherent and writes out an error message if they are not.  The runnables
all yield after their operations to allow more interleaving of the threads
in the test.  After constructing the threads with anonymous inner class
\code{Runnable} implementations, we start each of the threads and join them
so the message won't print and JVM won't exit until they're ll done.

The Ant target \code{synched-pair} runs the demo.  
%
\commandlinefollow{ant synched-pair}
\begin{verbatim}
OK
\end{verbatim}
%
The program takes about 5 seconds to run on our four-core notebook.


