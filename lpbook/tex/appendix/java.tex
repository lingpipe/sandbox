\chapter{Java Basics}\label{chapter:java}

In this appendix, we go over some of the basics of Java that are
particularly relevant for text and numerical processing.   We
cover most of the basics in the corresponding appendix in the
companion volume, {\it Text Processing with Java}.  It contains
introduction to Java primitive types and their bit-level
representations, arrays, objects, and synchronization.



\section{Generating Random Numbers}\label{section:java-random}

Java has a built-in pseudorandom number generator.  It is implemented
in the class \code{Random} in package \code{java.util}.  

\subsection{Generating Random Numbers}

Given an instance of \code{Random} (see the next section for
construction details), we are able to use it to generate random
outcomes of several primitive types. 

\subsubsection{Discrete Random Numbers}

The method \code{nextBoolean()} returns a random \code{boolean}n
value, \code{nextInt()} a random \code{int}, and \code{nextLong()} a
random long.  The algorithms generating these values are designed to
assign equal probabilities to all outcomes.  

The method \code{nextInt(int)} is particularly useful in that it
generates a value between 0 (inclusive) and its argument (exclusive).

Byte are generated using the C-like method \code{nextBytes(byte[])},
which fills the specified byte array with random bytes.  

\subsubsection{Floating Point Numbers}

As with all things floating point, random number generation is tricky.
The method \code{nextFloat()} and \code{nextDouble()} return values
beteween 0.0 (inclusive) and 1.0 (exclusive).  This is, in part, to
compensate for the uneven distribution of values across the bit-level
representations.  When values are large, there are larger gaps between
possible floating point representations.  The algorithms used are not
quite generating random bit-level representations; they are documented
in the class's Javadoc.

There is also a method \code{nextGaussian()} that generates a draw
from a unit normal, $\distro{Norm}(0,1)$, distribution.


\subsection{Seeds and Pseudorandomness}

A random number generator is constructed using a base costructor
\code{Random(long)}.  The argument is called the seed of the random
number generator.

The behavior of \code{Random} is fully determined by the seed.  A
generator with deterministic behavior like this is said to be
\stringmention{pseudorandom}.  By setting the seed, we are able to
generate the same ``random'' results whenever we run the algorithm.
This is immensely helpful for debugging purposes.

If you want true randomization, just use the zero-argument constructor
\code{Random()}.  This constructor combines the time obtained from
Java's \code{System.nanoTime()}) and a changing internal key in an
attempt to generate unique seeds each time it is called.

We sometimes use the following pattern to enable us a fresh random
seed for each run, but still enable reproducible results.
%
\codeblock{FragmentsJava.1}
%
By printing out the seed, we can reproduce any results from a run.
This is especially useful for random algorithms (that is, algorithms
that makes choices probabilistically, such as Gibbs samplers or
K-means clusterers).  

LingPipe is set up so that randomized algorithms take an explicit
\code{Random} argument, so that results may be reproduced.


\subsection{Alternative Randomization Implementations}

Because the \code{Random} class is not final, it is possible to plug
in different implementations.  In fact, \code{Random} was designed for
extension by implementing most of its work through the protected
\code{next(int)} method.  This method is used to define the other
random number generators, sometimes being called more than once in a
method.

\subsection{Synchronization}

Although not externally documented, there is a great deal of careful
synchronization underlying the methods in \code{Random}.  As of Java
1.6, the various \code{next\codeVar{X}()} methods (other than
\code{nextGaussian()}) use a scalable compare-and-set (CAS) strategy
through Java's built-in \code{AtomicLong} class.  The
\code{nextGaussian()} method uses its own synchronization.

This synchronization happens in the \code{next(int)} method, so
classes that extend \code{Random} need to be designed with thread
safety in mind. 

Even with its design for liveness, the Javadoc itself recommends using
a separate random number generator in each thread if possible to
reduce even the minimal lock contention imposed by the atomic long's
compare-and-set operation.

\subsection{The \code{Math.random()} Utility}

The \code{Math} utility class in the built-in \code{java.lang} package
provides a simple means of generating double values.  The utility
method \code{Math.random()} that returns the result of calling
\code{nextDouble()} on a random number generator, thus returning a
value between 0.0 (inclusive) and 1.0 (exclusive).  The random number
generator is lazily initialized thread safely so that calls to the
method are thread safe.









