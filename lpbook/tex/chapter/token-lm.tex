\chapter{Tokenized Language Models}\label{chap:tok-lm}

In addition to the character-level language models described
in \refchap{char-lm}, LingPipe provides token-level language
models.  These may be configured to predict arbitrary texts
including whitespace, or may be used just for predicting sequences
of tokens.  

\section{Applications of Tokenized Language Models}

Like character language models, some applications for which were
discussed in \refsec{char-lm-apps}, tokenized language models may be
used as the basis for for classification.  An additional applciation
of tokenized language models is for the discovering of new or
interesting phrases or collocations.  For significant phrase
extraction, a background language model characterizes a fixed
comparison text, such as all the text collected from newspapers in a
year, and a foreground model characterizes some other text, such as
the text collected this week; differential analysis is then done to
find phrases that appear more frequently than one would expect in the
foreground model, thus modeling a kind of ``hot topic'' detection.


\section{Token Language Model Interface}

In addition to the interfaces discussed in \refsec{char-lm-interfaces},
there is an interface for tokenized language models.

\subsection{Tokens: \code{LanguageModel.Tokenized}}

For language models based on tokens, the interface
\code{LanguageModel.Tokenized} supplies a method for
computing the probability of a sequence of tokens,
%
\begin{verbatim}
double tokenLog2Probability(String[] toks, int start, int end);
\end{verbatim}
%
This returns the log (base 2) probability of the specified slice of
tokens.  As usual, the start is inclusive and the end exclusive, so
that the tokens used are \code{toks[start]} to \code{toks[end-1]}.
There is also a method to return the linear probability, which is just
the result of exponentiating the log probability.  

Unlike character-based probabilities, this token probability only
counts the probability of the tokens, not the whitespaces (or other
non-token characters) surrounding them.  

A further disconnect for tokenized language models is that they may
use a tokenizer that reduces the signal, conflating some of the
inputs.  In that case, what is being modeled is the probability of
the sequence of tokens, not the raw character sequence.  For example,
any two character sequences producing the same tokens will produce
the same token probability even if they are different strings.
