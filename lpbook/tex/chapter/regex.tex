\chapter{Regular Expressions}\label{chapter:regex}

Generalized regular expressions, as implemented in Java and all the
scripting languages like Perl and Python, provide a general means for
describing spans of Unicode text.  Given a regular expression (regex)
and some text, basic regex functionality allows us to test whether a
string matches the regex or find non-overlapping substrings of a
string matching the regex.

\section{Matching Whole Texts}

Despite the rich functionality of regexes, Java's
\code{java.util.regex} package contains only two classes,
\code{Pattern} and \code{Matcher}.  An instance of \code{Pattern}
provides a stateless representation of a regular expression.  An
instance of \code{Matcher} represents the state of the matching
of a regular expression against a string.

It's easiest to start with an example of using a regular expression
for matching, which we wrote as a
\code{main()} method in the class \code{RegexMatch}, the work of
which is done by
%
\codeblock{RegexMatch.1}
%
First, we read the regex from the first command-line argument, then
the text from the second argument.  We then use the regular expression
to compile a pattern, using the static factory method
\code{pattern.compile()}.  This pattern is reusable.  We next
create a matcher instance, using the method \code{matcher()} on the
pattern we just created.  Finally, we assign a boolean variable
\code{matches} the value of calling the method \code{matches()} 
on the matcher.  And then we print out the result.


Regular expressions may consist of strings, in which case they
simply carry out exact string matching.  For example, the regex
\code{aab} does not match the string \stringmention{aabb}.  
There is an Ant target \code{regex-match} which feeds the command-line
arguments to our program.  For the example at hand, we have
%
\commandlinefollow{ant -Dregex="aab" -Dtext="aabb" regex-match}
\begin{verbatim}
Matches=false
\end{verbatim}
%
On the other hand, the regex \code{abc} does match
the string \stringmention{abc}.
%
\commandlinefollow{ant -Dregex="abc" -Dtext="abc" regex-match}
\begin{verbatim}
Matches=true
\end{verbatim}


\section{Finding Substrings}

The second main application of regular expressions is to find
substrings of a string that match the regular expression.  The
main method in our class \code{RegexFind} illustrates this.
%
\codeblock{RegexFind.1}
%



