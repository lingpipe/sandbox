\chapter{Symbol Tables}\label{chap:symbol-tables}

Strings are unwieldy objects --- they consume substantial memory and
are slow to compare, making them slow for sorting and use as hash
keys.  In many contexts, we don't care about the string itself {\it
  per se}, but only its identity.  For instance, document search
indexes like Lucene consider only the identity of a term.  Naive Bayes
classifiers, token-based language models and clusterers like K-means
or latent Dirichlet allocation do not care about the identity of
tokens either.  Even compilers for languages like Java do not care
about the names of variables, just their identity.

In applications where the identity of tokens is not important, the
traditional approach is to use a symbol table.  A symbol table
provides integer-based representations of strings, supporting mappings
from integer identifiers to strings and from strings to identifiers.
The usage pattern is to add symbols to a symbol table, then reason
with their integer representations.  So many of LingPipe's classes
operate either explicitly or implicitly with symbol tables, that
we're breaking them out into their own chapter.

\section{The \code{SymbolTable} Interface}

The interface \code{SymbolTable} in \code{com.aliasi.symbol} provides
a bidirectional mapping between string-based symbols and numerical
identifiers. 

\subsection{Querying a Symbol Table}

The two key methods are \code{symbolToID(String)}, which returns the
integer identifier for a string, or -1 if the symbol is not in the
symbol table.  The inverse method \code{idToSymbol(int)} returns the
string symbol for an identifier; it throws an
\code{IndexOutOfBoundsException} if the identifier is not valid.  

The method \code{numSymbols()} returns the total number of symbols
stored in the table as an integer (hence the upper bound on the size
of a symbol table is \code{Integer.MAX\_VALUE}).

\subsubsection{Modifying a Symbol Table}

Like the \code{Collection} interface in \code{java.util}, LingPipe's
symbol table interface specifies optional methods for modifying the
symbol table.  In concrete implementations of \code{SymbolTable},
these methods either modify the symbol table or throw an
\code{UnsupportedOperationException}.

The method \code{clear()} removes all the entires in the symbol table.
The method \code{removeSymbol(String)} removes the specified symbol
from the symbol table, returning its former identifier or -1 if it
was not in the table.  

The method \code{getOrAddSymbol(String)} returns the identifier
for the specified string, adding it to the symbol table if necessary.



\section{The \code{MapSymbolTable} Class}\label{section:symbol-map-symbol-table}

Of the two concrete implementations of \code{SymbolTable} provided in
LingPipe, the map-based table is the most flexible.  

\subsection{Demo: Basic Symbol Table Operations}

The class \code{SymbolTableDemo} contains a demo of the basic symbol
table operations using a \code{MapSymbolTable}.  The \code{main()}
method contains the following statements (followed by prints).
%
\codeblock{SymbolTableDemo.1}
%
The first statement constructs a new map symbol table.  Then three
symbols are added to it using the \code{getOrAddSymbol()} method, with
the resulting symbol being assigned to local variables.  Note that
\code{"foo"} is added to the table twice.  This method adds the
symbol if it doesn't exist, so the results returned are always
non-negative.

Next, we use the \code{symbolToID()} methods to get identifiers
and also assign them to local variables.  This method does not
add the symbol if it doesn't already exist.  Finally, we retrieve
the number of symbols.

The Ant target \code{symbol-table-demo} runs the command.
%
\commandlinefollow{ant symbol-table-demo}
\begin{verbatim}
id1=0          s1=foo
id2=1          s2=bar
id3=0          n=2
id4=0
id5=-1
\end{verbatim}
%
Note that \code{id1} and \code{id3} are the same, getting the
identifier 0 for string \code{"foo"}.  This is also the value returned
by the \code{symboltoID()} method, as shown by the value of
\code{id4}.  Becase we used \code{symbolToID("bazz")} rather than
\code{getOrAddSymbol("bazz")}, the value assigned to \code{id5} is -1,
the special value for symbols not in the symbol table.  

The \code{idToSymbol()} method invert the mappings as shown by their
results.  Finally, note there are 2 symbols in the symbol table when
we are done, \code{"foo"} and \code{"bar"}.


\subsection{Demo: Removing Symbols}

The class \code{RemoveSymbolsDemo} shows how the map symbol table
can be used to modify a symbol table.  Its \code{main()} method is
%
\codeblock{RemoveSymbolsDemo.1}
%
We first add the symbols \code{"foo"} and \code{"bar"}, then query the
number of symbols.  Then we remove \code{"foo"} using the method
\code{removeSymbol()} and query the numbe of symbols again.  Finally,
we clear the symbol table of all symbols using the method \code{clear()},
and query a final time.  

We can run the demo with the Ant target \code{remove-symbol-demo},
%
\commandlinefollow{ant remove-symbol-demo}
\begin{verbatim}
n1=2     n2=1     n3=0
\end{verbatim}

By removing symbols, we may create holes in the run of identifiers.
For instance, the identifier 17 may have a symbol but identifier 16
may not.  This needs to be kept in mind for printing or enumerating
the symbols, which is why there are symbol and identifier set views
in a map symbol table (see below).

\subsection{Additional \code{MapSymbolTable} Methods}

The implementation \code{MapSymbolTable} provides several methods not
found in the \code{SymbolTable} interface.  


\subsubsection{Boxed Methods}

The map symbol table duplicates the primitive-based \code{SymbolTable}
methods with methods accepting and returning \code{Integer} objects.
This avoids round trip automatic unboxing and boxing when integrating
symbol tables with collections in \code{java.util}.  

The methods \code{getOrAddSymbolInteger(String)},
\code{symbolToIDInteger(String)} return integer objects rather than
primitives.  The method \code{idToSymbol(Integer)} returns the symbol
corresponding to the integer object identifier.


\subsubsection{Symbol and Identifier Set Views}

The method \code{symbolSet()} returns the set of symbols stored in the
table as a set of strings, \code{Set<String>}.  The method
\code{idSet()} returns the set of IDs stored in the table as a set of
integer objects, \code{Set<Integer>}.  

These sets are views of the underlying data and as such track the data
in the symbol table from which they originated.  As the symbol table
changes, so do the sets of symbols and sets of identifiers.  

These sets are immutable.  All of the collection methods that might
modify their contents will throw unsupported operation exceptions.

Providing immutable views of underlying objects is a common interface
idiom.  It allows the object to maintain the consistency of the two
sets while letting the sets reflect changes in the underlying symbol
table.

We provide a demo in \code{SymbolTableViewsDemo}.  The \code{main()} method
uses the following code.
%
\codeblock{SymbolTableViewsDemo.1}
%
The two sets are assigned as soon as the symbol table is constructed.
We then print out the set of symbols and identifiers.  After that, we
add the symbols \code{"foo"} and \code{"bar"} and print again.  
Finally, we remove \code{"foo"} and print for the last time.

The Ant target \code{symbol-table-views} runs the demo,
%
\commandlinefollow{ant symbol-table-views-demo}
\vspace*{2pt} % don't know why I need this here -- variable spacing, I think
\begin{verbatim}
symbolSet=[] idSet=[]
symbolSet=[foo, bar] idSet=[0, 1]
symbolSet=[bar] idSet=[1]
\end{verbatim}
%
As advertised, the values track the underlying symbol table's values.


\subsubsection{Unmodifiable Symbol Table Views}

In many cases in LingPipe, models will guard their symbol tables
closely.  Rather than returning their actual symbol table, they return
an unmodifiable view of the entire table.  The method
\code{unmodifiableView(SymbolTable)} returns a symbol table that tracks
the specified symbol table, but does not support any of the modification
methods.  

This method employs the same pattern as the generic static utility
method \code{<T>unmodifiableSet(Set<?~extends~T>)} in the utility
class \code{Collections} from the built-in package \code{java.util};
it returns a value of type \code{Set<T>} which tracks the specified
set, but may not be modified.  This collection utility is used under
the hood to implement the symbol and identifier set views discussed in
the previous section.

\subsubsection{Thread Safety}

Map symbol tables are not thread safe.  They require read-write
synchronization, where all the methods that may modify the underlying
table are considered writers.

\subsubsection{Serialization}

Map symbol tables are serializable.  They deserialize to a
\code{MapSymbolTable} with the same behavior as the one serialized.
Specifically, the identifiers do no change.



\section{The \code{SymbolTableCompiler} Class}

The \code{SymbolTableCompiler} implementation of \code{SymbolTable} is
a legacy class used in some of the earlier LingPipe models.  We could
have gotten away without the entire \code{com.aliasi.symbol} package
and just dropped \code{MapSymbolTable} into \code{com.aliasi.util}.
Instead, we abstracted an interface and added a parallel
implementation in order to maintain backward compatibility at the
model level.

Once a symbol is added, it can't be removed.  Symbol table
compilers do not support the \code{remove()} or \code{clear()}
methods; these methods throw unsupported operation exceptions.

\subsection{Adding Symbols}

The symbol table compiler implements a very unwieldy usage pattern.
It begins with a no-argument constructor.  The usual
\code{getOrAddSymbol(String)} method from the \code{SymbolTable}
interface will throw an unsupported operation exception.  Instead, the
method \code{addSymbol(String)} is used to add symbols to the table,
returning \code{true} if the symbol hasn't previously been added.

The really broken part about this class is that it doesn't actually
implement \code{SymbolTable} properly until after its been compiled.

\subsection{Compilation}

A symbol table compiler implements LingPipe's \code{Compilable}
interface (see \refsec{io-compilable}).  

\subsection{Demo: Symbol Table Compiler}

The demo class \code{SymbolTableCompilerDemo} implements a \code{main()}
method demonstrating the usage of a symbol table compiler.  The method
begins by setting up the compiler and adding some symbols
%
\codeblock{SymbolTableCompilerDemo.1}
%
It sets some variables that result from attempting to get the symbols
again at this point.

Next, we compile the symbol table using the static utility method
\code{compile()} from LingPipe's \code{AbstractExternalizable} class.
We use the original symbol table then the compiled version to convert
symbols to identifiers.

The Ant target \code{symbol-table-compiler-demo} runs the command.
%
\commandlinefollow{ant symbol-table-compiler-demo}
\begin{verbatim}
n1=-1     n2=-1     n3=-1
n4=1     n5=0     n6=-1
n7=1     n8=0     n9=-1

st.getClass()=class com.aliasi.symbol.CompiledSymbolTable
\end{verbatim}
%
The symbols are all unknown before compilation, returning -1 as their
identifier.  After compilation, both the compiler and the compiled
version agree on symbol identifiers.  We also print the class for the
result, which is the package-protected class
\code{CompiledSymbolTable}.  This class does implement
\code{SymbolTable}, but the get-or-add method is not supported.

One nice property of the class is that we are guaranteed to have
identifiers numbered from 0 to the number of symbols minus 1.  We
are also guaranteed to get symbols stored in alphabetical order. 
Note that the symbol \code{"bar"} is assigned identifier 0 even
though it was added after the symbol \code{"foo"}.

Be careful with this class.  If more symbols are added, then the table
is compiled again, identifiers will change.  This is because it relies
on an underlying sorted array of symbols to store the symbols and look
up their identifiers.

\subsection{Static Factory Method}

The static utility method \code{asSymbolTable(String[])} constructs a
symbol table from an array of symbols.  It will throw an illegal
argument exception if there are duplicates in the symbol table.
Perhaps a \code{SortedSet} would've been a better interface here.


