\chapter{Tagging with Hidden Markov Models}\label{chap:hmm}

Hidden Markov models (HMMs) provide a generative statistical model of
tagging (see \refchap{tagging} for a general discussion of taggers).
HMMs may be thought of as a sequentially structured generalization of
naive Bayes classifiers (see \refchap{naive-bayes}).  Like language
models (see \refchap{char-lm}), their sequential structure is
Markovian.  What this means is that we only use a finite window of
past history in order to predict the tag for a given token.  Unlike
language models, the state depends on previous tags, not on previous
tokens.  These tags are not directly observable, which is the
etymology of the term ``hidden'' in their name.

As with classifiers, many tasks that do not superficially appear
to involve tagging, such as named-entity chunking, can be reduced
to tagging problems.


