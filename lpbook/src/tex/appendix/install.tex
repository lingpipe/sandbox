\mychapter{Installation}{INSTALLATION}

\firstchar{T}his appendix explains how to install everything you'll
ned to work with the code in this book.


\mysection{Java Standard Edition 6}

\noindent
The programs in this book are based on Java Standard Edition Version
6.  At the time of writing, this is the only version of Java being
publicly supported by Sun/Oracle.

In order to run the Java programs and their
associated libraries, you will need a Java runtime environment (JRE),
which contanis a Java virtual machine (JVM).  In order to compile Java
programs, you will need a Java development kit (JDK).

Java is available in 32-bit and 64-bit versions.  The 64-bit version
is required to allocate JVMs with heaps larger than 1.5 or 2 gigabytes
(the exact maximum for 32-bit Java depends on the platform).  

Licensing terms for the JRE are at
%
\begin{quote}
\hrefurl{http://www.java.com/en/download/license.jsp}{}
\end{quote}
and for the JDK at
%
\begin{quote}
\hrefurl{http://java.sun.com/javase/6/jdk-6u20-license.txt}
\end{quote}

Java is available for the Windows, Linux, and Solaris operating
systems in both 32-bit and 64-bit versions from
%
\begin{quote}
\hrefurl{http://java.sun.com/javase/downloads/index.jsp}
\end{quote}
%
The Java JDK and JRE are included as part of Mac OS X.  Updates are
available through
%
\begin{quote}
\hrefurl{http://developer.apple.com/java/download/}
\end{quote}
%
Java is updated regularly and it's worth having the latest version.
Updates include bug fixes and often include performance enhancements,
some of which can be quite substantial.

Java must be installed on the operating system so that shell commands
will execute it.  

\begin{quote}
{\small
\ttfamily
{\it\ttfamily > java -version}
\\
java version "1.6.0\_18"
\\
Java(TM) SE Runtime Environment (build 1.6.0\_18-b07)
\\
Java HotSpot(TM) 64-Bit Server VM (build 16.0-b13, mixed mode)
}
\end{quote}

\mysection{Ant}

\noindent
The build files for the code samples and examples are provided as Ant
files.  Ant build files are coded in XML and use easy to read element
and attribute names for build targets.  Unlike other build systems
such as Make or Maven which try to resolve dependencies and determine
if targets are up to date, Ant always executes all of a target,
including all of the subtargets.

Ant is is an Apache project, and as such, is subject to the Apache license,
\begin{quote}
\hrefurl{http://ant.apache.org/license.html}
\end{quote}

Ant is available from 

\begin{quote}
\hrefurl{http://ant.apache.org/}
\end{quote}

Ant must be 


\mysection{LingPipe}

\noindent
LingPipe may be downloaded from its web site:

\begin{quote}
\hrefurl{http://alias-i.com/lingpipe/}
\end{quote}


