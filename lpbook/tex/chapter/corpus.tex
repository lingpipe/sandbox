\chapter{Handlers, Parsers, and Corpora}\label{chapter:corpus}

LingPipe uses a parser-handler pattern for parsing objects out of
files and processing them.  

The basic usage is very much like that of
the SAX parser for XML built into Java.  A parser is created for the
format in which the data objects are represented.  Then a handler for
the data objects is created and attached to the parser.  At this
point, the parser may be used to parse objects from a file or other
input specification, and all objects found in the file will be passed
off to the handler for processing.

Many of LingPipe's modules provide online training through handlers.
For instance, the dynamic language models implement character sequence
handlers.  

LingPipe's batch training modules require an entire corpus for
training.  A corpus represents an entire data set with a built-in
training and test division.  A corpus has methods that take handlers
and deliver to them the training data, the test data, or both.

\section{Object Handlers}

The \code{ObjectHandler<E>} interface is used for handling data
objects of type \code{E}.%
%
\footnote{The \code{ObjectHandler} interface is defined to extend the
  legacy marker interface \code{Handler}.  After LingPipe 4, there is
  really no reason to use the \code{Handler} interface directly.  It
  only remains to preserve backward compatibility.}
%
It specifies a single method, \code{handle(E)}, which processes a
single data object of type \code{E}.

\subsection{Demo: Counting Characters}

We define a simple text handler in the demo class
\code{CountingTextHandler}.  It's so simple that it doesn't
even need a constructor.
%
\codeblock{CountingTextHandler.1}
%
We declare the class to implement \code{ObjectHandler<CharSequence>}.
This contract is satisfied by our implementation of the
\code{handle(CharSequence)} method.  That method just increments the
counters for number of characters and sequences it's seen.  

The \code{main()} method in the same class implements a simple
command-line demo.  It creates a conting text handler, then calls is
handle method with two strings (recall that \code{String} implements
\code{CharSequence}).
%
\codeblock{CountingTextHandler.2}

The Ant target \code{counting-text-handler} calls the main method,
which assumes no command-line arguments.  
%
\commandlinefollow{ant counting-text-handler}
\begin{verbatim}
# seqs=2 # chars=18
\end{verbatim}
%
The output comes from calling the handler's \code{toString()}
method, which is defined to print the number of sequences
and characters.





\section{Parsers}




\subsection{Demo: 20 Newsgroups Classification Corpus}\label{section:20-news-corpus}

