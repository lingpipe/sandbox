\chapter{Further Reading}\label{appendix:reading}

\firstchar{I}n order to fully appreciate contemporary approaches to natural language
processing requires three fairly broad areas of expertise:
linguistics, statistics, and algorithms.  We don't pretend to
introduce these areas in any depth in this book.  Instead, we
recommend the following textbooks, sorted by area, from among the many
that are currently available.

\section{Unicode}

Given that we're dealing with text data, and that Java uses Unicode
internally, it helps to be familiar with the specification.  Or at
least know where to find more information.  The actual specification
reference book is quite readable.  The latest available in print is
for version 5.0,
%
\begin{itemize}
\bibanno{Unicode Consortium, The.  
{\it The Unicode Standard 5.0}.  
Addison-Wesley.}
{This is the final word on the unicode standard, and
quite readable.}
\end{itemize}
%
You can also find the latest version (currently 5.2) online at
%
\urldisplay{http://unicode.org}
%
This includes all of the code tables for the latest version, which are
available in PDF files or through search, at 
%
\begin{quote}
\hrefurl{http://unicode.org/charts/}
\end{quote}


\section{Java}\label{section:reading-java}

\subsection{Overviews of Java}
%
For learning Java, we recommend the following three books pitched at
beginners and intermediate Java programmers,
%
\begin{itemize}
\bibanno{Arnold, Ken, James Gosling and David Holmes.  2005.
{\it The Java Programming Language}, 4th Edition.  Prentice-Hall.}
{Gosling invented Java, and this is the ``official'' introduction to the
language.  It's also very good.}
%
\bibanno{Bloch, Joshua. 2008. {\it Effective Java}, 2nd Edition.  Prentice-Hall.}
{Bloch was one of the main architects of Java from
version 1.1 forward.  This book is about taking Java program design
to the next level.}
%
\bibanno{Fowler, Martin, Kent Beck, John Brant, William Opdyke, and Don Roberts.
1999.
{\it Refactoring: Improving the Design of Existing Code}.
Addison-Wesley.}
{Great book on making code better, both existing
code and code yet to be created.}
\end{itemize}
%

\subsection{Books on Java Libraries and Packages}\label{appendix:java-libs}

\noindent
There is a substantial collection of widely used libraries for Java.
Here are reference books for the Java libraries we use in this book.
%
\begin{itemize}
%
\bibanno{
Loughran, Steve and Erik Hatcher. 2007
{\it Ant in Action}.  Manning.}
{This is technically the second edition of their 2002
book, {\it Java Development with Ant}.  Both authors are committers for
the Apache Ant project.}
%
\bibanno{Tahchiev, Peter, Felipe Leme,
Vincent Massol, and Gary Gregory. 2010.
{\it JUnit in Action}, Second Edition.  Manning.}
{A good overview for JUnit.  Should be up-to-date with
the latest attribute-driven versions, unlike either Kent Beck's book
(he's the author of Ant).  As much as we liked {\it The Pragmatic
Programmer}, Hunt and Thomas's book on JUnit is not particularly useful.}
%
\bibanno{McCandless, Michael, Erik Hatcher and Otis Gospodneti\'c.
2010.
{\it Lucene in Action}, Second Edition. Manning.}
{A fairly definitive overview of the Apaceh Lucene
search engine by three project committers.}
%
\bibanno{Hunter, Jason and William Crawford. 2001. {\it Java Servlet Programming}, Second Edition. O'Reilly.}
{It only covers version 2.2, so it's a bit out of date
relative to recent standards, but the basics haven't changed, and
although there are more elementary introductions, this book is a very
good introduction to what servlets can do.  It also explains HTTP
itself quite clearly.  The first author is an Apache Tomcat developer
and was involved in developing the original servlet standard.}
%
\bibanno{Goetz, Brian, Tim Peierls, Joshua Bloch, Joseph Bowbeer, David Holmes, and Doug Lea. 2006. {\it
Java Concurrency in Practice}.  Addison-Wesley.}
{This fantastic book covers threading and concurrency
patterns and their application, along with implementations in Java's
\code{java.util.concurrent} package.  Doug Lea was the original author
of \code{util.concurrent} before it was brought into Java itself, and
his previous book, {\it Concurrent Programming in Java}, Second
Edition (1999; Prentice Hall), is wroth reading for a deeper theoretical
understanding of concurrency in Java.}
%
\bibanno{Hitchens, Ron. 2002.  {\it Java NIO}. O'Reilly.}
{A concise and readable introduction to the ``new'' I/O package, \code{java.nio} and
the concepts behind it, like buffers and multicast reads.}
%
\bibanno{McLaughlin, Brett and Justin Edelson. 2006. {\it Java and XML}, Third Edition. O'Reilly.}
{A good introduction to the set of tools available in
Java for parsing and generating XML with SAX and DOM, specifying structure
with DTDs and XML Schema, transforming XML with XSLT, as well as the general 
high-level libraries JAXP and JAXB.}
\end{itemize}


\subsection{General Java Books}

There are also general-purpose books about Java that do
not focus on particular classes or applications.
%
\begin{itemize}
%
\bibanno{Shirazi, Jack. 2003. {\it Java Performance Tuning}, 2nd
  Edition. O'Reilly} 
{Although already dated, it provides a good introduction to
  profiling and thinking about performance tuning.  The author and friends
  also run the web site \code{javaperformancetuning.com}, which is a
  monthly digest of Java performance tuning tips (and advertising).
  If you know C, start with Jon Bentley's classic, {\it Programming
  Pearls} (Addison-Wesley).}
%
\end{itemize}


\subsection{Practice Makes Perfect}

\noindent
Really, the only way to learn how to code is to practice.  Being a
good coder means two things: being fast and efficient (assuming
accuracy).  By accurate, we mean writing code that does what it's
supposed to do.  By fast, we mean the coding goes quickly.  By efficient,
we mean the resulting code is efficient.

For practice, we highly recommend the algorithm section of the
TopCoder contests.  TopCoder provides a complete online Java
environment, well thought out problems at varying specified levels of
difficulty, unit tests that check your submissions, and examples of
what others did to provide code-reading practice.
%
\begin{quote}
\hrefurl{http://www.topcoder.com/tc}
\end{quote}
%



\section{General Programming}

\noindent
We would also recommend a book about programming in general, 
especially for those who have not worked collaboratively with
a good group of professional programmers.
%
\begin{itemize}
\bibanno{Hunt, Andrew and David Thomas.
1999.
{\it The Pragmatic Programmer}.
Addison-Wesley.}
{Read this book and follow its advice, which is just as relevant now
as when it was written.  The best book on how to program
we know.  A competitor to books like Steve McConnell's {\it Code Complete} (2004)
and books on ``agile'' or ``extreme'' programming.}
%
\bibanno{Collins-Susman, Ben, Brian W.\ Fitzpatrick,
and C. Michael Pilato.
2010.
Collins-Sussman et al.'s
\booktitle{Version Control with Subversion}. Subversion 1.6.
CreateSpace.}
{This is the official guide to Subversion from the
authors.  It's also available online at
%
\urldisplay{http://svnbook.red-bean.com/}
}
\end{itemize}

