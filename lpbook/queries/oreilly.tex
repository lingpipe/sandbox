\documentclass{letter}
\setlength{\topmargin}{-0.925in}
\setlength{\textheight}{10.5in}
\usepackage[english]{babel}
\lefthyphenmin=2
\hyphenpenalty=200
\tolerance=50
\linespread{1.1}
\begin{document}
\begin{letter}{}
\opening{Dear O'Reilly editors,}

\emph{Text Processing in Java} focuses on the core Java classes and
technologies needed to process natural language text data.
This book is written for anyone producing, managing, or consuming text data.
As the amount of both public domain and proprietary unstructured text data
increases, so do opportunities for new applications.
Also increasing are opportunities for data corruption, in Japanese called \emph{mojibake}
(character problems),
in German \emph{Buchstabensalat} (letter salad),
and in Serbian \emph{dubre} (trash).

This book lays out the fundamentals of \emph{encoding}, \emph{decoding},
and \emph{processing} character data in Java.
It is precise, explicit, and provides copious examples
as well as a suite of programs that can be used 
as diagnostics for the forensic study of character data.
It starts with the internal representation of characters in Java as Unicode code points
and shows how these are encoded in UTF-8, -16, -32, and legacy character encodings.
It covers the classes and methods from the \texttt{java.io} and \texttt{java.nio} packages
that convert text data to raw bytes and back again.
It shows how to normalize character data using the ICU 
(International Components for Unicode) package
in order to ensure proper sorting, indexing, and data deduplication.
It provides the background needed to use Java's
localization and internationalization features.
The chapter on regular expressions shows how to
write regexes for Unicode character data and how
and when and which characters must be escaped.
Complex regular expressions can be used to create
ad-hoc parsers to munge text data from one format to another.
The chapter on text data and the web covers the HTTP protocol,
how to specify encodings, and how to parse, generate, and escape
character data in XML, HTML, and JSON formats using open-source APIs.
It concludes with chapters on Lucene and Solr,
the Apache search engine written in Java.
Each chapter comes with a set of executable programs
that are simple, self-contained, and take a
debug-by-\texttt{printf} approach in reporting the state of the inputs
before, during, and/or after processing.

This book compliments existing books on Java programming because
of its in-depth focus on the precise details of character data processing.
It would be a good companion to the O'Reilly books
\emph{Localization/Internationalization}, 
\emph{XML and Java},
\emph{Java Servlets},
and \emph{Mastering Regular Expressions}.
It provides up-to-date information on Java~7.
It pulls together information that is scattered across javadocs, blog posts, and threads on 
StackOverflow but strips out distracting, problem-specific details. 

\closing{Sincerely,}
\end{letter}
\end{document}
